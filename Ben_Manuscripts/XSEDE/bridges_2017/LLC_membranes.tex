subsection{How do water and ions move in nanostructured charged polymeric membranes?}
\subsubsection*{Background} 

The ability to design nanostructured membranes with concomitant high chemical
specificity is an important goal of the membrane separations
community.~\cite{Humplik2011} The ability to control
the nanoscale architecture of membranes will allow membranes to be 
designed precisely for the purpose of separating specific compounds.
~\cite{SmithRC1997,Zhu2006} The membranes being studied in this project
have pores sizes smaller than 1 nanometer, allowing size-selective 
filtration of small molecules and gases, as well as desalination of salt
water based on charge repulsion and size-exclusion of hydrated salt ions.

%By studying the system with molecular detail, we will learning more
%about the mechanism of water and small solute transport through the
%membrane pores. These investigations will occur in close collaboration
%with experimentalists in the Gin and Noble group at the University of
%Colorado at Boulder who are working on the membranes at the lab
%scale. 

The membrane system we propose to model in this project is composed of
lyotropic liquid crystals (LLCs) which self-assemble into hexagonally
packed, cylindrical pores with acidic groups facing toward the cylinder
center, complexed to counterions. Subsequent cross-linking to forms
a mechanically strong structure (Fig.~\ref{figure:membrane}).
~\cite{Zhou2003,Feng2014,Feng2016} The pores, at the level of resolution
obtained by SAXS and TEM, appear to be straight and uniform in size 
(Fig.~\ref{figure:membrane}).  This is very different from most 
commercially available membranes which have a pore size distribution
with tortuous pathways that reduce selectivity and flux respectively.
The ordered structure exhibited by LLC membranes makes them well suited 
for modeling using molecular dynamics simulations because the
system has a defined structure that can be studied in a reasonably sized 
unit cell.

Our initial systems were built by first parameterizing a single
monomer using the General Amber Force Field (GAFF). Because the
self-assembly process is long relative to times which we can simulate,
monomers were rotated into layers of six monomers and stacked into
cylinders to give a starting configuration close to where we expect
the system to settle at equilibrium. While a single lyotropic liquid
crystal consists of a small number of atoms (138), the entire unit
cell consists of about 66K atoms. This number increases to 
approximately 100K when water molecules are added to the system.

% BJC: The following is technically true although I've been creating
% specific configurations for testing purposes. 
In order to assemble the structures shown in Fig.~\ref{figure:membrane}, a
suitable monomer configuration needed to be chosen. Thermal annealing
of the monomer using GROMACS yielded many low energy configurations
close to what is expected. 

Initial simulations have been performed which energy minimize and
allow the system to stabilize over the course of 20 ns in vacuum.  The
final structure and trajectory are analyzed quantitatively by
measuring pore size, distance between pores, and distribution of
sodium ions in the pore. These measurements are compared to values
measured experimentally with SAXS and TEM imaging in order to validate
the final structure. SAXS and TEM images were taken using dry membrane
which justifies our initial study of the system in vacuum.

\begin{figure}[h]
\begin{center}
\begin{tabular}{c}
\includegraphics[width=8cm]{modeled.png}\\
\includegraphics[height=5cm]{membrane2.pdf}\\
\end{tabular}
\end{center}
\caption{(above) SAXS and TEM experimental data, with hypothesized
  membrane structure. (Below, left) Top view of atomistic simulation,
  with hydrophilic head groups in orange, counterions in blue, and
  hydrophobic chains in blue. (Below, right) Side view of atomistic
  simulation showing counterions in dark blue and other atoms in light
  blue. These simulations demonstrate the heterogeneity in the
  modeled systems.~\label{figure:membrane}}
\end{figure}

\subsubsection*{Summary of Scientific Discoveries}

Our understanding of the microscopic structure of this type of LLC
membrane has greatly increased with the aid of simulations run using
Bridges. 

Equilibration simulations of greater than 500 nanoseconds yield
stable membrane configurations with the expected HII phase 
morphology. Various methods have been developed to characterize
the equilibrated system. Generally, all equilibrium properties are
compared to experimental measurements. We validated two methods
for measuring ionic conductivity from atomistic simulations. Both 
methods require long simulations (at least 500 ns) in order to give
accurate statistics. The distance between pores is an important
structural parameter which we have measured using atomic coordinates
and by simulating X-ray diffraction (XRD) experiments, a relatively
undeveloped technique in MD for periodic systems such as ours. We 
have been able to generate structures which match experimental pore
spacings within reason.

The X-ray diffraction simulations also give detailed information
about membrane structure on the angstrom lengthscale. Our simulations
have produced two dimensional X-ray diffraction patterns that 
contain all major features present in experimental studies. Producing
a matching pattern was not trivial and resulted in the discovery of
two metastable states. The two states are defined by the degree of 
local order inside the pore regions. The state which we had initially
studied is characterized by a disordered pore region, however the X-ray
diffraction pattern does not match experiment. We altered the starting 
configuration so that the benzene rings in the head group of each
monomer were stacked in a parallel displaced configuration relative
to each other. The resulting X-ray diffraction pattern of the new
configuration, after equilibration, is a much closer match to experiment.
Additionally, we were able to explain the spots that appear (red arrows
in figure xx) contrary to how it was originally reported. The spots 
were assumed to be caused by the 40 degree tilt angle of the alkyl
tails with respect to the plane of each stacked monomer layer, a 
common feature of liquid crystal systems. However, we were able to 
produce the same spots using configurations with an average tilt angle
close to zero. Our most impactful finding remains as the discovery of 
two metastable states. This will be the subject of a publication which 
will be submitted in the coming months. In the future, we will conduct
free energy calculations that will help explain the difference between
the two states and and help predict what experimental conditions might
lead to each. 

The next step for our system is to solvate it with water. In parallel 
to the preceding work, we have worked to develop methods which will be
used to study the solvated system. While measuring ionic conductivity 
and running XRD simulations will work just the same, equilibration is 
non-trivial. We do not know exactly the equilibrium content of water 
or where the water is situated. While it is clear that most water should be in the hydrophilic pore region, simulations have shown that an 
appreciable amount of water can exist in the tail region near the 
slightly hydrophilic ester group. The best way to figure out how much 
water should be in the pore is to run very long equilibration simulationsand allow the simulations to tell us. Our current approach is to create
water baths at each face of the membrane and allow water to diffuse into
the membrane. We have learned that we can equilibrate a membrane with
water in 1000 nanoseconds. Studies of the hydrated, 'lyotropic', phase
will be the subject of a subsequent publication.

\subsubsection*{Proposed Experiments and Justification of Resources}

% UPDATE
Vacuum simulations of this system with a unit cell consisting of 4
pores, each with twenty layers of six monomers, total 65K atoms. On
Bridges, using MPI-enabled GROMACS 5.1.2, we obtain
89 ns/day using 140 cores, beyond which the scaling becomes mildly
nonlinear (See Figure~\ref{figure:gromacs} in the ``Code Performance
and Scaling'' document for more details). Simulations of the same
system solvated in water results in a total of 104K atoms, which ran
at 86 ns/day using 224 cores, at the limit of the near linear scaling
regime.  By creating a system with 50 layers, we reached 210K atoms,
which scaled nearly linearly up to 64 ns/day on 336 cores. We will use
these timings to estimate the time required for these experiments.

%FREE ENERGY CALCULATIONS

%HIGH RESOLUTION XRD (to pick up additional features)

%CROSSLINKING?

Once equilibrated and cross-linked, we will examine the effect of
solvation of the membranes on the pores. We have learned that at least 
1000 ns of total simulation is required to observe full penetration of
water through pores resulting in a cost of 
%UPDATE
(2000 ns / (86 ns/day))\times~24
hours\times~224 cores = 124K required SU, if we assume the initial
membrane simulation boxes are sufficiently large.
%UPDATE

We will carry out the same procedure with a set of new monomers. In addition to the simulations outlined above, we will need to equilibrate each system
using the developed procedures. Equilibration for 500 ns  

\begin{wraptable}{r}{0.4\textwidth} 
\begin{tabular}{|l|c|}
\hline
Simulation Set                & SU \\ 
\hline
Initial vacuum equilibration      &  57K \\
Test of monomer structuring       & 114K \\
System-size dependence            & 102K  \\
Membrane solvation                &  375K \\
Diffusion studies                 & 72K \\
Differences in bound counterion  & 430K \\
\hline
            Total         & 1140K \\
\hline
\end{tabular}
