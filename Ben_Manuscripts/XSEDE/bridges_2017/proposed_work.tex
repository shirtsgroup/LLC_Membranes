\subsection{How do water and ions move in nanostructured charged polymeric membranes?}
\subsubsection*{Background} 

The ability to design nanostructured membranes with concomitant high chemical
specificity is an important goal of the membrane separations
community.~\cite{Humplik2011} The ability to control
the nanoscale architecture of membranes will allow membranes to be 
designed precisely for the purpose of separating specific compounds.
~\cite{SmithRC1997,Zhu2006} The membranes being studied in this project
have pores sizes smaller than 1 nanometer, allowing size-selective 
filtration of small molecules and gases, as well as desalination of salt
water based on charge repulsion and size-exclusion of hydrated salt ions.

The membrane system we propose to model in this project is composed of
lyotropic liquid crystals (LLCs) which self-assemble into hexagonally
packed, cylindrical pores with acidic groups facing toward the cylinder
center, complexed to counterions. Subsequent cross-linking of vinyl
groups on monomer tail ends forms a mechanically strong structure
(Fig.~\ref{figure:membrane}). ~\cite{Zhou2003,Feng2014,Feng2016}
The pores, at the level of resolution obtained by SAXS and TEM, appear
to be straight and uniform in size (Fig.~\ref{figure:membrane}). 
This is very different from most commercially available membranes 
which have a pore size distribution with tortuous pathways that reduce
selectivity and flux respectively. The ordered structure exhibited by
LLC membranes makes them well suited for modeling using molecular 
dynamics simulations because the system has a defined structure that
can be studied in detail with a reasonably sized unit cell.

Our initial systems were built by first parameterizing a single
monomer using the General Amber Force Field (GAFF). Because the
self-assembly process is long relative to times which we can simulate,
monomers were rotated into layers of six monomers and stacked into
cylinders to give a starting configuration close to where we expect
the system to settle at equilibrium. While a single lyotropic liquid
crystal consists of a small number of atoms (138), the entire unit
cell consists of about 66K atoms. This number increases to 
approximately 100K when water molecules are added to the system.

Equilibration simulations have been performed which energy minimize and
allow the system to stabilize over the course of 500 ns in vacuum.  The
final structure and trajectory are analyzed quantitatively by
measuring pore size, distance between pores, distribution of
sodium ions in the pore, simulating X-ray diffraction patterns and 
calculating ionic conductivity. These measurements are compared to values
measured experimentally with SAXS and TEM imaging in order to validate
the final structure. SAXS and TEM images were taken using dry membranes
which justifies our initial study of the system in vacuum.

\begin{figure}[h]
\begin{center}
\begin{tabular}{c}
\includegraphics[width=8cm]{modeled.png}\\
\includegraphics[height=5cm]{membraneviews.png}\\
\end{tabular}
\end{center}
\caption{(above) SAXS and TEM experimental data, with hypothesized
  membrane structure. (Below, left) Top view of atomistic simulation,
  with hydrophilic head groups in orange, counterions in blue, and
  hydrophobic chains in blue. (Below, right) Side view of atomistic
  simulation showing counterions in dark blue and other atoms in light
  blue. These simulations demonstrate the heterogeneity in the
  modeled systems.~\label{figure:membrane}}
\end{figure}

\subsubsection*{Proposed Experiments and Justification of Resources}

Vacuum simulations of this system with a unit cell consisting of 4
pores, each with twenty layers of six monomers, total 66K atoms. On
Bridges, using MPI-enabled GROMACS 5.1.2, we obtain
99 ns/day using 224 cores, beyond which the scaling becomes mildly
nonlinear (See Figure~\ref{figure:LLCscaling} in the ``Code Performance
and Scaling'' document for more details). Simulations of the same
system solvated in water results in a total of 100K atoms, which ran
at 84.2 ns/day using 224 cores, at the limit of the near linear scaling
regime.  By creating a system with 80 layers, we reached 265K atoms,
which scaled nearly linearly up to 29 ns/day on 224 cores. We will use
these timings to estimate the time required for these experiments.

We repeated scaling studies with the same systems using GPU-enabled 
GROMACS 2016 with both the NVIDIA K80 and NVIDIA P100 GPUs available on
Bridges. Using 4 K80 GPUs we obtain 58 ns/day, 56.4 ns/day and 10.1 ns/day
for the 66K, 100K and 256K atom systems respectively. Using 4 P100 GPUs, 
we obtain 83.6 ns/day, 62.0 ns/day and 18.8 ns/day for each system 
respectively. Although in some cases, the system scales nearly linearly
for greater than 4 GPUs, additional GPUs signficantly extends queue wait
times due to high demand. The cost will be the same, but starting simulations
earlier will allow analysis to be performed during simulations and prevent
valueless simulation from preceding, saving computer time. 

In order to study the relative stability of the two metastable states
we have discovered, we will need to perform a computationally 
intensive free energy calculation using the Multistate Bennett
Acceptance Ratio (MBAR) technique. MBAR estimates free energy 
differences with the currently lowest variance when compared to other
estimators. In order to make a reasonable estimate, we will need to 
conduct simulations of all intermediate states which lead from one metastable
state to another. Each configuration in the pathway mapping the two
states must be sufficiently similar to adjacent configurations so that
we achieve enough phase space overlap for a more precise calculation. 
We estimate that we will need at least 50 intermediate states, each run
for 50 ns for a total of 2500 ns of simulation time. This will require 
(2500 ns / (99.2 ns/day))* 24 hrs/day * 224 cores = 135K SU using
MPI on the RM partition, (2500 ns / (58 ns/day)) * 24 hours * 4 GPUs * 1 SU
/ GPU-hour = 4.1K SU on the K80 GPU nodes, and (2500 ns / (83.6 ns/day)) 
* 24 hours * 4 GPUs * 2.5 SU / GPU-hour = 7.2K SU on the P100 GPU nodes.

The resolution of simulated X-ray diffraction patterns is dependent on the
size of the simulated unit cell. To create higher resolution patterns in 
the x, y or z directions, requires an increase in the respective dimension
of the unit cell by adding more atoms. The fundamental reason for this
limitation is the necessity of meeting the Bragg condition. When met,
constructive X-ray interference occurs resulting in a signal which gives
details about the position of atoms relative to each other. The lattice
planes in the crystal, defined by the reciprocal space Miller indices 
h, k, and l, are separated by a distance, d. One can calculate all possible
d values given all unit cell parameters. It is not trivial to see that an
increase in box vector leads to a wider range of accessible hkl values 
and increases the spatial resolution. A simple way to estimate the simulated
resolution in each direction is using the equations qx = 2$\pi$/Lx, qy =
2$\pi$/Ly, qz = 2$\pi$/Lz. Our current resolution with a 66K atom system is ~ 0.078 $A^{-1}$ (Fig.~\ref{fig:expxrdcomp}). 
We propose a 4x increase in our z dimension resolution which will
help us to distinguish reflections deemed a consequence of benzene ring 
pi-stacking (reflection occurs at 1.53 inverse angstroms) from simple
alkane chain packing (reflection occurs between 1.4 and 1.57 inverse
angstroms). We also hope to pick up finer details such as the sharp line
that appears at ~.85 inverse angstroms experimentally but is absent in 
the simulated patterns. A system of this size is made of 265K atoms. 
Stacking equilibrated membrane layers directly on top of each will 
facilitate a fast equilibration of the large system. Equilibration 
simulations will be run for at least 50 ns followed by another 50 ns of 
simulation needed to collect enough information to simulate the XRD 
pattern. 100 ns of simulation time for a 265K atom system will require 
19K SU on the RM nodes, 950 SU on K80 GPU nodes, and 1.3K SU on P100 GPU nodes.     
\begin{figure}
\centering
\includegraphics[width=\linewidth]{Sim_exp_xrd_sidebyside.png}
	\caption{(left) Experimental Wide Angle X-ray Diffraction experiments contain
reflections at qz \approx 1.7 A $^{-1}$ indicating pi-stacking of benzene
rings in the monomer head group. A weak, correlated reflection is present at 
qz \approx 0.85 A $^{-1}$. (right) Simulated X-ray diffraction shows a pi-stacking reflection
at qz \approx 1.5 A $^{-1}$. The weak correlated reflection is not
present but may be visible with higher resolution simulations}
	\label{fig:expxrdcomp}

\end{figure}

Once equilibrated and cross-linked (an algorithm which can run without the 
aid of an HPC resource), we will examine the effect of membrane
solvation on the pore structure and corresponding transport properties.
We have learned that at least 1000 ns of total simulation is required
to fully equilibrate the system with water. These simulations will require
(1000 ns / (84.2 ns/day)) * 24 hours * 224 cores = 64K SU on RM nodes, 
(1000 ns / (56.4 ns/day)) * 24 hours * 4 GPUs * 1 SU / GPU-hour = 1.7K SU on K80 GPU nodes and
(1000 ns / (62 ns/day)) * 24 hours * 4 GPUs * 2.5 SU / GPU-hour = 3.9 SU on P100 GPU nodes. 

We will carry out the same procedure with a new set of monomers similar
in structure to the current monomer. Simple modifications can be made
to the monomer structure as outlined in Figure ~\ref{fig:chemistries}. 
We will need to equilibrate each system using developed procedures which
requires equilibration simulations of at least 500 ns. Of the 
possibilities outlined in Figure ~\ref{fig:chemistries} we will try all
variations of tail spacer and ionic headgroup and at least 3 counterions
of different charge (+1, +2, +3). This results in a total of 23 new systems 
that require 500 ns of equilbration time. These length simulations should
also give us the required information to calculate physical properties 
such as ionic conductivity, and to simulate X-ray diffraction patterns.
To run these simulations, a total of 12000 ns, we will require 623K, 19K, 
or 34K SU on the RM, K80 or P100 nodes respectively. To solvate all the test
systems, an additional 1250K, 38K, or 68K SU will be required for the RM
K80 or P100 nodes respectively.
% BJC: obviously this is a lot. We should talk more about this to narrow 
% the space. Although, we can always just ask for it and see what happens.

We will conduct transport studies on all systems in order to relate pore
structure to macroscopic observables such as water flux and solute
rejection. We will measure diffusion of various solutes across the 
membrane. Understanding the mechanims of transport of each solute will
provide us with sufficient understanding to suggest new modifications 
to monomer structure that should increase laboratory performance. We
anticipate simulations of at least 50 ns will be necessary to get 
sufficient sampling for an informative mean square displacement (MSD) 
curve. Based on the shape of the MSD curve we can identify potential 
transport mechanisms. Based on quantitative studies of the structure
of the 24 equilibrated systems already studied, we will narrow our 
design space to 5 with the most potential. We will study transport in % 5 is pretty arbitrary here
each of the five with at least 5 solutes of varying charge and 
hydrodynamic radii. This will require a total of 25 50 ns simulations for
a total of 1250 ns of solvated system simulation. This will require
80K, 2.1K or 4.8K SU using RM, K80 or P100 nodes respectively.   

\begin{figure}
\centering
\includegraphics[width=\linewidth]{chemistries.png}  % who made this picture? It has a typo : "ionic headgoup"
	\caption{LLC chemistries we will examine in order to understand determinants of pore size, structure and solute transport}
	\label{fig:chemistries}
\end{figure}

%\begin{wraptable}{r}{0.4\textwidth} 
%\begin{tabular}{|l|c|}
%\hline
%Simulation Set               & RM SU & K80 SU & P100 SU \\ 
%\hline
%Free Energy Calculation      & 135K  & 4.1K   & 7.2K    \\
%High Resolution XRD          & 19K   & 950    & 1.3K    \\
%Vacuum Equilibration         & 650K  & 19K    & 34K     \\
%Membrane Solvation           & 1250K & 38K    & 68K     \\
%Transport studies            & 80K   & 2.1K   & 4.8K    \\
%\hline
%            Total            & 2134K & 64.15K & 115.3K \\
%\hline
%\end{tabular}
