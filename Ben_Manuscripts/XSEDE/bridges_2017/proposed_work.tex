subsection{How do water and ions move in nanostructured charged polymeric membranes?}
\subsubsection*{Background} 

The ability to design nanostructured membranes with concomitant high chemical
specificity is an important goal of the membrane separations
community.~\cite{Humplik2011} The ability to control
the nanoscale architecture of membranes will allow membranes to be 
designed precisely for the purpose of separating specific compounds.
~\cite{SmithRC1997,Zhu2006} The membranes being studied in this project
have pores sizes smaller than 1 nanometer, allowing size-selective 
filtration of small molecules and gases, as well as desalination of salt
water based on charge repulsion and size-exclusion of hydrated salt ions.

The membrane system we propose to model in this project is composed of
lyotropic liquid crystals (LLCs) which self-assemble into hexagonally
packed, cylindrical pores with acidic groups facing toward the cylinder
center, complexed to counterions. Subsequent cross-linking to forms
a mechanically strong structure (Fig.~\ref{figure:membrane}).
~\cite{Zhou2003,Feng2014,Feng2016} The pores, at the level of resolution
obtained by SAXS and TEM, appear to be straight and uniform in size 
(Fig.~\ref{figure:membrane}).  This is very different from most 
commercially available membranes which have a pore size distribution
with tortuous pathways that reduce selectivity and flux respectively.
The ordered structure exhibited by LLC membranes makes them well suited 
for modeling using molecular dynamics simulations because the
system has a defined structure that can be studied in a reasonably sized 
unit cell.

Our initial systems were built by first parameterizing a single
monomer using the General Amber Force Field (GAFF). Because the
self-assembly process is long relative to times which we can simulate,
monomers were rotated into layers of six monomers and stacked into
cylinders to give a starting configuration close to where we expect
the system to settle at equilibrium. While a single lyotropic liquid
crystal consists of a small number of atoms (138), the entire unit
cell consists of about 66K atoms. This number increases to 
approximately 100K when water molecules are added to the system.

% BJC: The following is technically true although I've been creating
% specific configurations for testing purposes. 
In order to assemble the structures shown in Fig.~\ref{figure:membrane}, a
suitable monomer configuration needed to be chosen. Thermal annealing
of the monomer using GROMACS yielded many low energy configurations
close to what is expected. 

Initial simulations have been performed which energy minimize and
allow the system to stabilize over the course of 20 ns in vacuum.  The
final structure and trajectory are analyzed quantitatively by
measuring pore size, distance between pores, and distribution of
sodium ions in the pore. These measurements are compared to values
measured experimentally with SAXS and TEM imaging in order to validate
the final structure. SAXS and TEM images were taken using dry membrane
which justifies our initial study of the system in vacuum.

%\begin{figure}[h]
%\begin{center}
%\begin{tabular}{c}
%\includegraphics[width=8cm]{modeled.png}\\
%\includegraphics[height=5cm]{membrane2.pdf}\\
%\end{tabular}
%\end{center}
%\caption{(above) SAXS and TEM experimental data, with hypothesized
%  membrane structure. (Below, left) Top view of atomistic simulation,
%  with hydrophilic head groups in orange, counterions in blue, and
%  hydrophobic chains in blue. (Below, right) Side view of atomistic
%  simulation showing counterions in dark blue and other atoms in light
%  blue. These simulations demonstrate the heterogeneity in the
%  modeled systems.~\label{figure:membrane}}
%\end{figure}

\subsubsection*{Proposed Experiments and Justification of Resources}

Vacuum simulations of this system with a unit cell consisting of 4
pores, each with twenty layers of six monomers, total 66K atoms. On
Bridges, using MPI-enabled GROMACS 5.1.2, we obtain
99 ns/day using 224 cores, beyond which the scaling becomes mildly
nonlinear (See Figure~\ref{figure:gromacs} in the ``Code Performance
and Scaling'' document for more details). Simulations of the same
system solvated in water results in a total of 100K atoms, which ran
at 121 ns/day using 336 cores, at the limit of the near linear scaling
regime.  By creating a system with 80 layers, we reached 265K atoms,
which scaled nearly linearly up to 29 ns/day on 224 cores. We will use
these timings to estimate the time required for these experiments.

We repeated scaling studies with the same system using the GPU-enabled 
GROMACS 2016 which was compiled for optimal use with NVIDIA Tesla K80
accelerators. On Bridges we obtain 58 ns/day for the 66K atom system
using 4 GPUs and 28 cores. Simulating the solvated configuration, we
obtain 56.4 ns/day on 4 GPUs and 28 cores. Simulating the 265K atom 
system we obtain 10.1 ns/day on 4 GPUs and 28 cores.

In order to study the relative stability of the two metastable phases 
we have discovered, we need will need to perform a computationally 
intensive free energy calculation using the Multistate Bennett
Acceptance Ratio (MBAR) technique. MBAR estimates free energy 
difference with current lowest variance when compared to other estimators.
In order to make a reasonable estimate, we will need to conduct 
simulations all intermediate states which lead from one metastable
state to another. Each configuration in the pathway mapping the two
states must be sufficiently similar to adjacent configurations so that
we achieve enough phase space overlap for a more precise calculation. 
We estimate that we will need at least 50 intermediate states, each run
for 50 ns for a total of 2500 ns of simulation time. This will require 
(2500 ns / (99.2 ns/day))* 24 hrs/day * 224 cores = 135K SU using
MPI on the RM partition. This will require (2500 ns / (58 ns/day)) * 24
hours * 4 GPUs = 4.1K SU on the K80 GPU nodes.

The resolution of simulated X-ray diffraction patterns is dependent on the
size of the simulated unit cell. To create higher resolution patterns in 
the x, y or z directions, requires an increase in the respective dimension
of the unit cell by adding more atoms. The fundamental reason for this
limitation is the necessity of meeting the Bragg condition. When met,
constructive X-ray interference occurs, producing a signal on a nearby 
detector. The lattice planes in the crystal, defined by the reciprocal
space Miller indices h, k, and l, are separated by a distance, d. One can
calculate all possible d values given all unit cell parameters. It is
not trivial to see that an increase in box vector leads to a wider range
of accessible hkl values and increases the resolution. A simple way to 
estimate the simulated resolution in each direction is using the equations
qx = 2*pi/x, qy = 2*pi/y, qz = 2*pi/z. Our current resolution with a 66K
atom system is ~ 0.078 inverse angstroms. We propose a 4x increase in our
z dimension resolution which will help us to distinguish reflections 
deemed a consequence of benzene ring pi-stacking (reflection occurs at 1.53
inverse angstroms) from simple alkane chain packing (reflection occurs
between 1.4 and 1.57 inverse angstroms). We also hope to pick up finer 
details such as the sharp line that appears at ~.85 inverse angstroms 
experimentally but is absent in the simulated patterns. A system of this 
type is made of 265K atoms. Stacking equilibrated membrane layers 
directly on top of each will allow facilitate a fast equilibration of the
large system. Equilibration will be allowed to run for at least 50 ns 
followed by 50 ns of simulation needed to collect enough information to
simulate the XRD pattern. 100 ns of simulation time for a 265K atom system 
will require 19K SU on the RM nodes and 950 SU on K80 GPU nodes.     
%HIGH RESOLUTION XRD (to pick up additional features)

%CROSSLINKING?

Once equilibrated and cross-linked, we will examine the effect of
solvation of the membranes on the pores. We have learned that at least 
1000 ns of total simulation is required to observe full penetration of
water through pores resulting in a cost of 
%UPDATE
(2000 ns / (86 ns/day)) * 24
hours * 224 cores = 124K required SU, if we assume the initial
membrane simulation boxes are sufficiently large.
%UPDATE

We will carry out the same procedure with a set of new monomers. In addition to the simulations outlined above, we will need to equilibrate each system
using the developed procedures. Equilibration for 500 ns  

%\begin{wraptable}{r}{0.4\textwidth} 
%\begin{tabular}{|l|c|}
%\hline
%Simulation Set                & SU \\ 
%\hline
%Initial vacuum equilibration      &  57K \\
%Test of monomer structuring       & 114K \\
%System-size dependence            & 102K  \\
%Membrane solvation                &  375K \\
%Diffusion studies                 & 72K \\
%Differences in bound counterion  & 430K \\
%\hline
%            Total         & 1140K \\
%\hline
%\end{tabular}
