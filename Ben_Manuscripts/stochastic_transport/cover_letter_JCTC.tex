\documentclass[fontsize=11pt]{article}

\usepackage[margin=1in]{geometry}
\usepackage{graphicx}
\usepackage{wrapfig}
\usepackage{subcaption}
\usepackage{amsmath} 
\usepackage{siunitx}
\usepackage{booktabs}
\usepackage{gensymb}
\usepackage{hyperref}
\usepackage[export]{adjustbox}
\usepackage{anyfontsize}

\begin{document}

	\graphicspath{{./figures/}}

	%BJC: check address, phone number and fax (?) number
	\begin{figure}
	\centering
	\begin{minipage}{0.37\textwidth}
	\includegraphics[width=2.14in,left]{CUBoulder.pdf}
	\end{minipage}
	\begin{minipage}{0.62\textwidth}
	\scriptsize
	\noindent Department of Chemical and Biological Engineering \hfill t 303-735-7860~~~~~~~~~~~~~~~~~~ \\
	\noindent 596 UCB \hfill f 303-492-8425~~~~~~~~~~~~~~~~~~ \\
	\noindent Boulder, Colorado 80309 \hfill michael.shirts@colorado.edu \\
	\end{minipage}
	\end{figure}
	
	%\noindent May 3\textsuperscript{rd}, 2019\\
	\noindent \today \\

	\noindent Dear Editors,\\
	
	\newcommand{\ManuscriptTitle}{Capturing Subdiffusive Solute Dynamics and 
	Predicting Selectivity in Nanoscale Pores with Time Series Modeling}
	
	We are submitting ``\ManuscriptTitle'' by Benjamin J. Coscia
        and Michael R.  Shirts, for consideration for publication as
        an Article in the Journal of Chemical Theory and
        Computation. We expect this work to be of significant interest
        to researchers who study complex transport at the molecular
        level by adding important new techniques for estimating long
        time scale behavior and bulk transport behavior from shorter
        molecular trajectories exhibiting anomalous diffusion rather
        than Brownian motion. Please note that this article has been
        uploaded to arXiv with differences only in formatting:
        \href{https://arxiv.org/abs/2004.07905}{https://arxiv.org/abs/2004.07905}. \\
	
	In this article, we analyze solute trajectories generated from
        5$\mu$s molecular dynamics (MD) simulations of an inverted
        hexagonal (H\textsubscript{II}) phase self-assembled lyotropic
        liquid crystal (LLC) polymer membrane to parameterize two
        different stochastic time series approaches capable of
        predicting solute flux and selectivity on a macroscopic
        scale. Both approaches quantitatively reproduce the mean
        squared displacements (MSDs) exhibited by solutes in MD
        simulations. This is among the first studies to apply time
        series modeling of molecular simulations in order to infer
        propagate behavior to long time scales and estimate bulk
        materials properties with complex anomalous diffusion
        mechanisms.\\

	This article builds on recent work of ours published in
        J. Phys. Chem. B where we characterized transport mechanisms
        exhibited by a series of small polar solutes in an LLC
        Membrane (Coscia and Shirts \textit{J. Phys.  Chem. B}, 123,
        6314--6330 (2019)). In this work, we build on the chemical
        intuition gained from our previous study in order to formulate
        stochastic models which produce trajectory realizations that
        share the same dynamical properties as those exhibited by
        solutes in our MD simulations. We explore two main modeling
        approaches. First, using anomalous diffusion theory, we
        parameterize solute motion as subordinated fractional Brownian
        and L\'evy motion processes, based on the observation that
        solutes experience a series of dwell times separated by
        anti-correlated hop lengths. Secondly, we add state-dependent
        dynamics to standard Markov state modeling approaches, where
        we model solute motion as a series of transitions between
        various trapped states, each with different dynamical
        properties.  Finally, we demonstrate how one can project our
        best model onto macroscopic timescales in order to estimate
        solute flux and selectivity.\\
	
	%BJC: included phone numbers because it asked to do so in author's guide.
	\noindent Some suggestions for reviewers are:
	\begin{enumerate}
	
		\item Francisco Hung has used molecular simulations to study nematic liquid crystals
		and transport under confinement within various nanostructured materials 
		(Northeastern University, 617-373-8619,\\ \href{mailto:f.hung@northeastern.edu}{f.hung@northeastern.edu})
		
		\item Mahesh Mahanthappa has experience modeling soft materials using molecular dynamics
		including work on ion transport in various ordered liquid crystalline phases. 
		(University of Minnesota, 612-625-4599, \href{mailto:maheshkm@umn.edu}{maheshkm@umn.edu})
		
		\item Stephen Paddison uses various levels of molecular simulation theory to understand
		proton transport in nanostructured and charged polymer membranes. (The University of Tennessee 
		Knoxville, 865-974-2026, \href{mailto:spaddison@utk.edu}{spaddison@utk.edu})
		
	\end{enumerate}
	
	\noindent Please send correspondence regarding this paper to Michael R. Shirts (contact
	details in letterhead).\\	
	
	\noindent Sincerely,
	
	\noindent Michael R. Shirts \\
	\noindent Benjamin J. Coscia \\
	
\end{document}

% LocalWords:  BJC Subdiffusive Solute Nanoscale Coscia arXiv solute lyotropic
% LocalWords:  LLC MSDs solutes Phys Chem subdiffusive evy timescales Klafter
% LocalWords:  univeristy JPCB woould nanostructured Mahesh Mahanthappa klafter
% LocalWords:  Tel Paddison
