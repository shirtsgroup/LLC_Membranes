\documentclass{article}
\usepackage{graphicx}
\usepackage{wrapfig}
\usepackage{subcaption}
\usepackage[margin=1in]{geometry}
\usepackage{amsmath} % or simply amstext
\usepackage{siunitx}
\usepackage{booktabs}
\usepackage[export]{adjustbox}
\newcommand{\angstrom}{\textup{\AA}}
\newcommand{\colormap}{jet}  % colorbar to use
\usepackage{cleveref}
\usepackage{booktabs}
\usepackage{gensymb}
\usepackage{float}
\usepackage{xr}

\externaldocument[M-]{Draft}

\renewcommand{\thefigure}{S\arabic{figure}}
\renewcommand{\thesection}{S\arabic{section}}
\renewcommand{\thepage}{S\arabic{page}}
\renewcommand{\thetable}{S\arabic{table}}

\title{Supporting Information: Time Series Modeling of Solute Transport in an H\textsubscript{II} 
Phase Lyotropic Liquid Crystal Membrane}
\author{Benjamin J. Coscia \and Michael R. Shirts} 

\begin{document}

  \maketitle
  \graphicspath{{./supporting_figures/}}
  \bibliographystyle{ieeetr}
  
  \section{Setup and analysis scripts}\label{section:python_scripts}

  All python and bash scripts used to set up systems and conduct post-simulation trajectory
  analysis are available online at \texttt{https://github.com/shirtsgroup/LLC\_Membranes}.
  Documentation for the \texttt{LLC\_Membranes} repository is available at
  \texttt{https://llc-membranes.readthedocs.io/en/latest/}. Table~\ref{table:python_scripts}
  provides more detail about specific scripts used for each type of analysis performed in
  the main text.

  \begin{table}[htb!]
  \centering
  \newcolumntype{A}{ >{\centering\arraybackslash} m{2.5in} }
  \newcolumntype{B}{ >{\centering\arraybackslash} m{0.75in} }
  \newcolumntype{C}{m{2.75in}}
  \begin{tabular}{|A|B|C|}
  \hline
  \textbf{Script Name} & \textbf{Section} & ~~~~~~~~~~~~~~~~~~~~~\textbf{Description} \\
  \hline

  % mimic this
  \texttt{/setup/param.sh} & 2.1 & Parameterize liquid
  crystal monomers and solutes with GAFF \\ \hline

  \end{tabular}

  \caption{The first column provides the names of the python scripts available in
  the \texttt{LLC\_Membranes} GitHub repository that were used for system setup and
  post-simulation trajectory analysis. Paths preceding script names are relative to the
  \texttt{LLC\_Membranes/LLC\_Membranes} directory. The second columns lists the section in the main
  text where the output or usage of the script is first described. The third column
  gives a brief description of the purpose of each script.
  }~\label{table:python_scripts}

  \end{table}

  \section{Choosing a transport model}\label{section:transport_model_selection}

  We used the toolbox created by Meroz and Sokolov in order to justify our
  choice of transport model.\cite{meroz_toolbox_2015} The solutes in our systems
  exhibit anomalous transport properties characteristic of a Continuous Time
  Random Walk (CTRW). 

  \subsection*{Mean Squared Displacement}

  The general form of a mean squared displacement (MSD) curve is:
  \begin{equation}
	\langle x^2(t) \rangle \sim t ^ \alpha
	\label{eqn:msd}
  \end{equation}
  For brownian motion, $\alpha = 1$ and the MSD is linear. When $\alpha \neq
  1$, the particle of interest exhibits anomalous diffusion. Values of $\alpha$
  greater than 1 give rise to superdiffusion, while values of $\alpha$ less than
  1 give rise to subdiffusion.

  We can calculate the ensemble-averaged MSD curve by averaging the MSDs of
  each particle trajectory, where each MSD is calculated using:
  \begin{equation}
	\delta^2(t) = \| \mathbf{r}(t) - \mathbf{r}(0) \|^2
	\label{eqn:ensemble_msd}
  \end{equation}
  where $\|\cdot\|$ represents the Euclidean norm. 

  The mean squared displacement of solutes in our model is a non-linear
  function of time, with $\alpha < 1$ which is indicative of anomalous
  subdiffusion. Figure \ref{fig:msd_power_law}a plots the ensemble-averaged MSD
  curve for 24 ethanol molecules diffusing in a 10 wt\% water H\textsubscript{II}
  LLC membrane system. We fit a power law of the form $Ae^{\alpha}$ to the MSD
  curve. We performed 2000 bootstrap trials by randomly sampling 24 MSD curves
  with replacement from the 24 total ethanol MSD curves. The bootstrapped average
  value of $\alpha$ is 0.75 for this system. 
 
  \begin{figure}[!htb]
  \centering
% Generated with : msd.py -t PR_nojump.xtc -g PR.gro -r ETH -ensemble -power_law -a z -nboot 2000
% in directory: /home/bcoscia/Documents/Gromacs/Transport/NaGA3C11/ETH/10wt
  \includegraphics[width=0.8\linewidth]{msd_power_law.pdf}
  \caption{(a) We fit a curve with the form of Equation~\ref{eqn:msd} to the
	  ensemble-averaged MSD curve. (b) The average value of $\alpha$, obtained using
	  fits to MSDs calculated from bootstrapped ensembles, is less than 1 suggesting
	  that ethanol molecules in our model exhibit subdiffusive
	  behavior.}\label{fig:msd_power_law}
  \end{figure}

  \subsection*{Ergodicity}

  The ergodicity of a system can help us narrow down the possible anomalous
  diffusion mechanisms. In an ergodic system, the time-averaged behavior of an
  observable should yield the same result as the ensemble average of the same
  observable. Examples of anomalous diffusion processes that are ergodic include
  random walks on fractals (RWF) and fractional brownian motion (FBM).
  Non-ergodic systems generally give rise to CTRWs with the possibility of
  combination with a RWF and/or FBM.\cite{meroz_toolbox_2015} 

  We tested the ergodicity of our system by comparing the ensemble-averaged
  and time-averaged MSD curves. We calculated the MSD of each ethanol trajectory
  using Equation~\ref{eqn:ensemble_msd} and a time-averaged algorithm: 
  \begin{equation}
	\delta^2(t) = \dfrac{1}{N-t} \sum_{i=0}^{N-t-1} \| \mathbf{r}(i + t) - \mathbf{r}(i) \|^2
  \end{equation}
  where N is the total number of simulation frames, and t represents the length
  of subinterval or number of frames per subinterval. We averaged the MSD curves
  from each trajectory in order to create final MSD plots.

  The ethanol molecules exhibit non-ergodic behavior because their
  time-averaged and ensemble-averaged MSDs do not agree with each other
  (Figure~\ref{fig:ethanol_msd_comparison}). We validated our analysis using a 1
  ns simulation of a box of tip3p water molecules. As expected, since the
  particles exhibit Brownian motion, the time-averaged and ensemble-averaged MSDs
  agree with each within error (Figure~\ref{fig:water_box_msd_comparison}).

  \begin{figure}[!htb]
  \centering
  \begin{subfigure}{0.45\textwidth}
% Generated with : msd.py -t PR_nojump.xtc -g PR.gro -r ETH -compare -nboot 2000 -a z
% in directory: /home/bcoscia/Documents/Gromacs/Transport/NaGA3C11/ETH/10wt
  \includegraphics[width=\textwidth]{ethanol_msd_comparison.pdf}
  \caption{}\label{fig:ethanol_msd_comparison}
  \end{subfigure} 
  \begin{subfigure}{0.45\textwidth}
% Generated with msd.py -t traj_nojump.xtc -g npt.gro -r SOL -compare --fracshow 0.4 -nboot 2000 -a z
% in directory: /home/bcoscia/Documents/Gromacs/Transport/Solvent/solvent_boxes/pure_water
  \includegraphics[width=\textwidth]{water_box_msd_comparison.pdf}
  \caption{}\label{fig:water_box_msd_comparison}
  \end{subfigure} 
  \caption{(a) The time-averaged and the ensemble-averaged MSDs for ethanol in
	  an H\textsubscript{II} nanopore are not in agreement, implying non-ergodicity.
	  (b) A box of tip3p water molecules is expected to be ergodic and it is shown to
	  be true here because both MSDs are in agreement. }\label{fig:msd_comparison}
  \end{figure}

  \subsection*{Autocorrelation of steps}

% From Sokolov paper: "Assigning different waiting times τ i to each step, and assuming that the
% steps are uncorrelated as in a regular RW, gives rise to the CTRW model" -- I
% might need to check autocorrelation of steps lengths when a hop occurs rather
% than every time step. Not sure if there is enough data for that, but could check 
% a particularly hoppy trajectory after longer simulation.


  Based on the previous two sections, our model can likey be studied as a CTRW. 
  However, it is still possible that our CTRW model might also be convoluted with
  an FBM or a RWF process. In a pure CTRW, the steps are uncorrelated. 
  Both FBM and RWF exhibit anti-correlated steps. 

  The steps in our system are not correlated. We showed this by calculating the
  autocorrelation function (ACF) of the step lengths in the $z$-direction. The
  ACF of a representative trajectory is shown in Figure~\ref{fig:eth_autocorrelation}.  
  
  \begin{figure}[!htb]
  \centering
% Generated with brownian_test.py -t PR_nojump.xtc -g PR.gro -r ETH (and appropriate uncommenting -- need to rework that script)
% in directory: /home/bcoscia/Documents/Gromacs/Transport/NaGA3C11/ETH/10wt
  \includegraphics[width=0.8\textwidth]{eth_autocorrelation.pdf}
  \caption{The autocorrelation function (right) of a representative ethanol
	   center of mass $z$-coordinate trajectory (left) almost immediately decays to zero,
	   indicating a complete loss of memory of it's previous position. Noise increases
	   at large time lags due to decreased sampling.}\label{fig:eth_autocorrelation}
  \end{figure}
  
  \newpage
  \section{Estimating the Hurst Parameter}\label{section:H_estimate}
  
  We chose to estimate the Hurst parameter, $H$ by a least squares fit to the analytical
  autocorrelation function for fractional Brownian motion (the variance-normalized version 
  of Equation~\ref{M-eqn:fbm_autocorrelation} in the main text):
  
  \begin{equation}
    \gamma(k) = \dfrac{1}{2}\bigg[|k-1|^{2H} - 2|k|^{2H} + |k+1|^{2H}\bigg]
  \label{eqn:fbm_autocorrelation}
  \end{equation}  
  
  In Figure~\ref{fig:hurst_autocorrelation}, we plotted Equation~\ref{eqn:fbm_autocorrelation}
  for different values of $H$. When $H > 0.5$, Equation~\ref{eqn:fbm_autocorrelation} decays
  slowly to zero meaning one needs to study large time lags with high frequency in order to
  obtain accurate estimates. Fortunately, all of our solutes show anti-correlated motion, so
  most of the information in Equation \ref{eqn:fbm_autocorrelation} is contained within the 
  first few lags. 

  % /supporting_figures/hurst_autocorrelation.py
  \begin{figure}
  \centering
  \includegraphics[width=0.5\textwidth]{hurst_autocorrelation.pdf}
  \caption{The analytical autocorrelation function of FBM decays to zero faster when 
  H $<$ 0.5 compared to when H $>$ 0.5.}\label{fig:hurst_autocorrelation}
  \end{figure}
  
  The autocovariance function of Fractional L\'evy motion is different from fractional
  Brownian motion (see Equation~\ref{M-eqn:fbm_autocorrelation}
  and~\ref{M-eqn:flm_autocovariance} of the main text), but their autocorrelation 
  structures are the same. The autocovariance function of FLM is dependent on the 
  expected value of squared draws from the underlying L\'evy distribution, $E\big[L(1)^2\big]$. 
  This is effectively the distribution's variance, which is undefined for most 
  L\'evy stable distributions due to their heavy tails. As a consequence, one should 
  expect $E\big[L(1)^2\big]$ to grow as more samples are drawn from the distribution. 
  However, we are only interested in the autocorrelation function. In order to predict
  the Hurst parameter from the autocorrelation function, we must show that it has 
  well-defined structure and is independent of the coefficient in 
  Equation~\ref{M-eqn:flm_autocovariance} of the main text. In Figure~\ref{fig:flm_autocorrelation}, 
  we plot the average autocorrelation function from an FLM process with an increasing 
  number of observations per generated sequence. For all simulations we set $H$=0.35 
  and $\alpha$=1.4. The variance-normalized autocovariance function, i.e. the autocorrelation
  function does not change with increasing sequence length. Additionally, the
  autocorrelation function of FBM, with the same $H$ is the same.
  
  % supporting_figures/flm_autocov.py
  \begin{figure}
  \centering
  \includegraphics[width=0.5\textwidth]{flm_autocovariance.pdf}
  \caption{The autocorrelation function of an FLM process does not change with 
  increasing sequence length (N). It shares the same autocorrelation function as
  fractional Brownian motion (FBM). All sequenced used to make this plot were 
  generated using $H$=0.35 and $\alpha$=1.4 (for FLM).}\label{fig:flm_autocovariance}
  \end{figure}
  
  \section{Simulating Fractional L\'evy Motion}\label{section:sFLM}
  
  \subsection*{Achieving the right correlation structure}
  
  We simulated FLM using the algorithm of Stoev and Taqqu~\cite{stoev_simulation_2004}.
  There are no known exact methods for simulating FLM. As a consequence, passing a
  value of $H$ and $\alpha$ to the algorithm does not necessarily result in the correct
  correlation structure, although the marginal L\'evy stable distribution is correct. 
  We applied a database-based empirical correction in order to use the
  algorithm to achieve the correct marginal distribution and correlation structure.
  
  Stoev and Taqqu note that the transition between negatively and positively correlated
  draws occurs when $H = 1/ \alpha$. When $\alpha=2$, the marginal distribution is 
  Gaussian and $H=0.5$ as expected from FBM. We corrected the input $H$ so that the 
  value of $H$ measured based on the output sequence equaled the desired $H$. We 
  first adjusted the value of $H$ by adding ($1 / \alpha - 0.5$), effectively 
  recentering the correlation sign transition for any value of $1 \leq \alpha \leq 2$.
  This correction alone does a good job for input $H$ values near 0.5, but is
  insufficient if one desires a low values of $H$. The exact correction to $H$ is 
  not obvious so we created a database of output $H$ values tabulated as a function
  of input $H$ and $\alpha$ values. Figure~\ref{fig:hurst_correction} demonstrates the
  results of applying our correction. Without the correction, FLM realizations are
  more negatively correlated. This would result in under-predicted mean squared
  displacements when applying the model.
  
  % /supporting_figures/demonstrate_hurst_correction.py
  \begin{figure}
  \centering
  \includegraphics[width=0.5\textwidth]{hurst_correction.pdf}
  \caption{Correcting the Hurst parameter input to the algorithm of Stoev and Taqqu
  results in an FLM process with a more accurate correlation structure. We generated
  sequences with an input $H$ of 0.35. We estimated $H$ by fitting the autocorrelation
  function. Without the correction, $H$ is underestimated, meaning realizations are 
  more negatively correlated than they should be.}\label{fig:hurst_correction}
  \end{figure}

  \subsection*{FLM realizations from truncated L\'evy stable distributions}
  
  To generate realizations from an uncorrelated truncated L\'evy process, one would
  randomly sample from the base distribution and replace values that are too large
  with new random samples from the base distribution, repeating the process until
  all samples are under the desired cut-off. 
  
  This procedure is complicated by the correlation structure of FLM. At a high level,
  Stoev and Taqqu use Riemann-sum approximations of the stochastic integrals defining
  FLM in order to generate realizations. They do this efficiently with the help of 
  Fast Fourier Transforms. In practice, this requires one to Fourier transform a zero-padded
  vector of random samples drawn from the appropriate L\'evy stable distribution, multiply
  the vector in Fourier space by a kernel function and invert back to real space. The end
  result is a correlated vector of fractional L\'evy noise.
  
  If one is to truncate an FLM process, one can apply the simple procedure above for 
  drawing uncorrelated values from the marginal L\'evy stable distribution, but after
  adding correlation, the maximum drawn value is typically lower than the limit set 
  by the user. Additionally, the shape of the distribution itself changes. Analogous
  to the database used to correct the Hurst parameter, we created a database to 
  correct the input truncation parameter (the maximum desired draw). The database
  returns the value of the truncation parameter that will properly truncate the
  output marginal distribution based on $H$, $\alpha$ and $\sigma$ (the width parameter).
  Figure~\ref{fig:truncation_correction} shows the result of applying our correction.
  
  \begin{figure}
  \centering
  \includegraphics[width=0.5\textwidth]{truncation_correction.pdf}
  \caption{We can accurately truncate the marginal distribution of FLM innovations by
  applying a correction to the input truncation parameter. We generated FLM sequences
  and truncated the initial L\'evy stable distribution (before Fourier transforming) at 
  a value of 0.5. After correlation structure is added, the width of the distribution 
  of fractional L\'evy noise decreases significantly. We corrected the input truncation
  parameter with our database resulting in a much more accurate distribution with 
  a maximum value close to 0.5.}\label{fig:truncation_correction}
  \end{figure}

  \clearpage
  \bibliography{stochastic_transport}

\end{document}
