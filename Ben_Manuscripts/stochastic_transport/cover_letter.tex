\documentclass[fontsize=11pt]{article}

\usepackage[margin=1in]{geometry}
\usepackage{graphicx}
\usepackage{wrapfig}
\usepackage{subcaption}
\usepackage{amsmath} 
\usepackage{siunitx}
\usepackage{booktabs}
\usepackage{gensymb}
\usepackage{hyperref}
\usepackage[export]{adjustbox}
\usepackage{anyfontsize}

\begin{document}

	\graphicspath{{./figures/}}

	%BJC: check address, phone number and fax (?) number
	\begin{figure}
	\centering
	\begin{minipage}{0.37\textwidth}
	\includegraphics[width=2.14in,left]{CUBoulder.pdf}
	\end{minipage}
	\begin{minipage}{0.62\textwidth}
	\scriptsize
	\noindent Department of Chemical and Biological Engineering \hfill t 303-735-7860~~~~~~~~~~~~~~~~~~ \\
	\noindent 596 UCB \hfill f 303-492-8425~~~~~~~~~~~~~~~~~~ \\
	\noindent Boulder, Colorado 80309 \hfill michael.shirts@colorado.edu \\
	\end{minipage}
	\end{figure}
	
	\noindent January 27\textsuperscript{th}, 2020\\
	%\noindent \date

	\noindent Dear Editors, \\
	
	\newcommand{\ManuscriptTitle}{Capturing Subdiffusive Solute Dynamics and 
	Predicting Selectivity in Nanoscale Pores with Time Series Modeling}
	
	We are submitting ``\ManuscriptTitle'' by Benjamin J. Coscia and
	Michael R.  Shirts, for consideration for publication as a Research
	Article in Physical Review E. We expect this work to be of significant
	interest to researchers who study complex dynamics in both structured
	and amorphous soft materials.
	
	In this article, we use 5$\mu$s molecular dynamics (MD) simulations of
	solutes moving through the pores of an inverted hexagonal phase lyotropic
	liquid crystal (LLC) polymer membrane in order to parameterize two different
	stochastic time series models capable of predicting solute flux and selectivity
	on a macroscopic scale.  Realizations of both models quantitatively reproduce
	the mean squared displacements (MSDs) exhibited by the solutes in our MD
	simulations. We then demonstrate how we can use realizations orders of
	magnitude longer than our MD simulations in order to predict solute flux and
	selectivity.  This is among the first studies to apply time series modeling
	approaches in order to predict macroscopic transport properties from molecular
	dynamics trajectories.

	Based the chemical intuition gained from our previous studies, we
	formulate the stochastic models so that they share the same dynamical
	properties as those exhibited by solutes in our MD simulations. 
	Recently, we characterized transport mechanisms exhibited by a series
	of small polar solutes in an LLC Membrane (Coscia and Shirts \textit{J.
	Phys. Chem. B}, 123, 6314--6330 (2019)). We observed subdiffusive
	solute transport characterized by intermittent hops separated by
	periods of entrapment. 
	%In this work, we build on the chemical intuition
	%gained from our previous study in order to formulate stochastic models
	%which produce trajectory realizations that share the same dynamical
	%properties as those exhibited by solutes in our MD simulations. 
	In this work, we explore two approaches for modeling this behavior.
	First, based on the anomalous diffusion literature, we hypothesize that
	solutes are subject to either subordinated fractional Brownian or
	L\'evy motion. This implies that solutes experience a series of dwell
	times separated by anti-correlated hop lengths. Our second approach
	adds state-dependent dynamics to standard Markov state modeling
	approaches. We model solute motion as a series of transitions between
	various trapped states, each with different dynamical properties. \\
	
	\noindent Some suggestions for reviewers are:
	\begin{enumerate}

		\item Ralf Metzler is an expert in anomalous diffusion theory
			with applications to soft materials. (University of
			Potsdam, \href{mailto:kakania@uni-potsdam.de}{kakania @
			uni-potsdam.de})

		\item Aziz Ghoufi has studied molecular mechanisms of
			subdiffusive solute transport in nanostructured
			materials. (Institut de Physique de Rennes,
			\href:{mailto:aziz.ghoufi@univ-rennes1.fr}{aziz.ghoufi@univ-rennes1.fr})
		
		%BJC: elimelech is more experiment-focused
		%\item Menachem Elimelech is a leader in the development of
		%	membrane technologies for water desalination and water
		%	reuse (Yale University, 203-432-2789,
		%	\href{mailto:menachem.elimelech@yale.edu}{menachem.elimelech@yale.edu}).
	
		\item Richard Lueptow uses molecular simulations to understand
			transport processes in polymeric nanofiltration and
			reverse osmosis membranes. (Northwestern University,
			\href{mailto:r-lueptow@northwestern.edu}{r-lueptow@northwestern.edu})

		%BJC: I don't think Paddison is as strong as the others

	\end{enumerate}
	
	\noindent Please send correspondence regarding this paper to Michael R. Shirts (contact
	details in letterhead).\\	
	
	\noindent Sincerely,
	
	\noindent Michael R. Shirts \\
	\noindent Benjamin J. Coscia \\
	
\end{document}

% LocalWords:  BJC
