\documentclass[fontsize=11pt]{article}

\usepackage[margin=1in]{geometry}
\usepackage{graphicx}
\usepackage{wrapfig}
\usepackage{subcaption}
\usepackage{amsmath} 
\usepackage{siunitx}
\usepackage{booktabs}
\usepackage{gensymb}
\usepackage{hyperref}
\usepackage[export]{adjustbox}
\usepackage{anyfontsize}

\begin{document}

	\graphicspath{{./figures/}}

	%BJC: check address, phone number and fax (?) number
	\begin{figure}
	\centering
	\begin{minipage}{0.37\textwidth}
	\includegraphics[width=2.14in,left]{CUBoulder.pdf}
	\end{minipage}
	\begin{minipage}{0.62\textwidth}
	\scriptsize
	\noindent Department of Chemical and Biological Engineering \hfill t 303-735-7860~~~~~~~~~~~~~~~~~~ \\
	\noindent 596 UCB \hfill f 303-492-8425~~~~~~~~~~~~~~~~~~ \\
	\noindent Boulder, Colorado 80309 \hfill michael.shirts@colorado.edu \\
	\end{minipage}
	\end{figure}
	
	%\noindent May 3\textsuperscript{rd}, 2019\\
	\noindent \date

	\noindent Dear Editors,\\
	
	\newcommand{\ManuscriptTitle}{Capturing Subdiffusive Solute Dynamics and 
	Predicting Selectivity in Nanoscale Pores with Time Series Modeling}
	
	We are submitting ``\ManuscriptTitle'' by Benjamin J. Coscia and Michael R.
	Shirts, for	consideration for publication as a Research Article in Physical 
	Review E. We expect this work to be of significant interest to researchers 
	who study complex dynamics in both structured and amorphous soft materials.
	
	In this article, we use solute trajectories generated from 5$\mu$s molecular
	dynamics (MD) simulations of an inverted hexagonal (H\textsubsript{II}) phase
	self-assembled lyotropic liquid crystal (LLC) polymer membrane in order to 
	parameterize two different stochastic time series models capable of predicting
	solute flux and selectivity on a macroscopic scale. Realizations of both models
	quantitatively reproduce the mean squared displacements (MSDs) exhibited by 
	solutes in MD simulations. This is among the first studies to apply time 
	series modeling approaches in order to predict bulk materials properties. 

	Recently, we characterized transport mechanisms exhibited by a series of 
	small polar solutes in an LLC Membrane (Coscia and Shirts \textit{J. Phys.
	Chem. B}, 123, 6314--6330 (2019)). We observed subdiffusive solute transport
	characterized by intermittent hops separated by periods of entrapment. In
	this work, we build on the chemical intuition gained from our previous study
	in order to formulate stochastic models which produce trajectory realizations
	that share the same dynamical properties as those exhibited by solutes in 
	our MD simulations. We explore two modeling approaches. First, based on the
	anomalous diffusion literature, we hypothesize that solutes are subject 
	to subordinated fractional Brownian or L\'evy motion. This implies that 
	solutes experience a series of dwell times separated by anti-correlated hop
	lengths. Our second approach adds state-dependent dynamics to standard Markov
	state modeling approaches. We model solute motion as a series of transitions
	between various trapped states, each with different dynamical properties.
	
%        Recently, we characterized an atomistic molecular model of an
%        LLC membrane that is maximally consistent with experimental
%        observations (Coscia et al. \textit{J. Phys. Chem. B}, 123,
%        289--309 (2019)). In this work, we observe and quantify
%        transport of water, sodium and 20 small polar solutes within
%        the pores of our atomistic H\textsubscript{II} phase LLC
%        membrane model. In general, all solutes exhibit subdiffusive
%        transport behavior characterized by intermittent hops between
%        long periods of entrapment. However, due the pores'
%        inhomogeneous architecture, we observed 3 different mechanisms
%        of entrapment. First, solutes can diffuse out of the pores and
%        become entangled between tails. Second, many of the solutes
%        are capable of donating hydrogen bonds to monomer head groups
%        for extended periods of time. Finally, solutes can associate
%        with monomer head group counter-ions within nanopores. Each
%        solute is influenced by each mechanism to varying degrees
%        dependent on its size and chemical functionality. We begin our
%        discussion with a broad description of these mechanisms in the
%        context of all solutes before restricting the discussion to
%        more detailed analyses of subsets of chemically-similar
%        solutes.
	
	%BJC: included phone numbers because it asked to do so in author's guide.
	\noindent Some suggestions for reviewers are:
	\begin{enumerate}

		\item Menachem Elimelech is a leader in the development of membrane technologies
		for	water desalination and water reuse (Yale University, 203-432-2789,
		\href{mailto:menachem.elimelech@yale.edu}{menachem.elimelech@yale.edu}).
	
		\item Francisco Hung has used molecular simulations to study nematic liquid crystals
		and various	nanostructured materials (Northeastern University, 617-373-8619 	
		\href{mailto:f.hung@northeastern.edu}{f.hung@northeastern.edu})
		
		\item Mahesh Mahanthappa has experience modeling soft materials using molecular dynamics
		including work on ion transport in various ordered liquid crystalline phases. 
		(University of Minnesota, 612-625-4599,\href{mailto:maheshkm@umn.edu}{maheshkm@umn.edu})
		
	\end{enumerate}
	
	\noindent Please send correspondence regarding this paper to Michael R. Shirts (contact
	details in letterhead).\\	
	
	\noindent Sincerely,
	
	\noindent Michael R. Shirts \\
	\noindent Benjamin J. Coscia \\
	
\end{document}

% LocalWords:  BJC
