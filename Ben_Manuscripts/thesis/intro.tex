\chapter{Introduction}\label{chapter:intro}

  \section{Membranes for Selective Separations}
  
  \begin{itemize}
  
    \item Membranes are useful separation tools. Cover historical usage
	\item Some discussion of competing separation technologies. 
	\item They are more efficient and have a smaller footprint than thermal distillation.

  \end{itemize}
  
  \subsection{How Small Molecule Membrane Separations Work}

  \begin{itemize}
  
    \item Maybe briefly mention ultrafiltration, microfiltration
    \item Overview of reverse osmosis and nanofiltration.
    \item Separation mechanisms (pore flow versus solution diffusion)
  
  \end{itemize}
  
  \subsection{Membrane Separation Applications}
  
  \begin{itemize}
  	\item Desalination
  	\item Separation of Organic Micropollutants
  	\item Recovery of Valuable Dissolved Species
  	\item Breathable Barriers
  \end{itemize}

  \section{Competing Membrane Technologies}

  \subsection{Amorphous Membranes}
  
  \begin{itemize}  
  
  \item Mostly about the industry standard, thin film composite polyamide membranes.
  
  \end{itemize}
  
  \subsection{Nanostructured Membranes}
  
  \begin{itemize}  
  
  \item Graphene
  \item Carbon nanotubes
  \item zeolites
  \item MOFs (?) %BJC: these have mostly been used for gas separations. But I've seen some work with aqueous. Not sure if worth covering.
  %MRS: a type of nanostructuring, so worth a little bit. 
  \end{itemize}
  
  \section{Cross-linked Self Assembled Liquid Crystal Membranes for Selective Aqueous Separations}
  
  \begin{itemize}    
  
  \item A less studied class of membranes, LLCs, have the potential to outperform competing membranes.
  %MRS: be specific about peformance.  Probably have lower permeability, but higher selectivity.  
  \end{itemize}
  
  \subsection{The H\textsubscript{II} Phase}
  
  \begin{itemize}
    \item Ideal geometry for transport
    \item Go through history
    \item Highlight issues with synthesis
    \item Talk about alignment approaches by Xunda Feng.
  \end{itemize}
  
  \subsection{The Q\textsubscript{I} Phase}
  
  \begin{itemize}
    \item Easier to synthesize
    \item History of monomers used
  \end{itemize}
  
  \section{Atomistic Molecular Simulation of LLC Membranes}
  
  \begin{itemize}
    \item Why MD?
    \item Why HII phase?
    \item What work has been done already
  \end{itemize}

  Questions to be answered: %BJC: one major question per chapter
  \begin{enumerate}
    %\item How do we build and equilibrate these systems? 
    \item What is the nanoscopic structure?  %MRS: good instinct to focus on this one. Build and equilibrate is a 'how' that allows you to get at this more important question.  
    \item What factors affect solute transport?  %MRS: make ``factors'' more specific and physical.
    \item Can we estimate macroscopic properties?
    \item How can we learn mechanisms with minimal human intervention?  %BJC: i.e. machine learning methods
  \end{enumerate} 
