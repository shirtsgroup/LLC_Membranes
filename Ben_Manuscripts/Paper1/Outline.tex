\documentclass{article}
\usepackage{graphicx}
\usepackage{wrapfig}
\usepackage{subcaption}
\usepackage{geometry}
\geometry{legalpaper, margin=1in}
\usepackage{amsmath} % or simply amstext
\newcommand{\angstrom}{\textup{\AA}}
\title{Predicting Transport in Lyotropic Liquid Crystal Membranes with Molecular Dynamics Simulations -- Outline}
\author{Benjamin J. Coscia \and Douglas L. Gin \and Richard D. Noble \and Joe Yelk \and Matthew Glaser \and Xunda Feng \and Michael R. Shirts} 
\begin{document}
  \bibliographystyle{ieeetr}
  \graphicspath{{./figures/}}
  \maketitle
  \section*{Introduction}
  
  Nanostructured membrane materials have become increasingly popular for 
  aqueous separations applications such as desalination and biorefinement
  because they offer the ability to control membrane architecture at the
  atomic scale allowing the design of solute-specific separation membranes. \cite{humplik_nanostructured_2011}
  \begin{itemize}
    \item Most membrane-based aqueous separations of small molecules can 
    be achieved using reverse osmosis (RO) or nanofiltration (NF) \cite{van_der_bruggen_review_2003}
  \end{itemize}
  
  While RO and NF have seen many advances in the past few decades, they 
  are far from perfect separation technologies.
  \begin{itemize}
    \textit{RO membranes}
    \item \textbf{Inconsistent performance} : Current state-of-the-art RO membranes are unstructured with
    tortuous and polydisperse diffusion pathways which leads to 
    inconsistent performance \cite{song_nano_2011}
    \item \textbf{High energy requirements} : Necessarily high feed pressures 
    drive up energy requirements which strains developing regions and
    contributes strongly to CO\textsubscript{2} emissions. \cite{mcginnis_global_2008}
    \item \textbf{Separation based on differences in solubility and diffusivity:
    Moreover, designing RO membranes to achieve targeted separations of 
    specific solutes is nearly impossible because various solutes dissolve
    into and diffuse through the polymer matrix at different rates. \cite{wijmans_solution-diffusion_1995}
    \item At best, one can exploit these differences to create a functional
    selective barrier.
    \textit{NF membranes}
    \item NF was introduced as an intermediate between RO and ultrafiltration,
    having the ability to separate organic matter and salts on the order of 
    one nanometer in size.
    \item Larger and well-defined pores drive down energy requirements while
    still affording separation of solutes as small as ions to some degree \cite{van_der_bruggen_review_2003}
    \item NF is often used as a precursor to reverse osmosis
    \item Unfortunately, NF membranes, like RO, are produced with a pore size 
    distribution which limits their ability to perform precise separations \cite{bowen_modelling_2002}
  \end{itemize}
  
  Nanostructured membranes can bypass many of the performance issues which
  plague traditional NF and RO membranes.
  \begin{itemize}
    \item \textbf{Tune size and functionality of building blocks} to control pore
    size and shape: One can accomplish targeted separations with high 
    selectivity by tuning shape, size and functionality of the molecular
    building blocks which form these materials. % BJC: "these materials" --> "nanostructured membranes", or is that redundant?
    \item As a result, solute rejecting pores can have their sizes tuned
    uniformly, resulting in \textbf{strict size cut-offs}.
    \item Entirely \textbf{different mechanisms} may govern transport in a given
    nanostructured material which can inspire novel separation techniques.
  \end{itemize}
  
  Development of nanostructured materials has been limited by the ability
  to synthesize and scale various fundamentally sound technologies.
  \begin{itemize}
    \item \textbf{Graphene sheets} are atomically thick which results in excellent
    permeability but defects during manufacturing severely impact 
    selectivity. \cite{cohen-tanugi_multilayer_2016}
    \item Molecular dynamics simulations of \textbf{carbon nanotubes} show
    promise \cite{humplik_nanostructured_2011} but synthetic techniques are 
    unable to achieve scalable alignment and pore monodispersity.\cite{hata_water-assisted_2004,maruyama_growth_2005}
    \item \textbf{Zeolites} have sub-nm pores with a narrow pore size 
    distribution and MD simulations exhibit complete rejection of solvated ions, \cite{murad_molecular_1998}
    however, experimental rejection was low and attributed to interstitial
    defects formed during membrane synthesis \cite{li_desalination_2004}
    \item There is a need for a scalable nanostructured membrane
  \end{itemize}
  
  Self assembling lyotropic liquid crystals (LLCs) are a suitable candidate for
  aqueous separation applications. 
  \begin{itemize}
    \item LLCs share the characteristic ability of nanostructured membrane
    materials to create \textbf{highly ordered structures} with the added benefits
    of \textbf{low cost} and synthetic techniques feasible for 
    \textbf{large scale production} \cite{feng_scalable_2014}
    \item Neat liquid crystal monomer forms the thermotropic, Col\textsubscript{h}
    phase. The presence of small amounts of water results in the H\textsubscript{II} 
    phase.
    \item In both cases, monomers assemble into mesophases made of hexagonally
    packed, uniform size, cylinders with hydrophilic groups oriented inward
    towards the pore center and hydrophobic groups facing outward.
    \item H\textsubscript{II} and Col\textsubscript{h} phase systems created by
    the monomer named Na-GA3C11 has been extensively studied experimentally \cite{smith_ordered_1997, %BJC: IUPAC chemical name here?
    zhou_supported_2005,resel_h2-phase_2000,feng_scalable_2014,feng_thin_2016}. 
    \item Until recently, mesophases formed by Na-GA3C11 could not be macroscopically
    aligned, resulting in a low flux membrane, slowing research in the field.
    \item In 2014, Feng et al. showed that the mesophases could be aligned 
    using a magnetic field with subsequent crosslinking to lock the structure
    in place \cite{feng_scalable_2014}
    \item In 2016, Feng et al. showed that the same result could be obtained 
    using a technique termed soft confinement \cite{feng_thin_2016}.
    \item Following this breakthrough, research into LLC membranes has been
    reinvigorated
  \end{itemize}
  
  A clear picture of the nanoscopic LLC membrane structure, gained by building 
  a molecular model will provide evidence to answer existing and newly proposed
  questions.
    \begin{itemize}
    \item We will limit our initial studies to studying assemblies formed by Na-GA3C11
    \item We have also chosen to focus our initial efforts on the development of 
    a model of the Col\textsubscript{h} phase membrane.
    \item Compared to the H\textsubscript{II} phase, the Col\textsubscript{h}
    phase is a simpler starting point, due to the absence of water, and has
    equivalent experimental structural data.
    \item Despite the structural data, the
  
  A molecular level understanding of LLC membrane structure will elucidate
  small molecule transport mechanisms, providing guidelines to reduce the
  chemical space for the design of monomers used to create separation-specific
  membranes.
    \begin{itemize}
    \item LLCs are versatile and controllable with a \textbf{large chemical design
    space} available for membrane design
    \item It will be challenging to narrow down the design space and optimize 
    solute-specific membranes
    \item  We do not yet understand how to reduce the effective pore size or 
    how to tune the chemical environment in the nanopores for effective water
    desalination or small organic separations.
    \item 
    \end{itemize}
  
  We must show that the developed molecular model is consistent with
  physical observations so that we can rely on conclusions drawn about % better word for trust?
  structural features characteristic of the system.
