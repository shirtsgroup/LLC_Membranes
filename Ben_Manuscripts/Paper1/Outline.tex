\documentclass{article}
\usepackage{graphicx}
\usepackage{wrapfig}
\usepackage{subcaption}
\usepackage{geometry}
\geometry{legalpaper, margin=1in}
\usepackage{amsmath} % or simply amstext
\newcommand{\angstrom}{\textup{\AA}}
\title{Predicting Transport in Lyotropic Liquid Crystal Membranes with Molecular Dynamics Simulations -- Outline}
\author{Benjamin J. Coscia \and Douglas L. Gin \and Richard D. Noble \and Joe Yelk \and Matthew Glaser \and Xunda Feng \and Michael R. Shirts} 
\begin{document}
  \bibliographystyle{ieeetr}
  \graphicspath{{./figures/}}
  \maketitle
  \section*{Introduction}
  
  Nanostructured membrane materials have become increasingly popular for 
  aqueous separations applications such as desalination and biorefinement
  because they offer the ability to control membrane architecture at the
  atomic scale allowing the design of solute-specific separation membranes.
  \begin{itemize}
    \item Most membrane-based aqueous separations of small molecules can 
    be achieved using reverse osmosis (RO) or nanofiltration (NF)
  \end{itemize}
  
  While RO and NF have seen many advances in the past few decades, they 
  are far from perfect separation technologies.
  \begin{itemize}
    \textit{RO membranes}
    \item \textbf{Inconsistent performance} : Current state-of-the-art RO membranes are unstructured with
    tortuous and polydisperse diffusion pathways which leads to 
    inconsistent performance
    \item \textbf{High energy requirements} : Necessarily high feed pressures 
    drive up energy requirements which strains developing regions and
    contributes strongly to CO\textsubscript{2} emissions.
    \item \textbf{Separation based on differences in solubility and diffusivity:
    Moreover, designing RO membranes to achieve targeted separations of 
    specific solutes is nearly impossible because various solutes dissolve
    into and diffuse through the polymer matrix at different rates.
    \item At best, one can exploit these differences to create a functional
    selective barrier.
    \textit{NF membranes}
    \item NF was introduced as an intermediate between RO and ultrafiltration,
    having the ability to separate organic matter and salts on the order of 
    one nanometer in size.
    \item Larger and well-defined pores drive down energy requirements while
    still affording separation of solutes as small as ions to some degree
    \item NF is often used as a precursor to reverse osmosis
    \item Unfortunately, NF membranes, like RO, are produced with a pore size 
    distribution which limits their ability to perform precise separations
  \end{itemize}
  
  Nanostructured membranes can bypass many of the performance issues which
  plague traditional NF and RO membranes.
  \begin{itemize}
    \item \textbf{Tune size and functionality of building blocks} to control pore
    size and shape: One can accomplish targeted separations with high 
    selectivity by tuning shape, size and functionality of the molecular
    building blocks which form these materials. % BJC: "these materials" --> "nanostructured membranes", or is that redundant?
    \item As a result, solute rejecting pores can have their sizes tuned
    uniformly, resulting in \textbf{strict size cut-offs}.
    \item Entirely \textbf{different mechanisms} may govern transport in a given
    nanostructured material which can inspire novel separation techniques.
  \end{itemize}
  
  Development of nanostructured materials has been limited by the ability
  to synthesize and scale various fundamentally sound technologies.
  \begin{itemize}
    \item \textbf{Graphene sheets} are atomically thick which results in excellent
    permeability but defects during manufacturing severely impact 
    selectivity.
    \item Molecular dynamics simulations of \textbf{carbon nanotubes} show
    promise but synthetic techniques are unable to achieve scalable alignment 
    and pore monodispersity.
    \item \textbf{Zeolites} have sub-nm pores with a narrow pore size 
    distribution and MD simulations exhibit complete rejection of solvated ions,
    however, experimental rejection was low and attributed to interstitial
    defects formed during membrane synthesis
  
  Self assembling lyotropic liquid crystals (LLCs) are a suitable candidate
  for aqueous separation applications.
  
  A molecular level understanding of LLC membrane structure will elucidate
  small molecule transport mechanisms, providing guidelines to reduce the
  chemical space for the design of monomers used to create separation-specific
  membranes.
  
  A clear picture of the nanoscopic LLC membrane structure, gained by building 
  a molecular model will provide evidence to answer existing and newly proposed
  questions.
  
  We must show that the developed molecular model is consistent with
  physical observations so that we can rely on conclusions drawn about % better word for trust?
  structural features characteristic of the system.
