\documentclass{article}
\usepackage{graphicx}
\usepackage{wrapfig}
\usepackage{subcaption}
\usepackage{geometry}
\geometry{legalpaper, margin=1in}
\usepackage{amsmath} % or simply amstext
\newcommand{\angstrom}{\textup{\AA}}
\title{Predicting Transport in Lyotropic Liquid Crystal Membranes with Molecular Dynamics Simulations -- Outline}
\author{Benjamin J. Coscia \and Douglas L. Gin \and Richard D. Noble \and Joe Yelk \and Matthew Glaser \and Xunda Feng \and Michael R. Shirts} 
\begin{document}
  \bibliographystyle{ieeetr}
  \graphicspath{{./figures/}}
  \maketitle
  \section*{Introduction}
  
  Nanostructured membrane materials have become increasingly popular for 
  aqueous separations applications such as desalination and biorefinement
  because they offer the ability to control membrane architecture at the
  atomic scale allowing the design of solute-specific separation membranes
  
  While RO and NF have seen many advances in the past few decades, they 
  are far from perfect separation technologies.
  
  Nanostructured membranes can bypass many of the performance issues which
  plague traditional NF and RO membranes.
  
  Development of nanostructured materials has been limited by the ability
  to synthesize and scale various fundamentally sound technologies.
  
  Self assembling lyotropic liquid crystals (LLCs) are a suitable candidate
  for aqueous separation applications.
  
  A molecular level understanding of LLC membrane structure will elucidate
  small molecule transport mechanisms, providing guidelines to reduce the
  chemical space for the design of monomers used to create separation-specific
  membranes.
  
  A clear picture of the nanoscopic LLC membrane structure, gained by building 
  a molecular model will provide evidence to answer existing and newly proposed
  questions.
  
  We must show that the developed molecular model is consistent with
  physical observations so that we can rely on conclusions drawn about % better word for trust?
  structural features characteristic of the system.
