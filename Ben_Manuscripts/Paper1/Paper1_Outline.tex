\documentclass{article}
\title{Predicting Transport in Lyotropic Liquid Crystal Membranes with Molecular Dynamics Simulations -- Outline}
%MRS2: read over the start of the grant for more ideas about details for the motivation.
\author{Benjamin J. Coscia \and Douglas L. Gin \and Richard D. Noble \and Michael R. Shirts} 
\begin{document}
	\bibliographystyle{ieeetr}
	\maketitle
	\section{Introduction}
	Nanostructured membrane materials have become increasingly popular in desalination and wastewater treatment applications because they offer the ability to control membrane architecture at the atomic scale.
	\begin{itemize}
		\item Current state-of-the-art reverse osmosis membranes are dense and unstructured with tortuous and polydisperse pores which lead to inconsistent performance
		\item Tortuosity and polydispersity drive up energy requirements which strain developing regions and contribute strongly to CO2 emissions
		\item With nanostructured materials, solute rejecting pores can be tuned uniformly -- drives down energy requirements
	\end{itemize}
	
	Development of nanostructured materials has been limited by the ability to synthesize and scale various fundamentally sound technologies.
	\begin{itemize}
		\item Leading technologies and their limitations:
		\begin{itemize}
			\item Graphene sheets -atomically thick which gives excellent permeability but defects during manufacturing severely impact performance
			\item Carbon Nanotubes - MD studies are promising but synthetic techniques unable to achieve necessary alignment and pore monodispersity
		\end{itemize}
	\end{itemize} 
	Self assembling lyotropic liquid crystals (LLCs) share the characteristic ability of nanostructured membrane materials to create highly ordered structures with the benefits of low cost and synthetic techniques feasible for large scale production.
	\begin{itemize}
		\item Forms lamellar, bicontinuous cubic and hexagonal phase based on solution composition
		\item My monomer form two types of self assembled LLCs - thermotropic (Colh) and lyotropic (HII) 
		\item The thermotropic, Colh is formed by the self assembly of neat monomer
		\item The lyotropic, HII phase is formed in the presence of small percentages of water (around 8 percent)
		\item Both assemble into cylinders with hydrophilic groups oriented inward towards the pore center and hydrophobic group facing outward. The only difference is the inclusion of water in the structure which leads to minor variations in the structure with potentially different filtration properties (although no filtration experiments have been done on Colh)  
		\item Hydrophilic regions point towards pore centers
		\item Until recently, they could not be aligned - hindered progress
		\item Yale aligns them, then crosslinks them to lock in the structure
		\item LLC HII phase membranes offer potential for high permeability and selectivity which equals low energy consumption
		\item The Colh phase shares the same structural features with the HII phase with the exception of the presence of water. This paper will focus on the development of a model of the Colh phase since it is a simpler starting point and has just as much experimental data. The analysis used in this paper can be readily extended to the HII phase. 
                  %MRS: Doesn't the data appear to be that they are mostly the same in structure? Should address this to the extent we can.
		  %BJC: They're structures are almost identical (same d100 plane spacing, monomer orientation) with the exception of water being present in the HII phase. I rephrased the last item to make that clearer.
	\end{itemize}
	

	% BJC - I think that this paragraph will not be included since we are not getting into mechansims at all
        %MRS2: some of this is good motivation for WHY we care about the structure of the pore.  Without the structure, we
        % can't understand the mechanism.
        A molecular level understanding of solute transport in LLC membranes will accelerate development efforts by providing guidelines for monomer design.
        \begin{itemize}
                \item Two types of self assembled LLCs have been made - thermotropic and lyotropic -- explain difference -- shift focus to Colh
                \item Colh phase studies currently limited mostly to one monomer with minor variations
                \item Current optimization efforts performed through trial and error
                \item Macroscopic models are the only source of predictive modeling
                \item An atomistic understanding of the mechanism of solute transport can identify
                performance bottle necks and direct design of future monomers/membranes
                \begin{itemize}
                        \item Do ions have trouble getting through because of interactions with other things in the pores (e.g. ions, carbonyl groups, benzene rings) -- related to ionic conduction
                        \item Does concentration of ions in the pore repel incoming ions?
                \end{itemize}
        \end{itemize}

        %MRS2: correct, focus should be on the picture (the result) not the model (the means to obtain the result)
	A clearer picture of the microscopic structure of LLC membranes, gained by building a molecular model, will change how we think about their design. 
	\begin{itemize}
		\item The arrangement of sodium ions in the channels is thought to be confined to the pore walls. It is possible they are arranged more randomly
		\begin{itemize}
			\item This could change how one thinks about molecules diffusing through membrane
			\item Could also be a difference between lyotropic and thermotropic phases
		\end{itemize}
		\item The Colh phase is described as having pores made of disks or layers stacked on top of one another, each containing a set number of monomers. 
		\item There is no clear answer in literature defining what that set number of monomers is. I'd like to confirm or deny past predictions
		\begin{itemize}
			\item A simple simulation study of a similar molecule (Head group is a sulfonate in the meta position) suggests that there are 4 monomers in each disk
			\item Calculations based on the volume of the liquid crystal suggest that there are seven monomers in each layer
			\item We can directly compare the plausibility of each situation along with other numbers of monomers per layer - 5, 6, 8
			\item We can also test other, less simplistic arrangements - such as having alternating monomers per layer  %BJC: maybe leave mention of this out right and here and just address it in the results?  
       %MRS2: right.  Generally, you want to use the introduction to raise questions, not list everything you will do.
		\end{itemize}
	\end{itemize}
	
        %MRS: this is not the best way to phrase the thesis.  I would say that we want to ask what the most likely features of the membrane are using molecular modeling.  we can talk about the phrasing.  But the model should not be the scientific output of the paper. 
	%BJC: Okay, tried again below, - basically what I want to communicate is that we want our model to tell us useful things and we can't do that unless we know that the model is realistic. We can have a good idea that it's realistic if we put thought into how we build it and if we can reproduce experimental results. I also want to close off the conclusion since it should be a good segway into the Methods section.
        %MRS2: fine for now.  I'll think about better phrasing as we go.
	To trust our understanding of which structural features are most likely characteristic of the system, and the conclusions drawn thereof, we must show that our molecular model is physically realistic.   

    
	% Old thesis: A physically realistic molecular model with easily modified structural features will enable a qualitatively different type of membrane design aimed optimization of self assembled LLC membranes.
	\begin{itemize}
		\item This paper will illustrate the careful development of a predictive molecular model and the steps taken to ensure it mimics the real system as best we can 
		\item To understand how physically realistic the model is, validation by comparison to experiment is necessary
		\item We are primarily interested in reproducing the conclusions about structure which can be made from SAXS experiments, predicting ionic conductivity with a reasonable comparison to experiment, and reproducing experimental density measurements.          
	\end{itemize}
	
	\section{Methods}
	
	HII monomers were parameterized using the Generalized Amber Forcefield with the Antechamber package provided with AmberTools16. All molecular dynamics simulations were run using Gromacs version 5.1.2.

	An ensemble of characteristic, low-energy vacuum monomer configurations were constructed by applying a simulated annealing process to a parameterized monomer.
	\begin{itemize}
          %MRS: this first about unfavorable structure not relevant - people will understand it needs to equilibrated/annealed.
	  %BJC: But still talk about what I did to anneal it right? 
          %MRS2: Yes.
		\item Initially parameterized monomer gives unfavorable structure % Will delete this once I clarify the above comment
		\item Structure cooled from 1000 to 50 K over 10 nanoseconds
		\item Result not global minimum but close enough for structure building
		\item Multiple configurations saved from annealing trajectory to prove independence of starting config
		\item New configurations used for reparamaterization, BUT most likely will just use one set of charges from one configuration since I've done 90 percent of simulations based on one parameterization. In the future, I will have a single set of parameters. So, I have multiple configurations all with the same set of charges for now.
%MRS2: now is the time to decide exactly what will be presented in the paper regarding alternative configurations/charges.
	\end{itemize}

	The timescale for self assembly of monomers into the hexagonal phase is unknown and likely outside of a reasonable length for an atomistic simulation. 
	\begin{itemize}
		\item Work done that shows coarse grain model self assembly in ~1000 ns , Citation: J. Phys. Chem. B 2013, 117, 4254-4262
		\item Attempts with Colh system not fruitful  
		\begin{itemize}
			\item Packed monomers into box
			\item Simulated for ~100 ns with no real progress shown towards self assembly
		\end{itemize}  
		\item Wrote code to assemble monomers into Colh configuration (described in intro) close to what is expected 
		\item Equilibration simulations allow structure to relax into expected configuration 
	\end{itemize}

	Twenty monomer layers per pore provides a balance of structural accuracy and computational efficiency. 
        %MRS2: describe quantitatively how you decided.  What is the information that made you make the decision?
	\begin{itemize}
		\item Space between membrane layers in z direction - semi-isotropic sims fix z box dimension
		\item Small number of layers create micellar structure
		\item Beyond 20 layers is unnecessary to get expected configuration
		\item layers spaced far enough apart to avoid large energy repulsions but close enough to prevent disks slipping between one another ~ 5 angstroms works.
	\end{itemize}
	
        %MRS: A lot of these investigations probably belong in results, because they are answering the questions you have: what are the likely physical properties of these membranes? The parameters are actually variables you are investgating. 
	%BJC: Okay, condensed a couple paragraphs into one. I just want to explain here where the measurements came from and how 
	\noindent Initial guesses for the remaining structural parameters were chosen based on experimental data and treated as variables during model development 
	\begin{itemize}
		\item SAXS gives an idea of Pore-to-Pore distances
		\item TEM images and rejection give a pore size estimate
		\item Ran long NPT simulations at 300 K and 1 bar ( > 200 ns )
	\end{itemize}
	
        %MRS: Results of the crosslinking are likely in results -- the fact that it didn't really change the structure of the membrame. 
	%BJC: Okay, then I will just give the gist of how I went about crosslinking and why I think it is the correct way to do it.
	Using an equilibrated structure, a crosslinking procedure was performed in order to better parallel synthetic procedures. 
	\begin{itemize}
		\item Crosslinking maintains alignment of cylindrical mesophases - emphasize that replicating the mechanism/kinetics is not important 
		\item head to tail addition dominates so I only implemented that
		\item racemic mixture - don't have to be too concerned about direction of attack 
		\item Details of crosslinking algorithm (refer to appendix or supplemental info but give a brief overview here)
	\end{itemize}  
	
	\section{Results and Discussion}

	Proper selection of initial structural parameters results in an equilibrated model that is experimentally consistent 
	\begin{itemize}
		\item Visual perspective - show top view and cross section
		\item System validated using multiple monomer configurations 
		\item Pore to pore distance
		\begin{itemize}
			\item We know from SAXS data what the distance between pores should be
			\item We require long simulation times to reach an equilibrated structure with the correct dimensions
			\item Crosslinking locks in structure, maintaining pore distances even after long simulations
		\item Pore Radius - a less reliable validation because we don't have an agreed upon way to measure this parameter experimentally
		\item Density - there are crude lab measurements which my model is in agreement with (no one has reported any values since it wasn't relevant to them)
		\end{itemize}
	\end{itemize}
	
	The monomers are arranged into disks containing XXXX monomers in each layer. (Or are there defined layers -- I need to test this idea a bit more) 
%MRS2: yes, examine this more. The idea of # of defined layers may be another idea to challenge. 
	\begin{itemize}
		\item Hypothesis: There are an average of 7 monomers per layer when defined per unit volume but there are not well-defined layers as pictured in the literature. While long range order is maintained, hexagonal mesophases are disordered within their hydrophilic and hydrophobic domains. Staying completely ordered, stacked on top of each other is not entropically favorable.
	        \item It has been suggested that there are 4 monomers in a each disk/layer, however, simulations have shown that this leads to unstable configurations with dimensions that are too small compared to experiment.
                \item Stable simulations with 6 and 7 monomers have been performed (will do one with five soon) which give structural characteristics consistent with experiment
                \item Stable systems have also been simulated consisting of varying numbers of monomers per layer
                \item This suggests that the arrangement of monomers is more complicated than simple layers stacked on top of each other
	\end{itemize}
%MRS2: Need to put into the outline how you quantitatively decided that layers wasnt meaningful.  Atom density as a function of Z, showed that it was uniform, not periodic? 
	Sodium counterions are distributed randomly within the hydrophilic pore regions. (I should actually come up with a distribution e.g. gaussian) 
%MRS2: right, there are many ways to be distributed randomly.
	\begin{itemize}
		\item Past literature reports ions arranged in a circle about pore walls implying some kind of void space as a pore. Simulations under a variety of conditions suggest that ions prefer disorder within the pores. 
		\item A size exclusion mechanism has been proposed in the past, however that might no be the only force at play
		\item Ions may play a role in transport, hindering flux of solute and solvent by slowing their diffusion
		\item The membrane may exhibit the permeability-selectivity tradeoff inherent to solution-diffusion type membranes. 
%MRS2 good to get into the implied mechanism.
        \end{itemize}	

	The model gives reasonable estimates of ionic conductivity.
	\begin{itemize}
		\item There are a few ways to estimate ionic conductivity as seen in literature. We prefer a method which can extract an estimate based purely on an equilibrium trajectory
		\item We must also be sure that our analysis of results is consistent with the method use for experimental evaluation (i.e. AC impedance spectroscopy)
		\item We must also link our perfectly straight microscopic system to the not-so-straight macroscopic system
		\item Two methods used to for prediction
		\item Nernst Einstein Relation:
		\begin{itemize}
			\item Widely used equation for estimating ionic conductivity
			\item Estimates DC ionic conductivity -- Frequency used during AC impedance slow enough to be approximated by dc at short enough timescales
			\item Relates the diffusive motion of ions in the membrane to the membrane's ionic conductivity
			\item Concentration is concentration of ions in the whole membrane, not just channels
		\end{itemize}
		\item Collective Diffusion:
		\begin{itemize}
			\item Defines a collective coordinate, Q (charge), to quantify the amount of charge transfer through the system
			\item In the limit of infinite time, the MSD of Q can be used to formulate a diffusion coefficient of Q that can be related to ionic conductivity
			\item The model is valid for non-equilibrium and equilibrium simulations. Our analysis is based on the latter
			\item A similar model has been derived and validate to predict water permeability using equilibrium simulations
			\item The pore region is defined as the entire membrane system since lab IC measurements are done on bulk membrane rather than on individual pores. One would expect single channel IC to be much larger than the bulk membrane
                \end{itemize}                    
	\end{itemize}

%MRS: not really a result, but can be included in the discussion. Don't need to talk too much about future work though - a paper should be focused on the current work.
%BJC: Okay, I'll be brief when writing this up
%MRS2: or not even enumerate all the specific questions, just that it is a generalizable model.
	The procedure used to create and validate our model can be used to evaluate other liquid crystalline assemblies
		\begin{itemize}
			\item Liquid crystal used - or a mixture of different liquid crystals
			\item Minor structural variations (e.g. 7, 8, 9, 10, 11 ... CH2's in the tails)
			\item Counterion -- Size/valence of counterion
			\item Functional head groups
			\item How does varying these things effect pore separation, pore size, phase stability, and transport and why are these effects observed
		\end{itemize} 
		
	\section{Conclusion}

        %MRS: restate in terms of the physical things learned.  THe output is knowledge, not a model.
	In this work, we have 
%MRS2: revealed 
%MRS: revealed is too strong.
suggested 
a more detailed picture of the structure of a self assembled thermotropic liquid crystal membrane using an atomistic molecular model.
	%In this work, a molecular model has been developed intended to predict transport in a thermotropic liquid crystal membrane. 
	\begin{itemize}
		\item Channels are more disordered than previously thought 
                  %MRS2: not only disordered, but filled, rather than open.
		\item There are likely 6 or 7 monomers per layer (or somewhere in between if there is something like alternating layers)
		\item Results presented for Colh phase monomer but can be adapted to other LCs
		\item Model can be used for prediction of transport properties in new membranes
		\item We will solvate the system to establish procedures for HII phase prediction
	\end{itemize}
%MRS2: should this be in there you discussed in the email?
%A crosslinked system to show that the system shrinks upon crosslinking as expected

%MRS2: This is worth examining, in the paper, but ONLY if it gives
%results like the X-ray diffraction shown earlier.  
% A system with forced pi stacking (via the dipole method) could help us explain (or not) the missing peak in the wide angle data

%MRS2: we should think about where new X-ray data  would go.
	
\end{document}
