\section{Methods}
	
HII monomers were parameterized using the Generalized Amber Forcefield
with the Antechamber package provided with AmberTools16. Charges were
assigned using tools from Openeye Scientific Software. All molecular dynamics
simulations were run using Gromacs version 5.1.2 and Gromacs version 2016.
	
An ensemble of characteristic, low-energy vacuum monomer configurations
were constructed by applying a simulated annealing process to a
parameterized monomer. Monomers were cooled from 1000K to 50K over 10
nanoseconds. A low energy configuration was pulled from the trajectory 
and charges were reassigned using the am1bccsym method of molcharge
shipped with Openeye Scientific software's QUACPAC. Using the new
charges, the monomer system was annealed again and multiple monomer
configurations were pulled from the trajectory.
	
The timescale for self assembly of monomers into the hexagonal phase is
unknown and likely outside of a reasonable length for an atomistic
simulation, calling for a more efficient way to build the system. 
Previous work has shown a coarse grain model self assemble into the HII
phase configuration in $\approx$ 1000 ns. Atomistic self-assembly was 
attempted by packing monomers into a box using Packmol. Simulations of >
100 ns show no progress towards an ordered system. To bypass the 
slow self-assembly process, python code was written in order that 
assembles monomers into a configuration close to the expected 
equilibrium configuration. A short, restrained equilibration, followed 
by NPT simulations > 500 ns, allows the initial configuration to relax
into an equilibrium configuration.

Each pore consists of twenty stacked monomer layers with periodic 
continuity in all directions, avoiding any edge effects and creating an
infinite length pore ideal for studying transport. A small number of
layers is preferred in order to reduce computational cost and to allow
us look at longer timescales. Ultimately, we chose to build a system
with 20 layers of monomer in each pore in order to give sufficient
resolution when simulating X-ray diffraction patterns.

% Questionably needed figure
%\begin{figure}
%\includegraphics{placeholder.png}
%	\caption{Monomers are placed in an initial configuration close to
%	         the expected equilibrium configuration and allowed to relax}
%	\label{fig:initial}
%\end{figure}

Initial guesses for the remaining structural parameters were chosen
based on experimental data and treated as variables during model
development. The distance between pores was based on experimental SAXS
data for this system. Initial configuration pore spacing was chosen to
be 10 \% larger than the experimental value. The layer spacing was based
on reflections from experimental WAXS data at 3.7 \angstrom, thought to
represent pi-stacking between stacked benzene rings. The tendency of 
simulations to equilibrate to slightly larger interlayer spacing 
inspired systems starting with layer spacings > 4 \angstrom.
% There is plenty I can say about initial configuration. Starting layers
% too far apart, too close, tilted initial configurations etc.
The relative interlayer orientation was chosen based on the various 
stacking modes of benzene and substituted benzene rings: sandwiched,
T-shaped and parallel-displaced. The T-shaped configuration was ruled
out based on the inconsistency of its preferred stacking distance of 
$\approx$ 5 \angstrom, with diffraction data. The system's preference
towards the sandwiched vs. parallel displaced stacking modes was left
as a variable. Initial pore radius estimates were based on past TEM 
images and size exclusion rejection data.    

An equilibration scheme with position restraints placed on benzene rings
prevents unrealistic jumps during early equilibration steps. Restraints
allow the system to remain in the sandwiched and parallel-displaced 
configurations while monomer tails settle. Doing so also helps prevent
system dependence on initial monomer configuration. During the restrained
equilibration, force constants are reduced by the square root of the 
previous force constant every 50 ps, with a starting value of 1e6 kJ
mol\textsuperscript{-1} nm\textsuperscript{-2}. Once the force constant
is below 10 KJ mol\textsuperscript{-1} nm\textsuperscript{-2}, the 
restraints are slowly released until there is no more restriaining 
potential. All restrained equilibrations are run in the NVT ensemeble. 
The resulting unrestrained structure is subsequently allowed to 
equilibrate in the NPT ensemble for > 200 ns.  

Determination of equilibration 
\begin{itemize}
	\item Pore spacing 
	\item Benzene ring stacking
	\item 

Using an equilibrated structure, a crosslinking procedure was performed
in order to better parallel synthetic procedures. The purpose of 
crosslinking is to maintain macroscopic alignment of the crystalline
domains, ensuring aligned, hexagonally packed pores. For that reason, we
are not concerned with replicating the kinetics of the reaction, but
instead emphasize the consistency of the final structure with experimental
structural data. The algorithm was developed based on the known reaction
mechanism. Crosslinking of this system is a free radical polymerization (FRP)
taking place between terminal alkyl groups present on each of the three
monomer tails. FRPs require an initiator which bonds to the system, 
meaning new atoms were introduced into the system. For simplicity, the 
initiator was simulated as hydrogen and made present in the simulation
by including them in all possible bonding positions as dummy atoms.
The crosslinking procedure is carried out iteratively. During each 
iteration, bonding carbon atoms are chosen based on a distance cut-off.
The topology is updated with new bonds and dummy hydrogen atoms are 
changed to appropriate hydrogen types. Head-to-tail addition was the
only propagation mode considered due to its dominance in real systems.
Direction of attack was not considered because the resultant mixture is
racemic. The resulting crosslinked structure has an even distribution of
crosslinks between monomer tails of the same monomer, monomers stacked on
top of each other and monomers in other pores, including across periodic
boundaries. The pore spacing shrinks by $\approx$ 1 nm and stays 
constant under a range of simulation conditions (see supplemental info). 

Simulated X-ray diffraction patterns were generated based on atomic
coordinates to give a deeper understanding of the pore structure and
spacing. All atomic coordinates were simulated as gaussian spheres of
electron density  corresponding to each atom's atomic number. A three
dimensional fourier transform (FT) of the array of electron density 
results in a 3 dimensional representation of the unit cell in reciprocal
space. Cross sections of the 3D FFT result in two dimensional reciprocal
space patterns which represent 2D diffraction patterns. Spherical
integration of the FT yields a 1 dimensional curve which is the same
as that generated in a diffraction experiment. We matched experimental
1D and 2D WAXS patterns by iterative improvement of our choice of 
initial structure and equilibration procedure 

% Ionic conductivity here probably
