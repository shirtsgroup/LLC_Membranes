\section{Methods}
	
HII monomers were parameterized using the Generalized Amber Forcefield
with the Antechamber package provided with AmberTools16. Charges were
assigned using tools from Openeye Scientific Software. All molecular dynamics
simulations were run using Gromacs version 5.1.2 and Gromacs version 2016.
	
An ensemble of characteristic, low-energy vacuum monomer configurations
were constructed by applying a simulated annealing process to a
parameterized monomer. Monomers were cooled from 1000K to 50K over 10
nanoseconds. A low energy configuration was pulled from the trajectory 
and charges were reassigned using the am1bccsym method of molcharge
shipped with Openeye Scientific software's QUACPAC. Using the new
charges, the monomer system was annealed again and multiple monomer
configurations were pulled from the trajectory.
	
The timescale for self assembly of monomers into the hexagonal phase is
unknown and likely outside of a reasonable length for an atomistic
simulation, calling for a more efficient way to build the system. 
Previous work has shown a coarse grain model self assemble into the HII
phase configuration in $\approx$ 1000 ns. Atomistic self-assembly was 
attempted by packing monomers into a box using Packmol. Simulations of >
100 ns show no progress towards an ordered system. To bypass the 
slow self-assembly process, python code was written in order that 
assembles monomers into a configuration close to the expected 
equilibrium configuration. A short, restrained equilibration, followed 
by NPT simulations > 500 ns, allows the initial configuration to relax
into an equilibrium configuration.

Each pore consists of twenty stacked monomer layers with periodic 
continuity in all directions, avoiding any edge effects and creating an
infinite length pore ideal for studying transport
(Fig.~\ref{fig:initial}). A small number of layers is preferred in order
to reduce computational cost and to allow us look at longer timescales.
Ultimately, we chose to build a system with 20 layers of monomer in each
pore in order to give sufficient resolution when simulating X-ray 
diffraction patterns.

% Questionably needed figure
\begin{figure}
\includegraphics{placeholder.png}
	\caption{Monomers are placed in an initial configuration close to
	         the expected equilibrium configuration and allowed to relax}
	\label{fig:initial}
\end{figure}

Initial guesses for the remaining structural parameters were chosen
based on experimental data and treated as variables during model
development. The distance between pores was based on experimental SAXS
data for this system. Initial configuration pore spacing was chosen to
be 10 \% larger than the experimental value. The layer spacing was based
on reflections from experimental WAXS data at 3.7 \angstrom, thought to
represent pi-stacking between stacked benzene rings. The tendency of 
simulations to equilibrate to slightly larger interlayer spacing 
inspired systems starting with layer spacings > 4 \angstrom.
% There is plenty I can say about initial configuration. Starting layers
% too far apart, too close, tilted initial configurations etc.
The relative interlayer orientation was chosen based on the various 
stacking modes of benzene and substituted benzene rings: sandwiched,
T-shaped and parallel-displaced. The T-shaped configuration was ruled
out based on the inconsistency of its preferred stacking distance of 
$\approx$ 5 \angstrom, with diffraction data. The system's preference
towards the sandwiched vs. parallel displaced stacking modes was left
as a variable. Initial pore radius estimates were based on past TEM 
images and size exclusion rejection data.    

An equilibration scheme with position restraints placed on benzene rings
prevents unrealistic jumps during early equilibration steps. Restraints
allow the system to remain in the sandwiched and parallel-displaced 
configurations while monomer tails settle. Doing so also helps prevent
system dependence on initial monomer configuration. During the restrained
equilibration, force constants are reduced by the square root of the 
previous force constant, with a starting value of 1e6.  by the
\begin{itemize}
	\item Equilibration scheme:
	\begin{itemize}
		\item Apply position restraints to monomer head groups during energy minimization 
		\item Leave position restraints on for nvt simulations to allow tails to intermingle (this also helps ensure independence of starting configuration)
		\item Gradually reduce force constants from 1000000 (by square root every 50 ps) until they are completely off
		\item Run long NPT simulations at 300 K and 1 bar ( $>$200 ns ) to fully equilibrate 
	\end{itemize}
\end{itemize}

Using an equilibrated structure, a crosslinking procedure was performed in order to better parallel synthetic procedures. 
\begin{itemize}
	\item Crosslinking maintains alignment of cylindrical mesophases - emphasize that replicating the mechanism/kinetics is not important 
	\item head to tail addition dominates so I only implemented that
	\item racemic mixture - don't have to be too concerned about direction of attack 
	\item Details of crosslinking algorithm (refer to appendix or supplemental info but give a brief overview here)
\end{itemize}  
	
Simulated X-ray diffraction patterns were generated based on atomic coordinates to give a deeper understanding of the pore structure and spacing. 
\begin{itemize}
	\item 3 dimensional fourier transformed electron density generates simulated 1D and 2D diffraction patterns
        \item The 1D patterns are generated by spherical integration of the FT
        \item 2D patterns are generated by taking cross sections of the FT in the qx, qy and qz planes
        \item We matched experiments based on iterative improvement of our choice in initial structure and equilibration procedure
\end{itemize}

% Ionic conductivity here probably
