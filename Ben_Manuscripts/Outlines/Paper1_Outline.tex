\documentclass{article}
\usepackage{graphicx}
\usepackage{wrapfig}
\usepackage{subcaption}
\usepackage{geometry}
\geometry{legalpaper, margin=1in}
\usepackage{amsmath} % or simply amstext
\newcommand{\angstrom}{\textup{\AA}}
\title{Predicting Transport in Lyotropic Liquid Crystal Membranes with Molecular Dynamics Simulations -- Outline}
\author{Benjamin J. Coscia \and Douglas L. Gin \and Richard D. Noble \and Michael R. Shirts} 
%MRS: we probably will want to add in Glaser's student (and possibly glaser) for their work on the X-ray simulations.
%MRS: also possibly Xunda if we end up using X-ray diffraction images of theirs that are not published.
\begin{document}
	\bibliographystyle{ieeetr}
	\graphicspath{{./figures/}}
	\maketitle
	\section{Introduction}
	Nanostructured membrane materials have become increasingly popular for aqueous separations applications because they offer the ability to control membrane architecture at the atomic scale allowing design of solute-specific separation membranes.
	\begin{itemize}
		\item Current state-of-the-art reverse osmosis membranes are dense and unstructured with tortuous and polydisperse pores which lead to inconsistent performance
		\item Tortuosity and polydispersity drive up energy requirements which strain developing regions and contribute strongly to CO2 emissions
		\item Designing RO membranes to achieve targeted separations of specific solutes is nearly impossible due to the separation (hypothesized to be) controlled by fluctuating polymer voids
		\item RO has difficulty separating neutral organics because they tend to dissolve in polymer matrix
		\item Many current RO membranes degrade in typical chlorine filled municipal water supplies (debating this point because there aren't any studies of LLC membrane fouling resistance)
                  %MRS: Doug should have information on the chemistry here and how resistant they might be.
		\item With nanostructured materials, solute rejecting pores can be tuned uniformly -- drives down energy requirements
		\item Targeted separations can be accomplished by tuning the molecular building blocks which form these materials
		\item Entirely different mechanisms govern the separation processes in various nanostructured materials which can inspire novel separation techniques
	\end{itemize}
	
	Development of nanostructured materials has been limited by the ability to synthesize and scale various fundamentally sound technologies.
	\begin{itemize}
		\item Leading technologies and their limitations:
		\begin{itemize}
			\item Graphene sheets -atomically thick which gives excellent permeability but defects during manufacturing severely impact selectivity
			\item Carbon Nanotubes - MD studies are promising but synthetic techniques unable to achieve necessary alignment and pore monodispersity
			%BJC: The following are ideas pulled from the grant. I need to do some reserach and come up with some one liners as to why each have not been successful
			\item Track-etch membranes
		    	\item Molecular squares
			\item Macrocyclic surfactants 
		\end{itemize}
	\end{itemize}
	  
	Self assembling lyotropic liquid crystals (LLCs) share the characteristic ability of nanostructured membrane materials to create highly ordered structures with the benefits of low cost and synthetic techniques feasible for large scale production.
	\begin{itemize}
		\item LLCs are versatile and controllable with a large chemical design space available for membrane design
		\item Synthetic techniques are cheap and amenable to creating any monomer in this large design space
		\item LLCs forms lamellar, bicontinuous cubic and hexagonal phases based on solution composition
		\item Na-GA3C11 has been described in literature as forming two types of self assembled phases - thermotropic (Colh) and lyotropic (HII)
		\item The thermotropic, Colh, is formed by the self assembly of neat monomer
		\item The lyotropic, HII phase is formed in the presence of small amounts of water
		\item Both assemble into cylinders with hydrophilic groups oriented inward towards the pore center and hydrophobic groups facing outward. The only difference is the inclusion of water in the structure which leads to minor variations in the structure with potentially different filtration properties (although no filtration experiments have been done on Colh)  
		\item Hydrophilic regions point towards pore centers
		\item Until recently, they could not be aligned - hindered progress
		\item Yale aligns them, then crosslinks them to lock in the structure - reference 2014 and 2016 papers. They say that they are scalable techniques
		\item LLC HII phase membranes offer potential for high permeability and selectivity which equals low energy consumption
		\item The Colh phase shares the same structural features with the HII phase with the exception of the presence of water. This paper will focus on the development of a model of the Colh phase since it is a simpler starting point and has just as much experimental data. The analysis used in this paper can be readily extended to the HII phase. 
	\end{itemize}
	
	A molecular level understanding of LLC membrane structure will elucidate small molecule transport mechanisms, providing guidelines to reduce the chemical space for design of monomers used to create separation-specific membranes.
	\begin{itemize}
                \item We do not yet understand how to reduce the effective pore size and/or tune the chemical environment in the HII nanopores for effective water desalination and small organic separations. Rejection studies show that this membrane can not do desalination yet
%MRS: ^^^nice way to turn a weakness of the materials class into an opportunity to improve them requring research! 
		\item Colh phase studies currently limited to one monomer
                \item Optimization efforts performed through trial and error over the past 20 years
                \item Macroscopic models are the only source of predictive modeling and existing theories do not adequately describe transport at these length scales
                \begin{itemize}
                        \item Do ions have trouble getting through because of interactions with other things in the pores (e.g. ions, carbonyl groups, benzene rings) -- related to ionic conduction
                          %MRS: vvv this point below should be made more specific.  Ionic gradient at the end?  Donnan exclusion?
                        \item Does concentration of ions in the pore repel incoming ions?
                        \item Do neutral solutes get rejected based solely on size rejection, or do interactions within the pore lead to selective rejection?
                         %MRS: vvv can you be a bit more specific about ``transport barriers?''
                        \item Is water structured inside the pores leading to transport barriers?
                          %MRS: vvv maybe put this higher, since the other bullet points above are dependent on this.  Also, this point conflacts understanding the structure, and then modifying it, which are sort of two steps.
			\item What does the microscopic pore structure look like and how can we relate that to monomer design and transport?
		\end{itemize}
		\item An atomistic understanding of the mechanism of solute transport can identify performance bottle necks and direct design of future monomers/membranes
		\item We can use molecular dynamics simulations to enhance our understanding 
	\end{itemize}
	 
	A clear picture of the nanoscopic structure of LLC membranes, gained by building a molecular model, 
%MRS: a molecular model can't really confirm or deny, because of the approximations.  provide evidence to support or call into question previous conclusions.
will confirm or deny past drawn conclusions that have largely guided our understanding of separation mechanisms. 
	\begin{itemize}
		\item The arrangement of sodium ions in the channels is thought to be confined to the pore walls. It is possible they are arranged more randomly
		\begin{itemize}
			\item This could change how one thinks about molecules diffusing through membrane
			\item Could also be a difference between lyotropic and thermotropic phases
		\end{itemize}
		\item The Colh phase is described as having pores made of disks or layers stacked on top of one another, each containing a set number of monomers. 
		\begin{itemize}
			\item How do the monomer head groups pack together? Do the benzene rings prefer to be stacked on top of each other or in another pi-stacking mode.
			\item Gas phase ab initio studies of benzene dimers have shown a clear energetic advantage for a parallel displaced or T-shaped conformation versus a stacked conformation. 
			\item Substituted benzene preferred stacking mode? Learning more about this   	
			\item A simple simulation study of a similar molecule (Head group is a sulfonate in the meta position) suggests that there are 4 monomers in each disk
			\item Calculations based on the volume of the liquid crystal suggest that there are seven monomers in each layer 
                          %MRS: vvv this point has a partial answer as well as a question -- do you want that here?
			\item Does the equilibrated structure exist in distinct layers? Fourier analysis can help answer this question
		\end{itemize}
		\item It is possible there is more than one metastable states associated with this LLC system
		\begin{itemize}
			\item Which phase is consistent with experiment?
			\item Can both phases be created experimentally?
			\item How will each state affect transport?
		\end{itemize}
	\end{itemize}
	
	We must show that the developed molecular model is consistent with physical observations so that we can trust conclusions drawn about structural features characteristic of the system.
	\begin{itemize}
          %MRS: ``careful'': better to show, not tell that it is careful.
		\item This paper will illustrate the careful development of a predictive molecular model and the steps taken to ensure it mimics the real system as best we can 
		\item To understand how physically realistic the model is, validation by comparison to experiment is necessary
		\item We are primarily interested in reproducing the conclusions about structure which have been made from XRD experiments and ionic conductivity measurements.    
                  %MRS: algorithm not really the main finding.
		\item We have written an algorithm to simulate the crosslinking mechanism and understand its influence on membrane structure
                  %MRS: the fact that we can is not that much interest, rather that we do.  Restate along lines of ``we havecompared simulated x-ray diffraction to experiment''
		\item We can simulate x ray diffraction patterns based on atomic coordinates in order to compare major features present in the 2D patterns
		\item We can predict ionic conductivity using two agreeing methods -- Collective diffusion and nernst einstein
	\end{itemize}
	
	\section{Methods}
	
	HII monomers were parameterized using the Generalized Amber Forcefield with the Antechamber package provided with AmberTools16. All molecular dynamics simulations were run using Gromacs version 5.1.2 and Gromacs version 2016.
	
	An ensemble of characteristic, low-energy vacuum monomer configurations were constructed by applying a simulated annealing process to a parameterized monomer.
	\begin{itemize}
		\item Structure cooled from 1000 to 50 K over 10 nanoseconds
		\item Result not global minimum but close enough for structure building
		\item Antechamber used for atomtyping with gaff forcefield
		\item Used Openeye Quacpac molcharge.py to assign charges %MRS: explain reason for 2 charge steps.
		\item Anneal again 
		\item Multiple configurations saved from annealing trajectory to prove independence of starting config
		\item Manual modifications to the structure were made to create specific geometries for xray diffraction experiments
	\end{itemize}
	
	The timescale for self assembly of monomers into the hexagonal phase is unknown and likely outside of a reasonable length for an atomistic simulation, calling for a more efficient way to build the system. 
	\begin{itemize}
		\item Work done shows coarse grain model self assembly in ~1000 ns , Citation: J. Phys. Chem. B 2013, 117, 4254-4262
		\item Attempts with Colh system not fruitful  
		\begin{itemize}
			\item Packed monomers into box
			\item Simulated for ~100 ns with no progress shown towards self assembly
		\end{itemize}  
		\item Wrote code to assemble monomers into Colh configuration close to what is expected 
		\item Equilibration simulations allow structure to relax into expected configuration 
	\end{itemize}
%MRS: justify choice of 20?	
	Each pore is made of twenty stacked monomer layers with periodic continuity in all directions, avoiding any edge effects and creating an infinite length pore ideal for studying transport (Fig.~\ref{fig:initial})
	\begin{itemize}
		\item A thinner system is better to reduce the computational cost and allow us to look at longer timescales
		\item Number of layers chosen to give sufficient resolution when simulating XRD patterns
	\end{itemize}

	\begin{figure}
          %MRS: this probably isn't so important it should be a figure in the main paper (maybe supporting).
	\includegraphics{placeholder.png}
		\caption{Monomers are placed in an initial configuration close to the expected equilibrium configuration and allowed to relax}
		\label{fig:initial}
	\end{figure}

	\noindent Initial guesses for the remaining structural parameters were chosen based on experimental data and treated as variables during model development 
	\begin{itemize}
		\item XRD gives Pore-to-Pore distances of $\approx$ 4.1 nm and indicates possible pi-stacking at $\approx$ 3.7 \angstrom (see figure~\ref{fig:xrd}) on - comparison of experimental vs. simulated)
		\item Pi-stacking exists in multiple stable configurations: sandwiched, T-shaped and parallel-displaced
		\item T-shaped and parallel-displaced are nearly isoenergetic and more stable than the sandwiched configuration.
		\item T-shaped configuration is most stable when benzene centers are $\approx$ 5 \angstrom apart which is not consistent with WAXS. % http://www.jbc.org/content/273/25/15458.full 
		\item System made with stacked and parallel-displaced benzene rings to see what is favored and matches XRD
		\item TEM images and rejection studies give a pore size estimate
	\end{itemize}

	An equilibration scheme with position restraints placed on benzene rings prevents unrealistic jumps during early equilibration steps.
	\begin{itemize}
		\item Equilibration scheme:
		\begin{itemize}
			\item Apply position restraints to monomer head groups during energy minimization 
			\item Leave position restraints on for nvt simulations to allow tails to intermingle (this also helps ensure independence of starting configuration)
			\item Gradually reduce force constants from 1000000 (by square root every 50 ps) until they are completely off
			\item Run long NPT simulations at 300 K and 1 bar ( $>$200 ns ) to fully equilibrate 
		\end{itemize}
	\end{itemize}
	
	Using an equilibrated structure, a crosslinking procedure was performed in order to better parallel synthetic procedures. 
	\begin{itemize}
		\item Crosslinking maintains alignment of cylindrical mesophases - emphasize that replicating the mechanism/kinetics is not important 
		\item head to tail addition dominates so I only implemented that
		\item racemic mixture - don't have to be too concerned about direction of attack 
		\item Details of crosslinking algorithm (refer to appendix or supplemental info but give a brief overview here)
	\end{itemize}  
	
	Simulated X-ray diffraction patterns were generated based on atomic coordinates to give a deeper understanding of the pore structure and spacing. 
        \begin{itemize}
               \item 3 dimensional fourier transformed electron density generates simulated 1D and 2D diffraction patterns
               \item The 1D patterns are generated by spherical integration of the FT
               \item 2D patterns are generated by taking cross sections of the FT in the qx, qy and qz planes
               \item We matched experiments based on iterative improvement of our choice in initial structure and equilibration procedure
        \end{itemize}

	% Ionic conductivity here probably
	\section{Results and Discussion}
	
	Varying number of monomers in each stacked layer between 4 and 8, as well as the benzene stacking configuration between sandwiched and parallel displaced, results in more than one stable structures.
        \begin{itemize}
                \item We did a systematic study of the properties resulting from each starting configuration
		\item In the real system, layers are stacked 3.7 \angstrom apart based on WAXS data. Pore spacing should be $\approx$ 4.1 nm
		\item We showed that we can rule out systems consisting of 4, 7, and 8 monomers based purely on membrane dimensions (Fig~\ref{table:p2p})  % I think even six is too big. Leaving it in consideration though since the disordered version gives pretty good p2p distances
		\item A characteristic of all systems simulated in this way, is a defined, cylindrical and open pore structure. Benzene rings arrange in a helical conformation after equil. Membrane about 8 nm thick (Fig ~\ref{fig:phaseA}) % BJC: Still trying to prove that layered is trying to go towards a helical conformation. Also, maybe include thickness in the table?
		\item This will be called phase A for simplicity  % BJC: I can come up with better phase names if that's important
%MRS: I though sandwitched not that stable?
		\item Sandwiched and parallel-displaced configurations are both stable for 100's of nanoseconds and give reasonable xray diffraction patterns.
		\item hypothesis: Parallel displaced is more stable and allows the benzene rings to stack closer together  
	\end{itemize} 

	%\begin{wraptable}{m}{5.5cm}
	\begin{table}
	\centering
	\begin{tabular}{|c|c|}
	\hline
	table & here \\
	\hline
	\end{tabular}
	\caption{Pore spacings inconsistent with experimentally measured data indicate that the respective configuration will%MRS: maybe not be the most stable? 
not be stable in a laboratory setting and should not be studied further~\label{table:p2p}}  
	\end{table}
	%\end{wraptable}

	%BJC: Simulations of systems built by stacking monomers in the parallel-displaced and stacked configurations results in two different structures that are both stable for $>$ 500 ns, suggesting the existence of more than one metastable state.
	Simulations of systems built with layers stacked 5 \angstrom apart results in a pore structure characterized by high radial disorder.
	\begin{itemize}
		\item This will be called phase B (Fig~\ref{fig:phaseB})
		\item The arrangement of sodium ions (which are closely bound to carbonyl head groups) can be fit to a gaussian distribution % BJC: supplemental info probably %MRS: anything can be fit to.  Do you mean well-approximated by?
		\item Like phase A, phase B can form at 280K. The only difference in simulations leading to this state, is the starting configuration
		\item The phase is also present when phase A is heated close to its isotropic transition point
		\item There are distinct differences in the membrane and pore structures between each state (Fig~\ref{fig:porestructures})
		\begin{itemize}
			\item Phase B has a closed pore, while phase A is open. This will impact transport mechanisms
			\item Phase B membranes are thicker
			\item Consequently, the pore spacing is smaller
		\end{itemize}
		\item We have at least two metastable states
	\end{itemize}

	\begin{figure}
	\centering
	\begin{subfigure}{.45\textwidth}
		\centering
		\includegraphics[width=\linewidth]{placeholder.png}
		\caption{~\label{fig:phaseA}}
	\end{subfigure}
	\begin{subfigure}{.45\textwidth}
		\includegraphics[width=\linewidth]{placeholder.png}
		\centering
		\caption{~\label{fig:phaseB}}
	\end{subfigure}
	\caption{Different pore structure are clear between systems built with benzene rings stacked (a) 3.7 \angstrom apart and (b) 5.0 \angstrom apart}
	\label{fig:porestructures}
	\end{figure}
%MRS: probably should point out things that both models get right, like the C-C isotropic band.
	Comparison of experimental 2D WAXS with simulated X-ray diffraction patterns produced from MD trajectories shows the most consistency with the parallel-displaced configuration
	\begin{itemize}
		\item Phase A
		\begin{itemize}	
			\item Purely stacked configuration is missing reflections at 2, 4, 8 and 10 0'clock, but they are there in the parallel displaced conformation
			\item There is evidence that the stacked configuration will shift towards an offset configuration over time. WAXS spots show up after short simulations, in which case there is visible movement away from a perfectly stacked, towards a parallel displaced conformation 
			\item Pi-stacking reflections are present in all cases but at spacings higher than shown experimentally. This is not surprising since GAFF will not properly treat pi-stacking.
%MRS: just wondering, are the offset configurations pi-stacking distances closer than the direct stacking ones?  Seem like might be.
		\end{itemize}
		\item Phase B
		\begin{itemize}
			\item Stacking refelctions present indicating separation of $\approx$ 5 \angstrom
			\item No spots at all which is consistent with the disorder shown by the rings in the pore
		\end{itemize}
	\end{itemize} 

	\begin{figure}
	\centering
	\begin{subfigure}{.3\textwidth}
		\centering
		\includegraphics[width=\linewidth]{placeholder.png}
		\caption{~\label{fig:xrdA}}
	\end{subfigure}
	\begin{subfigure}{.3\textwidth}
		\centering
		\includegraphics[width=\linewidth]{placeholder.png}
		\caption{~\label{fig:xrdexp}}
	\end{subfigure}
        \begin{subfigure}{.3\textwidth}
                \centering
                \includegraphics[width=\linewidth]{placeholder.png}
                \caption{~\label{fig:xrdB}}
        \end{subfigure}
	\caption{Phase A (a) provides a good match to experiment (b). Phase B (c) is missing reflections present in the experimental pattern}
	\label{fig:xrd}
	\end{figure}

	%BJC: generated 1D SAXS somewhere?? Just to show that we get the spacing right
        %MRS: reasonable estimates for which structure?
	The model gives reasonable estimates of ionic conductivity.
	\begin{itemize}
		\item There are a few ways to estimate ionic conductivity as seen in literature. We prefer a method which can extract an estimate based purely on an equilibrium trajectory
		\item We must also be sure that our analysis of results is consistent with the method use for experimental evaluation (i.e. AC impedance spectroscopy)
		\item We must also link our perfectly straight microscopic system to the not-so-straight macroscopic system. %BJC: worth getting into? See comment below.
		%BJC: We should discuss how we will actually make the link. We talked about it once but kind of left it hanging. It will likely be just some constant that we multiply our number by. To calculate the constant, I think using the azimuthal distribution is a good way to go about it. Although for that we'd need more data points telling us ionic conductivity as a function of azimuthal angle. That data will definitely be hard to come by in the near future so maybe a linear model is an okay estimate for now -- just to have something. For that, we know ionic conductivity with randomly oriented crystalline domains and we know ionic conductivity of a 'perfectly' aligned system (from our simulations). So we can make ionic conductivity as a linear function of alignment, I(x) (Actually it'd be piecewise with opposite slopes on either side of the an azimuthal angle of 90). Then we can integrate the azimuthal distribution weighted by I(x) to get a number to multiply our measurements by. 
		\item Two methods used to for prediction
		%BJC: probably belongs in methods:
		\item Nernst Einstein Relation:
		\begin{itemize}
			\item Widely used equation for estimating ionic conductivity
			\item Estimates DC ionic conductivity -- Frequency used during AC impedance slow enough to be approximated by dc at short enough timescales
			\item Relates the diffusive motion of ions in the membrane to the membrane's ionic conductivity
			\item Concentration is concentration of ions in the whole membrane, not just channels
		\end{itemize}
		\item Collective Diffusion:  %BJC: will leave explanation of how I did it for supplemental info (so most of the following). Will just reference paper
		\begin{itemize}
			\item Defines a collective coordinate, Q (charge), to quantify the amount of charge transfer through the system
			\item In the limit of infinite time, the MSD of Q can be used to formulate a diffusion coefficient of Q that can be related to ionic conductivity
			\item The model is valid for non-equilibrium and equilibrium simulations. Our analysis is based on the latter
			\item A similar model has been derived and validate to predict water permeability using equilibrium simulations
			\item The pore region is defined as the entire membrane system since lab IC measurements are done on bulk membrane rather than on individual pores. One would expect single channel IC to be much larger than the bulk membrane
		\end{itemize}                    
		\item See comparison of methods in Table~\ref{table:conductivity}
	\end{itemize}
	
	\begin{table}
	\centering
	\begin{tabular}{|c|c|}
	\hline
	Nernst Einstein & Collective Diffusion \\
	\hline
	\end{tabular}
	\caption{Calculated ionic conductivity using Nernst-Einsten and Collective Diffusion agree within error~\label{table:conductivity}}
	\end{table}

	The procedure used to create and validate our model can be used to evaluate other liquid crystalline assemblies. Using the design framework and analysis methods applied herin, we have the ability to reliably predict structures of new nanoporous membranes.

	%MRS: seems like needs to be a bit more discussion about 5 vs 6 vs 7, and which is most likely? 
        %MRS: will have to discuss a bit more about pi-stacking in GAFF and in reality, and how we might expect that will chage things.

	\section{Conclusion}
	
	In this work, we have suggested a more detailed picture of the structure of a self assembled thermotropic liquid crystal membrane using an atomistic molecular model.
	\begin{itemize}
		\item The model's physical properties are consistent with experimental measurements
		\item Channels are more disordered than previously thought and are filled with organic matter rather than hollow 
                  %MRS: vvvv this is true even for parallel displaced?
		\item There are likely no defined layers
		\item Results presented for Colh phase monomer but methods are readily adapted to other LCs
	\end{itemize}
	
	\section{Supplemental Information}
	
	Monomer configurations
	\begin{itemize}
		\item Pore-to-pore equilibration plots
                  %MRS: ^^^ probably should include how we determined if we ran long enough.
		\item 3D visualizations of different configurations tested
                  %MRS: vvv this might be in the main text, since it is what allows us to say what is more likely.
		\item Table of measurements on 5-8 monomer per layer configurations
		\item Sodium ion distribution in disordered phase
	\end{itemize}
	
	\noindent Crosslinking details
	\begin{itemize}
		\item Algorithm description. Link to full algorithm in git repository
		\item A figure showing the new crosslinks
	\end{itemize}
	
	\noindent Ionic conductivity % plus more details on implementation.
	\begin{itemize}
		\item MSD plots
	\end{itemize}
\end{document}
