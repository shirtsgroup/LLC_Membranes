\documentclass{article}
\usepackage{graphicx}
\usepackage{wrapfig}
\usepackage{subcaption}
\usepackage{geometry}
\geometry{legalpaper, margin=1in}
\usepackage{amsmath} % or simply amstext
\newcommand{\angstrom}{\textup{\AA}}
\title{Predicting Transport in Lyotropic Liquid Crystal Membranes with Molecular Dynamics Simulations -- Outline}
\author{Benjamin J. Coscia \and Douglas L. Gin \and Richard D. Noble \and Joseph Yelk \and Matthew Glaser \and Xunda Feng \and Michael R. Shirts} 
%MRS2: might need Osuji on this -- you can ask Xunda.  Check with people's published work on whether they use their middle initial. 
\begin{document}
	\bibliographystyle{ieeetr}
	\graphicspath{{./figures/}}
	\maketitle
	\section{Introduction}
	Nanostructured membrane materials have become increasingly popular for aqueous separations applications because they offer the ability to control membrane architecture at the atomic scale allowing design of solute-specific separation membranes.
	
	While RO anad NF have seen many advances in the past few decades, they are far from perfect separation technologies
	\begin{itemize}
		\item Current state-of-the-art reverse osmosis membranes are dense and unstructured with tortuous and polydisperse pores which lead to 
%MRS2: is inconsistent the right word?
inconsistent performance
		\item Tortuosity and polydispersity drive up energy requirements which 
%MRS2: are ``strain developing regions'' the right words?
strain developing regions and contribute strongly to CO2 emissions
		\item Designing RO membranes to achieve targeted separations of specific solutes is nearly impossible due to the separation (hypothesized to be) controlled by fluctuating polymer voids
		\item RO has difficulty separating neutral organics because they tend to dissolve in polymer matrix
		\item Many current RO membranes degrade in typical chlorine filled municipal water supplies (debating this point because there aren't any studies of LLC membrane fouling resistance)
                %MRS: Doug should have information on the chemistry here and how resistant they might be.
		%BJC: Doug said they didn't do any chemical stability testing since the membranes had flux problems in the first place. So I won't mention this.
                %MRS2: sounds good.  Or could simply say (somewhere) that a greater variety of chemistries are needed to handle diverse types of inflow streams.
		\item Nanofiltration was introduced as an intermediate between RO and ultrafiltration
		\item Larger and well-defined pores drive down energy requirements while still affording separation of solutes as small as ions, to some degree
		\item Like RO membranes, NF membranes have a pore size distribution which limits their ability to perform precise separations
	\end{itemize}	

	Nanostructured membranes can bypass many of the performance issues which plague traditional NF and RdO membranes.
	\begin{itemize}
		\item With nanostructured materials, solute rejecting pores can be tuned uniformly -- drives down energy requirements
		\item Targeted separations can be accomplished by tuning the molecular building blocks which form these materials
		\item Entirely different mechanisms govern the separation processes in various nanostructured materials which can inspire novel separation techniques
	\end{itemize}
	
	Development of nanostructured materials has been limited by the ability to synthesize and scale various fundamentally sound technologies.
	\begin{itemize}
		\item Leading technologies and their limitations:
		\begin{itemize}
			\item Graphene sheets -atomically thick which gives excellent permeability but defects during manufacturing severely impact selectivity
			\item Carbon Nanotubes - MD studies are promising but synthetic techniques unable to achieve necessary alignment and pore monodispersity
			\item Zeolites - sub-nm pores with good MD results but interstitial defects created during manufacturing kill selectivity 
		\end{itemize}
	\end{itemize}
	  
	Self assembling lyotropic liquid crystals (LLCs) share the characteristic ability of nanostructured membrane materials to create highly ordered structures with the benefits of low cost and synthetic techniques feasible for large scale production.
	\begin{itemize}
		\item LLCs are versatile and controllable with a large chemical design space available for membrane design
		\item Synthetic techniques are cheap and amenable to creating any monomer in this large design space
		\item LLCs forms lamellar, bicontinuous cubic and hexagonal phases based on solution composition
		\item Na-GA3C11 has been described in literature as forming two types of self assembled phases - thermotropic (Colh) and lyotropic (HII)
		\item The thermotropic, Colh, is formed by the self assembly of neat monomer
                  %MRS2: small change below.
		\item The lyotropic, HII phase is formed without (or in practice, with very small amounts of water)
		\item Both assemble into cylinders with hydrophilic groups oriented inward towards the pore center and hydrophobic groups facing outward. The only difference is the inclusion of water in the structure which leads to minor variations in the structure with potentially different filtration properties (although no filtration experiments have been done on Colh)  
		\item Hydrophilic regions point towards pore centers
                  %MRS2: Be more specific on ``hindered progress'' 
		\item Until recently, they could not be aligned - hindered progress
		\item Yale aligns them, then crosslinks them to lock in the structure - reference 2014 and 2016 papers. They say that they are scalable techniques
		\item LLC HII phase membranes offer potential for high permeability and selectivity which equals low energy consumption
		\item The Colh phase shares the same structural features with the HII phase with the exception of the presence of water. This paper will focus on the development of a model of the Colh phase since it is a simpler starting point and has just as much experimental data. The analysis used in this paper can be readily extended to the HII phase. 
	\end{itemize}
	
	A molecular level understanding of LLC membrane structure will elucidate small molecule transport mechanisms, providing guidelines to reduce the chemical space for design of monomers used to create separation-specific membranes.
	\begin{itemize}
                \item We do not yet understand how to reduce the effective pore size and/or tune the chemical environment in the HII nanopores for effective water desalination and small organic separations. Rejection studies show that this membrane can not do desalination yet
%MRS: ^^^nice way to turn a weakness of the materials class into an opportunity to improve them requring research! 
		\item Colh phase studies currently limited to one monomer
                \item Optimization efforts performed through trial and error over the past 20 years
                \item Macroscopic models are the only source of predictive modeling and existing theories do not adequately describe transport at these length scales
                \begin{itemize}
                        \item What does the microscopic pore structure look like?
			\item Do ions have trouble getting through because of interactions with other things in the pores (e.g. ions, carbonyl groups, benzene rings) -- related to ionic conduction
			\item Is rejection of ions due to Donnan exclusion?
                        \item Do neutral solutes get rejected based solely on size rejection, or do interactions within the pore lead to selective rejection?
                        \item Is water structured inside the pores, restricting low energy pathways for solutes to follow? 
%BJC: saw something like this in a paper 
%MRS2: good, cite if you can.
		\end{itemize}
		\item How can microscopic pore structure guide membrane design?
                  \begin{itemize}
		\item An atomistic understanding of the mechanism of solute transport can identify performance bottle necks and direct design of future monomers/membranes
		\item We can use molecular dynamics simulations to enhance our understanding 
                  \end{itemize}
	\end{itemize}
	 
	A clear picture of the nanoscopic structure of LLC membranes, gained by building 
%MRS2: suggest ``molecular models'', as one can argue that each way of doing it is a different model, or that it might require trying 2-3 things.
molecular models, will provide evidence to support or call into question past drawn conclusions that have largely guided our understanding of separation mechanisms. 
	\begin{itemize}
%MRS2: a thought - it would be great to add what is exactly meant by ``pore walls'' and how well defined they are is not clear. For example, how strict is the division between the hydrophobic tails, the head groups, and the ionic groups?  This directly affects what is meant by the ``size'' of the nanopore, and (to put somewhere else in the outline) what size exclusion means.
 		\item The arrangement of sodium ions in the channels is thought to be confined to the pore walls. It is possible they are arranged more randomly
		\begin{itemize}
			\item This could change how one thinks about molecules diffusing through membrane
			\item Could also be a difference between lyotropic and thermotropic phases
		\end{itemize}
		\item The Colh phase is described as having pores made of disks or layers stacked on top of one another, each containing a set number of monomers. 
		\begin{itemize}
			\item How do the monomer head groups pack together? Do the benzene rings prefer to be stacked on top of each other or in another pi-stacking mode.
			\item Gas phase ab initio studies of benzene dimers have shown a clear energetic advantage for a parallel displaced or T-shaped conformation versus a stacked conformation. 
			\item Substituted benzene rings exhibit an even stronger pi-stacking attraction %BJC: reference ab initio papers   	
			\item A simple simulation study of a similar molecule (Head group is a sulfonate in the meta position) suggests that there are 4 monomers in each disk
			\item Calculations based on the volume of the liquid crystal suggest that there are seven monomers in each layer % reference 2005-ish Zhou paper 
		\end{itemize}
		\item It is possible there is more than one metastable states associated with this LLC system
		\begin{itemize}
			\item Which phase is consistent with experiment?
			\item Can both phases be created experimentally?
			\item How will each state affect transport?
		\end{itemize}
	\end{itemize}
	
	We must show that the developed molecular model is consistent with physical observations so that we can trust conclusions drawn about structural features characteristic of the system.
	\begin{itemize}
		\item This paper will illustrate the development of a predictive molecular model and the steps taken to ensure it mimics the real system as best we can 
		\item To understand how physically realistic the model is, validation by comparison to experiment is necessary
		\item We are primarily interested in reproducing the conclusions about structure which have been made from XRD experiments and ionic conductivity measurements.    
		\item We have comparied simulated X-ray diffraction patterns to experiment in order to match major features present in the 2D patterns
		\item We can predict ionic conductivity using two agreeing methods -- Collective diffusion and nernst einstein
		\item We examined crosslinking mechanism and understand its influence on membrane structure
	\end{itemize}
	
	\section{Methods}
	
	HII monomers were parameterized using the Generalized Amber Forcefield with the Antechamber package provided with AmberTools16. All molecular dynamics simulations were run using Gromacs version 5.1.2 and Gromacs version 2016.
	
	An ensemble of characteristic, low-energy vacuum monomer configurations were constructed by applying a simulated annealing process to a parameterized monomer.
	\begin{itemize}
		\item Structure cooled from 1000 to 50 K over 10 nanoseconds
		\item Result not global minimum but close enough for structure building
		\item Antechamber used for atomtyping with gaff forcefield
		\item Used Openeye Quacpac molcharge.py to assign charges
		\item The am1bccsym method provided with Quacpac performs a conformational search and applies charges symmetrically based on the lowest energy subset of possible conformations
		\item Anneal again 
		\item Multiple configurations saved from annealing trajectory to prove independence of starting config. A representative structure is given in Figure ~\ref{fig:monatomistic}
		\item Manual modifications to the structure were made to create specific geometries for X-ray diffraction experiments
	\end{itemize}
	
	%BJC: I'm not sure where to place this figure. ACS nano requires the methods section to go last (after conclusions). It seems like this should go earlier, but it fits the context best around here.
	%BJC2: I think I might make this a six panel figure. The other two panels will show a cartoon and atomistic pore view.
	\begin{figure}
	\centering
	\begin{subfigure}{.45\linewidth}
		\centering
		\includegraphics[width=\linewidth]{placeholder.png}
		\caption{}\label{fig:chemdraw}
	\end{subfigure}
	\begin{subfigure}{.45\linewidth}
		\centering
		\includegraphics[width=\linewidth]{placeholder.png}
		\caption{}\label{fig:monatomistic}
	\end{subfigure}
	\medskip
	\begin{subfigure}{.45\linewidth}
		\centering
		\includegraphics[width=\linewidth]{placeholder.png}
		\caption{}\label{fig:cartoonlayer}
	\end{subfigure}
	\begin{subfigure}{.45\linewidth}
		\centering
		\includegraphics[width=\linewidth]{placeholder.png}
		\caption{}\label{fig:layeratomistic}
	\end{subfigure}
	\caption{(a) The chemical structure of NAGA3C11 (b) The chemical structure visualized atomistically enhances intuition how monomers might pack together %MRS2: in what way?
(c) Wedge-shaped monomers assemble into layers (d) An atomistic view of a layer suggests a reasonable range of monomers per layer to test\label{fig:cartoonvatomistic}  
	\end{figure}	
	
	The timescale for self assembly of monomers into the hexagonal phase is unknown and likely outside of a reasonable length for an atomistic simulation, calling for a more efficient way to build the system. 
	\begin{itemize}
          %MRS2: specify what type of self assembly -- could be that layers are faster than hexagonal pores to get to.
	  %BJC2: Don't I suggest the type of self assembly to be assembly into the hexagonal phase? Or should I be more broad as in, we see no type of self-assembly occuring.
		\item Work done shows coarse grain model self assembly in ~1000 ns , Citation: J. Phys. Chem. B 2013, 117, 4254-4262
		\item Attempts with Colh system not fruitful  
		\begin{itemize}
			\item Packed monomers into box with Packmol
			\item Simulated for ~100 ns with no progress shown towards self assembly
		\end{itemize}  
		\item Wrote code to assemble monomers into Colh configuration close to what is expected 
		\item Equilibration simulations allow structure to relax into expected configuration 
	\end{itemize}
	
%MRS2: need to be careful describing which stacking/layering is described where/when based on experiments actually reported.
%BJC2: can you clarify this point?
	Each pore is made of twenty stacked monomer layers with periodic continuity in all directions, avoiding any edge effects and creating an infinite length pore ideal for studying transport (Fig.~\ref{fig:initial})
	\begin{itemize}
		\item A thinner system is better to reduce the computational cost and allow us to look at longer timescales
		\item Number of layers chosen to give sufficient resolution when simulating XRD patterns
                \item Decided on 20 layers using a system which held a vacuum gap above and below the membrane
                \item Too few layers leads to micelle-like formations %MRS: with gap.
	\end{itemize}

	\noindent Initial guesses for the remaining structural parameters were chosen based on experimental data and varied in order to develop a model that agrees with experiment. 
%MRS2: ``treated as variables'' is vague. Be specific about what it means you are doing.
	\begin{itemize}
		\item XRD gives pore-to-pore distances of $\approx$ 4.1 nm and indicates possible pi-stacking at $\approx$ 3.7 \angstrom (see figure~\ref{fig:xrd}) on - comparison of experimental vs. simulated)
		\item Pi-stacking exists in multiple stable configurations: sandwiched, T-shaped and parallel-displaced
		\item T-shaped and parallel-displaced are nearly isoenergetic and more stable than the sandwiched configuration.
		\item T-shaped configuration is most stable when benzene centers are $\approx$ 5 \angstrom apart which is not consistent with WAXS. % http://www.jbc.org/content/273/25/15458.full 
		\item System made with stacked and parallel-displaced benzene rings to see what is favored and matches XRD
		\item TEM images and rejection studies give a pore size estimate
	\end{itemize}

	An equilibration scheme with position restraints placed on benzene rings prevents unrealistic jumps during early equilibration steps.
	\begin{itemize}
		\item Equilibration scheme:
		\begin{itemize}
			\item Apply position restraints to monomer head groups during energy minimization 
			\item Leave position restraints on for nvt simulations to allow tails to intermingle (this also helps ensure independence of starting monomer configuration)
			\item Gradually reduce force constants from 1000000 (by square root every 50 ps) until they are completely off
			\item Run long NPT simulations at 300 K and 1 bar ( $>$200 ns ) to fully equilibrate 
		\end{itemize}
	\end{itemize}
	
	Using an equilibrated structure, a crosslinking procedure was performed in order to better parallel synthetic procedures. 
	\begin{itemize}
		\item Crosslinking maintains alignment of cylindrical mesophases - emphasize that replicating the mechanism/kinetics is not important 
		\item head to tail addition dominates so I only implemented that
		\item racemic mixture - don't have to be too concerned about direction of attack 
		\item Details of crosslinking algorithm (refer to appendix or supplemental info but give a brief overview here)
	\end{itemize}  
	
	Simulated X-ray diffraction patterns were generated based on atomic coordinates to give a deeper understanding of the pore structure and spacing. 
        \begin{itemize}
               \item 3 dimensional Fourier transformed electron density generates simulated 1D and 2D diffraction patterns
               \item The 1D patterns are generated by spherical integration of the FT
               \item 2D patterns are generated by taking cross sections of the FT in the qx, qy and qz planes
               \item We matched experiment by varying the initial structure. The selection of initial structure is explained later
        \end{itemize}

	The Nernst Einstein relation and the Collective Diffusion model were both applied to the system in order to estimate ionic conductivity.
	\begin{itemize}
		\item There are a few ways to estimate ionic conductivity as seen in literature. We prefer a method which can extract an estimate based purely on an equilibrium trajectory (reference to computational electrophysiology) 
                \item We must also be sure that our analysis of results is consistent with the method used for experimental evaluation (i.e. AC impedance spectroscopy)
                \item We must also link our perfectly straight microscopic system to the not-so-straight macroscopic system. %BJC: worth getting into? Could also just say 
		% that we expect our calculated value to be higher since ours is perfectly aligned relative to the real system.
                \item Two methods used to for prediction
                \item Nernst Einstein Relation:
                \begin{itemize}
                        \item Widely used equation for estimating ionic conductivity
                        \item Estimates DC ionic conductivity -- Frequency used during AC impedance slow enough to be approximated by dc at short enough timescales
                        \item Relates the diffusive motion of ions in the membrane to the membrane's ionic conductivity
                        \item Concentration is concentration of ions in the whole membrane, not just channels
                \end{itemize}
                \item Collective Diffusion:  %BJC: will have a very topical explanation followed by a reference to the paper from which I got the method
                \begin{itemize}
                        \item Defines a collective coordinate, Q (charge), to quantify the amount of charge transfer through the system
                        \item In the limit of infinite time, the MSD of Q can be used to formulate a diffusion coefficient of Q that can be related to ionic conductivity
                        \item The model is valid for non-equilibrium and equilibrium simulations. Our analysis is based on the latter
                        \item A similar model has been derived and validate to predict water permeability using equilibrium simulations
                        \item The pore region is defined as the entire membrane system since lab IC measurements are done on bulk membrane rather than on individual pores. One would expect single channel IC to be much larger than the bulk membrane %MRS: per area, since extensively, bulk would be be bigger (multiple pores)
                \end{itemize}
        \end{itemize}

	\section{Results and Discussion}

	In order to construct an initial configuration which gives reliable trends, we need to understand the composition of layers, how far apart to stack the layers, and how to orient them with respect to each other.
		
	To understand the composition of the monomer layers, we ran simulations created with 4 - 8 monomers per layer 
	\begin{itemize}
		\item All configurations were stable for at least a short time %BJC: I am going to run out these simulations further and will have more to say
                \item We showed that we can rule out systems consisting of 4, 7, and 8 monomers based purely on membrane dimensions (Table~\ref{table:p2p})
        \end{itemize}
        
	The initial distance between layers is a key determining factor of the equilibrium configuration.
	\begin{itemize}
                \item In the real system, layers are stacked 3.7 \angstrom apart based on WAXS data. 
                \item A characteristic of all systems simulated in this way, is a defined, cylindrical and open pore structure. Benzene rings arrange in a helical conformation after equil. Membrane about 8 nm thick (Fig ~\ref{fig:phaseA}) % BJC: Still trying to prove that layered is trying to go towards a helical conformation. Also, maybe include thickness in the table?
                \item This will be called phase A for simplicity  
	        \item Simulations of systems built with layers stacked 5 \angstrom apart results in a pore structure characterized by high radial disorder, while maintaining partitioning between hydrophobic and hydrophilic regions.
        	\begin{itemize}
                	\item This will be called phase B (Fig~\ref{fig:phaseB})
                	\item The arrangement of sodium ions (which are closely bound to carbonyl head groups) can be well-approximated by a gaussian distribution 
                          %MRS2: below: ``Only difference''.  Only difference is in how they are prepared.  But presumably, there are a number of initial configurations that can lead to this one, so emphasizing that it's the only difference isn't that important.  Also, we want to be emphasizing the thermodynamic differences, not kinetics.
                	\item Like phase A, phase B can form at 280K. %BJC2: ^^ so get rid of this part : The only difference in simulations leading to this state, is the initial interlayer spacing
                	\item The phase is also present when phase A is heated close to its isotropic transition point
                	\item There are distinct differences in the membrane and pore structures between each state (Fig~\ref{fig:porestructures})
                	\begin{itemize}
                        	\item Phase B has a closed pore, while phase A is open. This will impact transport mechanisms
                        	\item Phase B membranes are thicker
                        	\item Consequently, the pore spacing is smaller
                	\end{itemize}
		\end{itemize}
                \item We have at least two metastable states
        \end{itemize}

	We varied the relative interlayer orientation between sandwiched and parallel-displaced based on our knowledge of the stability of these two pi-stacking modes.
        \begin{itemize}
		\item Simulated Xray diffraction patterns generated from initial configurations of each pi-stacking mode, shows a 
%MRS2: ``clear relation'' is a bit indirect. 
%BJC2: clear link? 
clear link between the parallel displaced configuration and major features present in the XRD patterns %BJC: A figure would probably be good here %MRS2: yep, refer a figure here.
		\item Diagonal spots in the alkane chain region are only generated for structures created in the parallel displaced configuration
		\item However, both configurations are stable for 100's of nanoseconds
		\item Xray diffraction of the sandwiched configuration after equilibration shows the spots. This indicates that there is a shift towards the offset configuration.
	\end{itemize} 

	The stability of crystalline phases is strongly dependent on temperature, requiring evaluation of our force field's ability to represent the temperature at which we are simulating.
	\begin{itemize}
		\item All of the above simulations were ran at 280K and 300K
		\item Phase A is very stable at 280K
		\item when bumped to 300K it is also stable, however with a larger degree of disorder %BJC: can quantify this with order parameter
		\item We see a clear transition to Phase B for any starting configuration when the temperature is raised above 310K (~\ref{fig:transition})
		\item Simulating the system at room temperature should be sufficient to see both phases
                  %MRS2: what does ``sufficient'' mean here?  Sufficient for what purpose?
	\end{itemize}

        %MRS2: Tables should also have specific theses as well, so think about that.  Merely saying ``here is the information'' isn't as useful without a guide through it.
	A summary of all experiments and relevant structural parameters are presented in Table ~\ref{table:p2p}

	\begin{figure}
	\centering
	\begin{subfigure}
		\centering
		\includegraphics[width=\linewidth]{placeholder.png}
		\caption{}\label{fig:orderparameter}
	\end{subfigure}
	\begin{subfigure}
		\centering
		\includegraphics[width=\linewidth]{placeholder.png}
		\caption{}\label{fig:order}
	\end{subfigure}
	\begin{subfigure}
		\centering
		\includegraphics[width=\linewidth]{placeholder.png}
		\caption{}\label{fig:disorder}
	\end{subfigure}
	\caption{(a) As temperature is increased, Phase A transitions to Phase B, becoming more disordered. An order parameter of 1 indicates perfect ordering, while 0 indicates disorder. (b) The system is ordered, characteristic of Phase A, initially (c) The system becomes disordered, characteristic of Phase B, as temperature is raised beyond 310K}\label{fig:transition}
%MRS2: maybe provide a bit more information what data will be here?  Order parametr vs time? One panel with order paramter vs. time, one with configurations from 0 and 1 endpoints?
	\end{figure}	

	\begin{table}
	\centering
	\begin{tabular}{|c|c|}
	\hline
	table & here \\
	\hline
	\end{tabular}
	\caption{The pore spacing of the model increases as number of monomers in each layer increases. The pore spacing of a system starting in the sandwiched configuration is systematically lower than that started in an offset configuration. Systems built with 5 monomers per layer in a parallel displaced configuration results in a pore spacing closest to experimental data}\label{table:p2p} 
	\end{table}
	
	\begin{figure}
	\centering
	\begin{subfigure}{.45\textwidth}
		\centering
		\includegraphics[width=\linewidth]{placeholder.png}
		\caption{~\label{fig:phaseA}}
	\end{subfigure}
	\begin{subfigure}{.45\textwidth}
		\includegraphics[width=\linewidth]{placeholder.png}
		\centering
		\caption{~\label{fig:phaseB}}
	\end{subfigure}
	\caption{There are clear differences in pore structure between systems built with layers stacked (a) 3.7 \angstrom apart and (b) 5.0 \angstrom apart}
        %MRS2: explain theses?  Also, the exact configurations used to start them are probably not that important; many different initial configurations will probably lead to most of them.  It's the thermodyamics that are important.
	%BJC: okay, so just rephrase it, worded in a way that doesn't reference the initial configuration/layer spacing?
	%BJC: what do you  mean by 'thermodynamics that are important' -- just distinguishing which state is more favorable under what conditions?
	\label{fig:porestructures}
	\end{figure}

	%BJC: I might eliminate the other paragraph (2 before this one), where I justify parallel-displaced vs. sandwiched, and do it all in this paragraph.	
        %MRS2: a little hard to say until the data is in.
	Full comparison of experimental 2D WAXS with simulated X-ray diffraction patterns produced from MD trajectories shows the most consistency with the parallel-displaced configuration made up of 5 monomers per layer.
	\begin{itemize}
		\item Phase A
		\begin{itemize}	
			\item Purely stacked configuration is missing reflections at 2, 4, 8 and 10 0'clock, but they are there in the parallel displaced conformation
			\item There is evidence that the stacked configuration will shift towards an offset configuration over time. WAXS spots show up after short simulations, in which case there is visible movement away from a perfectly stacked, towards a parallel displaced conformation 
			\item Pi-stacking reflections are present in all cases but at spacings higher than shown experimentally. This is not surprising since GAFF will not properly treat pi-stacking.
			\item Interestingly, the spots, which are usually associated with alkyl chain tilt, still appear with an average tilt angle of 0 degrees
		\end{itemize}
		\item Phase B
		\begin{itemize}
			\item Stacking refelctions present indicating separation of $\approx$ 5 \angstrom
			\item No spots at all which is consistent with the disorder shown by the rings in the pore
		\end{itemize}
		\item Both phases show a ring in the simulated 2D WAXS patterns at q $\approx$ 1.4 \angstrom\textsuperscript{-1} which is typical of packed alkane chains
		\item Both phases exhibit axial reflections. Even though the rings are not ordered in phase B, 
%MRS2: not quite clear what physical structures the axial reflections correspond to (what are the distances? What do they correspond to?)
they are still partitioned into layers which gives rise to an axial refelction at the same distance where we expect alkane chain packing to be present   
%MRS2: how are they layers defined here? Can you bring in more information about the nonuniformity of the density?
%BJC2: Are you thinking about the fourier transforms? I could bring that in
	\end{itemize} 

	\begin{figure}
	\centering
	\begin{subfigure}{.3\textwidth}
		\centering
		\includegraphics[width=\linewidth]{placeholder.png}
		\caption{~\label{fig:xrdA}}
	\end{subfigure}
	\begin{subfigure}{.3\textwidth}
		\centering
		\includegraphics[width=\linewidth]{placeholder.png}
		\caption{~\label{fig:xrdexp}}
	\end{subfigure}
        \begin{subfigure}{.3\textwidth}
                \centering
                \includegraphics[width=\linewidth]{placeholder.png}
                \caption{~\label{fig:xrdB}}
        \end{subfigure}
	\caption{Phase A (a) provides a good match to experiment (b). Phase B (c) is missing reflections present in the experimental pattern} % BJC: any more detail?  %MRS2: yes, give distances, and describe what you mean as ``good match''
%BJC2: what is a clear way to describe the spots in the ring. Diagonal reflections? radial spots?
	\label{fig:xrd}
	\end{figure}

	The model gives reasonable estimates of ionic conductivity for both phases.
	\begin{itemize}
		\item See comparison of methods in Table~\ref{table:conductivity}
		\item The methods agree reasonably well within error
		\item In the future, we will probably only use Nernst Einstein because it requires less simulation to get good statistics. %BJC: probably don't need to mention this %MRS2: can mention it peforms better, informing future work.
		\item Our calculated values are higher than experimental values, as expected.
		\item The real system, although mostly aligned and straight, still has a distribution of azimuthal angles, lowering the effective ionic conductivity of the bulk membrane. 
		\item The ordering from isotropic to mostly aligned mesophases showed an 85x increase in ionic conductivity. We would expect more gains in a perfect system.
		\item The ordered phase has a higher calculated ionic conductivity than the disordered phase.
	\end{itemize}
	
	\begin{table}
	\centering
	\begin{tabular}{|c|c|}
	\hline
	Nernst Einstein & Collective Diffusion \\
	\hline
	\end{tabular}
	\caption{Calculated ionic conductivity using Nernst-Einsten and Collective Diffusion agree within error. Both methods give calculated values of ionic conductivity which are higher, but reasonably close to experimental values~\label{table:conductivity}}
%MRS2: I would say that they are reasonable close to experiment, though higher. Also somewhat contradicted in other places.
	\end{table}

	The procedure used to create and validate our model can be used to evaluate other liquid crystalline assemblies. Using the design framework and analysis methods applied herin, we have the ability to reliably predict structures of new nanoporous membranes.

	\section{Conclusion}
	
	In this work, we have suggested a more detailed picture of the structure of a self assembled thermotropic liquid crystal membrane using an atomistic molecular model.
	\begin{itemize}
		\item The model's physical properties are consistent with experimental measurements
                  %MRS2: are these two disordered and offset?  If so, then one is stable (Though we are not sure at 300K which is which).  Or are you referring to three phases?
		\item We have discovered the existence of two metastable configurations that persist at room temperature
		\item This methodology has been developed for a specific case but can be readily adapted to other LLCs
	\end{itemize}
	%MRS: focus around what we learned about the system - not that we have a model

	\section{Supplemental Information} %MRS2: check, might be called supporting information, depending on the journal. %BJC2: Also, it's a separate document in ACS nano at least
	
	Monomer configurations
	\begin{itemize}
		\item Pore-to-pore equilibration plots
		\item Plots of other things that level off indicating equilibration
		\item 3D visualizations of different configurations tested
		\item Sodium ion distribution in disordered phase
	\end{itemize}
	
        \begin{figure}
		\centering
		\includegraphics{placeholder.png}
                \caption{Monomers are placed in an initial configuration close to the expected equilibrium configuration and allowed to relax}
                %MRS2: what is the thesis?
                \label{fig:initial}
        \end{figure}

	Equilibration procedure

	\noindent Crosslinking details
	\begin{itemize}
		\item Algorithm description. Link to full algorithm in git repository
		\item A figure showing the new crosslinks
	\end{itemize}
	
	\noindent Ionic conductivity % plus more details on implementation.
	\begin{itemize}
		\item MSD plots
	\end{itemize}
\end{document}
