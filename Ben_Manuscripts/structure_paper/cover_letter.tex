\documentclass[fontsize=11pt]{article}

\usepackage[margin=1in]{geometry}
\usepackage{graphicx}
\usepackage{wrapfig}
\usepackage{subcaption}
\usepackage{amsmath} 
\usepackage{siunitx}
\usepackage{booktabs}
\usepackage{gensymb}
\usepackage{hyperref}
\usepackage[export]{adjustbox}
\usepackage{anyfontsize}

\begin{document}

	\graphicspath{{./figures/}}

	%BJC: check address, phone number and fax (?) number
	\begin{figure}
	\centering
	\begin{minipage}{0.37\textwidth}
	\includegraphics[width=2.14in,left]{CUBoulder.pdf}
	\end{minipage}
	\begin{minipage}{0.62\textwidth}
	\scriptsize
	\noindent Department of Chemical and Biological Engineering \hfill t 303-735-7860~~~~~~~~~~~~~~~~~~ \\
	\noindent 596 UCB \hfill f 303-735-7860~~~~~~~~~~~~~~~~~~ \\
	\noindent Boulder, Colorado 80309 \hfill michael.shirts@colorado.edu \\
	\end{minipage}
	\end{figure}
	
	\noindent October 1st, 2018\\

	\noindent Dear Editors,\\
	
	We are submitting ``Understanding the Nanoscale Structure of Inverted Hexagonal 
	Phase Lyotropic Liquid Crystal Polymer Membranes'' by Benjamin J. Coscia, Joseph
	Yelk, Matthew A. Glaser, Douglas L. Gin, Xunda Feng, and Michael R. Shirts, for
	consideration for publication as an Article in the Journal of Physical Chemistry
	B. We expect this work to be of significant interest to researchers who study 
	nanostructured polymer membranes for aqueous separations.
	
	This article examines the atomistic structure of an inverted hexagonal phase 
	self-assembled lyotropic liquid crystal (LLC) polymer membrane. To create these
	materials, liquid crystal (LC) monomers self-assemble into ordered, uniform-sized
	and hexagonally-packed nanopores. The nm-sized pore centers are charged and hydrophilic 
	in nature which makes this type of membrane useful for highly selective aqueous 
	separations of small solutes. By tuning the shape, size and chemical functionality
	of LC monomers, one could intelligently design the pore environment. However, our 
	current understanding of the nanoscopic pore structure is not yet adequate for us to
	be able to controllably design membranes for solute-specific separations.
	
	In this paper, we used molecular dynamics (MD) simulations to model an
	experimentally characterized LLC polymer membrane and learn about its detailed
	atomistic structure. We simulated XRD patterns from the MD simulation trajectories
	for comparison to experimental 2D wide-angle X-ray scattering (WAXS) and 1D 
	small-angle X-ray scattering patterns. We found that 5 columns of stacked LLC 
	monomers are likely to pack around the hydrophilic core of each pore. These 
	columns likely move independently of each other	over time scales that we can not
	simulate. Some of the structure previously attributed to monomer tail tilt is 
	likely instead due to ordered tail packing. Although the system studied	has been
	reported as ``dry'', small amounts of water appear necessary to fully reproduce 
	all features from the experimental 2D-WAXS pattern due to asymmetries introduced 
	by hydrogen bonds between the LLC monomer head groups and water molecules. 
	Finally, we explored the composition and structure of the nanopores, showing there is 
	composition gradient rather than an abrupt partition between the hydrophilic and
	hydrophobic regions. The clearer picture of the nanoscopic structure
	of these membranes provided in this study will enable a better understanding of the
	mechanisms of small molecule transport within these nanopores. \\
	
	%BJC: included phone numbers because it asked to do so in author's guide.
	\noindent Some suggestions for reviewers are:
	\begin{enumerate}

		\item Menachem Elimelech is a leader in the development of membrane technologies
		for	water desalination and water reuse (Yale University, 203-432-2789,
		\href{mailto:menachem.elimelech@yale.edu}{menachem.elimelech@yale.edu}).
	
		\item Francisco Hung has used molecular simulations to study nematic liquid crystals
		and various	nanostructured materials (Northeastern Univerity, 617-373-8619 	
		\href{mailto:f.hung@northeastern.edu}{f.hung@northeastern.edu})
		
		\item Eric Jankowski has experience modeling soft materials using molecular dynamics
		including work done simulating structure factors from simulations, much like
		the techniques we use in our work (Boise State University, 208-426-5681,
		\href{mailto:ericjankowski@boisestate.edu}{ericjankowski@boisestate.edu})
		
	\end{enumerate}
	
	\noindent Please send correspondence regarding this paper to Michael R. Shirts (contact
	details in letterhead).\\	
	
	\noindent Sincerely,
	
	\noindent Michael R. Shirts \\
    \noindent Benjamin J. Coscia \\
    \noindent Joseph Yelk \\
    \noindent Matthew A. Glaser
    \noindent Douglas L. Gin \\
    \noindent Xunda Feng \\
	
\end{document}

% LocalWords:  BJC
