\documentclass[fontsize=11pt]{article}

\usepackage[margin=1in]{geometry}
\usepackage{graphicx}
\usepackage{wrapfig}
\usepackage{subcaption}
\usepackage{amsmath} 
\usepackage{siunitx}
\usepackage{booktabs}
\usepackage{gensymb}
\usepackage{hyperref}
\usepackage[export]{adjustbox}
\usepackage{anyfontsize}

\begin{document}

	\graphicspath{{./figures/}}

	%BJC: check address, phone number and fax (?) number
	\begin{figure}
	\centering
	\begin{minipage}{0.37\textwidth}
	\includegraphics[width=2.14in,left]{CUBoulder.pdf}
	\end{minipage}
	\begin{minipage}{0.62\textwidth}
	\scriptsize
	\noindent Department of Chemical and Biological Engineering \hfill t 303-735-7860~~~~~~~~~~~~~~~~~~ \\
	\noindent 596 UCB \hfill f 303-735-7860~~~~~~~~~~~~~~~~~~ \\
	\noindent Boulder, Colorado 80309 \hfill michael.shirts@colorado.edu \\
	\end{minipage}
	\end{figure}
	
	\noindent October 1st, 2018\\

	%MRS: I shortened this a bit and made it more direct; was a bit long.
	\noindent Dear Editors,\\
	%MRS: don't need to put they are graduate students or postdics. 
	We are submitting ``Understanding the Nanoscale Structure of Inverted Hexagonal 
	Phase Lyotropic Liquid Crystal Polymer Membranes'' by Benjamin J. Coscia, Joseph Yelk, Matthew A. Glaser, Douglas
	L. Gin, Xunda Feng, and Michael R. Shirts, for
	consideration for publication as an Article in the Journal of Physical Chemistry
	B. We expect this work to be of significant interest to researchers who study 
	nanostructured polymer membranes for aqueous separations.
	
        %MRS: maybe ``examines'' instead of studies. 
	This article examines the atomistic structure of an inverted hexagonal phase 
	self-assembled lyotropic liquid crystal (LLC) polymer membrane. To create these
	materials, liquid crystal (LC) monomers self-assemble into ordered, uniform-sized
	and hexagonally-packed nanopores. The nm-sized pore centers are charged and hydrophilic 
	in nature which makes this type of membrane useful for highly selective aqueous 
	separations of small solutes. By tuning the shape, size and chemical functionality
	of LC monomers, one could 
        %MRS: suggest can->could.
intelligently design the pore environment. However, our 
	current understanding of the nanoscopic pore structure is not yet adequate for us to
	be able to controllably design membranes for solute-specific separations.
	
	%BJC: following is quite similar to the abstract
	In this paper, we used molecular dynamics (MD) simulations to model an
	experimentally characterized LLC polymer membrane and learn about its detailed atomistic
	structure. We simulated XRD patterns from the MD simulation trajectories
	for comparison to experimental 2D wide-angle X-ray scattering (WAXS) and 1D 
	small-angle X-ray scattering patterns. We found that 5 columns of stacked LLC monomers are 
	likely to pack around the hydrophilic core of each pore. These 
	columns likely move independently of each other	over time scales that we can not
	simulate. Some of the structure previously attributed to monomer
	tail tilt is likely instead due to ordered tail packing. Although the system studied
	has been reported as ``dry'', small amounts of water appear necessary to 
	fully reproduce all features from the experimental 2D-WAXS pattern due to asymmetries
	introduced by hydrogen bonds between the LLC monomer head groups and water molecules.
	Finally, we explored the composition and structure of the nanopores, showing there is 
        composition gradient rather than an abrupt partition between the 
	hydrophilic and hydrophobic regions. The clearer picture of the nanoscopic structure
	of these membranes provided in this study will enable a better understanding of the
	mechanisms of small molecule transport within these nanopores. \\
	
	%MRS: don't need phone numbers.
	\noindent Some suggestions for reviewers are:
	\begin{enumerate}
		\item Chinedum Osuji has published work related to the synthesis and physical
		properties of the same inverted hexagonal phase LLC membrane studied in our 
		work (University of Pennsylvania, 
		\href{mailto:cosuji@seas.upenn.edu}{cosuji@seas.upenn.edu}).
                %MRS: osuji might be a conflict of interest due to supervising Xunda recently plus joint grants with Gin, but I don't know for sure. Can leave him on for now.
		\item Menachem Elimelech is a leader in the development of membrane technologies
		for	water desalination and water reuse (Yale University, 
		\href{mailto:menachem.elimelech@yale.edu}{menachem.elimelech@yale.edu}).
		\item Arun Yethiraj has studied molecular simulations of the self-assembly
		of liquid crystals into ordered nanostructures, including the type studied
		in our work. (University of Wisconsin Madison, 
		\href{mailto:yethiraj@chem.wisc.edu}{yethiraj@chem.wisc.edu})  % questionable due to related self-assembly research. MRS: I looked over his recent papers, and I think will be OK.
	\end{enumerate}
	
	%BJC: Other reviewer ideas:
	% Jason Bara - University of Alabama -- former member of Gin/Noble group who worked on
	% bicontinuous cubic phase at one point. He helped me figure out how to print the 3D models
	% Jeff McCutcheon - My undergrad advisor. Advised by Elimelech for his PhD.
        % conflicts of interest for the first two with the authors.
	% Benoit Roux? - No, too busy and critical.
	% Eric Jankowski - Boise State, met at FOMMS. One of his students works with simulated XRD patterns. I actually think this is a much better idea than Yethiraj. Good choice, can put as well as Yethiraj.
	
	\noindent Please send correspondence regarding this paper to Michael R. Shirts (contact
	details in letterhead).\\	
	
	\noindent Sincerely,
	
	\noindent Michael R. Shirts \\
    \noindent Benjamin J. Coscia \\
    %BJC: so many authors, it runs off the page. Can just say "Sincerely, the authors" or list names on single line separated by commas. 
    %Now there is space.
    \noindent Joseph Yelk \\
    \noindent Matthew A. Glaser
    \noindent Douglas L. Gin \\
    \noindent Xunda Feng \\
	
\end{document}

% LocalWords:  BJC
