\documentclass[fontsize=11pt]{article}

\usepackage[margin=1in]{geometry}
\usepackage{graphicx}
\usepackage{wrapfig}
\usepackage{subcaption}
\usepackage{amsmath} 
\usepackage{siunitx}
\usepackage{booktabs}
\usepackage{gensymb}
\usepackage{hyperref}
\usepackage[export]{adjustbox}
\usepackage{anyfontsize}

\begin{document}

	\graphicspath{{./figures/}}

	%BJC: check address, phone number and fax (?) number
	\begin{figure}
	\centering
	\begin{minipage}{0.37\textwidth}
	\includegraphics[width=2.14in,left]{CUBoulder.pdf}
	\end{minipage}
	\begin{minipage}{0.62\textwidth}
	\scriptsize
	\noindent Department of Chemical and Biological Engineering \hfill t 303-735-7860~~~~~~~~~~~~~~~~~~ \\
	\noindent 596 UCB \hfill f 303-492-8425~~~~~~~~~~~~~~~~~~ \\
	\noindent Boulder, Colorado 80309 \hfill michael.shirts@colorado.edu \\
	\end{minipage}
	\end{figure}
	
	%\noindent May 3\textsuperscript{rd}, 2019\\
	\noindent \today

	\noindent Dear Editors,\\
	
	\newcommand{\ManuscriptTitle}{Statistical Inference of Transport Mechanisms and
	Long Time Scale Behavior from Time Series of Solute Trajectories in 
	Nanostructured Membranes}
	
	We are submitting ``\ManuscriptTitle'' by Benjamin J. Coscia, Christopher P.
	Calderon and Michael R. Shirts, for	consideration for publication as a Research
	Article in the ``Machine Learning in Physical Chemistry" special issue of the Journal 
	of Physical Chemistry, specifically in part B, section B4: Fluid 
	Interfaces, Colloids, Polymers, Soft Matter, Surfactants, and Glassy Materials.
	We expect this work to be of significant interest to researchers who study complex 
	dynamics in both structured and amorphous soft materials.
	
	In this article, we apply the sticky hierarchical Dirichlet process autoregressive
	hidden Markov model (HDP-AR-HMM), a non-parametric Bayesian time series
	classification technique, to solute center-of-mass trajectories generated from
	5 $\mu s$ molecular dynamics (MD) simulations in a cross-linked inverted hexagonal
	phase lyotropic liquid crystal (LLC) membrane in order to automatically detect a
	variety of solute dynamical modes. We can use this information in order to 
	better understand the complex transport mechanisms exhibited by the solutes in 
	our simulation which will greatly increase the efficiency with which we 
	design LLC membranes for solute-specific separations.
	
	Our work represents a considerable advancement over our most recent study 
	characterizing the same set of solute center-of-mass time series trajectories
	(Coscia and Shirts, arXiv:2004.07905 [cond-mat.soft]). In that work, we used
	our previously gained qualitative understanding of solute transport (Coscia 
	and Shirts \textit{J. Phys. Chem. B}, 123, 6314--6330 (2019)) in order to 
	formulate stochastic models which could be used to project solute behavior
	on macroscopic time scales, thus allowing prediction of experimentally relevant
	properties like membrane selectivity. Using the HDP-AR-HMM, we can analyze solute
	trajectories with no prior knowledge of their dynamic behavior and easily produce
	models which quantitatively describe their long time scale dynamics. Since the 
	approach is non-parametric, the complexity of our data drives the complexity of
	the model. We cluster the parameter sets in order to reduce the total space of 
	dynamical states to a small set of distinct parameters which allow us to clearly
	understand physical	mechanisms in terms of solute-membrane interactions. To our 
	knowledge, this is the first application of the HDP-AR-HMM to MD data and it is
	the first time this framework has been used to gain both mechanistic and kinetic
	insight. 
	
	\noindent Some suggestions for reviewers are:
	\begin{enumerate}
	
		\item Francisco Hung has used molecular simulations to study nematic liquid crystals
		and transport under confinement within various nanostructured materials 
		(Northeastern University, 617-373-8619,\\ \href{mailto:f.hung@northeastern.edu}{f.hung@northeastern.edu}).
		
		\item Mahesh Mahanthappa has experience modeling soft materials using molecular dynamics
		including work on ion transport in various ordered liquid crystalline phases. 
		(University of Minnesota, 612-625-4599, \href{mailto:maheshkm@umn.edu}{maheshkm@umn.edu}).
		
		\item Stephen Paddison uses various levels of molecular simulation theory to understand
		proton transport in nanostructured and charged polymer membranes. (The University of Tennessee 
		Knoxville, 865-974-2026, \href{mailto:spaddison@utk.edu}{spaddison@utk.edu}).
		
		\item Venkat Ganesan uses multiscale modeling to study solute and ion transport in complex
		nanostructured morphologies in order to learn how to design novel materials. (The University
		of Texas at Austin, 512-471-4856, \href{mailto:venkat@che.utexas.edu}{venkat@che.utexas.edu}).
		
		%BJC: from structure paper
		\item Eric Jankowski has experience modeling soft materials using molecular dynamics.
		(Boise State University, 208-426-5681, \href{mailto:ericjankowski@boisestate.edu}{ericjankowski@boisestate.edu})
		
		\item Coray Colina uses various simulation techniques to understand structure-property relations
		in functional materials including nanoporous materials for separation applications. (University
		of Florida, 352-294-3488, \href{mailto:colina@chem.ufl.edu}{colina@chem.ufl.edu}).
		
%		\item Martin Lindén has use variational Bayesian treatments of hidden Markov models in
%		order to learn from thousands of single-particle trajectories.
%		
%		\item Steve Presse
%		
	\end{enumerate}
	
	\noindent Please send correspondence regarding this paper to Michael R. Shirts (contact
	details in letterhead).\\	
	
	\noindent Sincerely,
	
	\noindent Michael R. Shirts \\
	\noindent Benjamin J. Coscia \\
	\noindent Christopher P. Calderon \\
	
\end{document}

% LocalWords:  BJC
