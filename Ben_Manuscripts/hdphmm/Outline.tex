\documentclass{article}
\usepackage{graphicx}
\usepackage{wrapfig}
\usepackage{subcaption}
\usepackage[margin=1in]{geometry}
\usepackage{amsmath} % or simply amstext
\usepackage{amssymb}
\usepackage{siunitx}
\usepackage{booktabs}
\usepackage[export]{adjustbox}
\newcommand{\angstrom}{\textup{\AA}}
\newcommand{\colormap}{jet}  % colorbar to use
\usepackage{cleveref}
\usepackage{booktabs}
\usepackage{gensymb}
\usepackage{float}

\title{Statistical Inference of Transport Mechanisms and Long Time Scale Behavior from Time Series 
       of Solute Trajectories in Nanostructured Membranes.}

\author{Benjamin J. Coscia, Christopher P. Calderon \and Michael R. Shirts} 

\begin{document}

  \graphicspath{{./figures/}}
  \maketitle

  \section{Introduction}
  
  There is a need for highly selective membranes in order to perform efficient 
  separations of components of complex aqueous streams.
  %MRS2: add some more examples (When you get around to it).
  \begin{itemize}
    \item Organic micropollutants
    \item Desalination and boric acid removal from seawater.
    \item While many researchers focus on membrane permeability, we may be 
    able to reduce costs of commercial nanofiltration and reverse osmosis with
    higher selectivity.~\cite{werber_materials_2016}
  \end{itemize}

%  \noindent Amphiphilic molecules are capable of self-assembling into ordered nanostructures.
%  \begin{itemize}
%    \item 
%  \end{itemize}

  Lyotropic liquid crystals (LLC) are a class of amphiphilic molecules whose ordered phases
  can be cross-linked into mechanically strong membranes capable of highly selective
  separations.
  \begin{itemize}
    \item The shape of the LLC monomers and water content dictates the ordered phase 
    that they form. There are two phases of particular interest for membrane applications.
  	\item H\textsubscript{II} phase lyotropic liquid crystals are characterized by 
  	hexagonally packed, straight, pores while the Q\textsubscript{I} phase consists of
  	a tortuous network of 3D interconnected pores. 
  	\item In both cases the pores are uniform in size with radii on the order of 1 nm
    %MRS2: is it really a molecular weight cutoff?
    %BJC2: is molecular size better? Obviously there are other factors like shape, but I don't want to get into that
  	giving them a very strict molecular size cut-off.
	\item Additionally, they have the potential to disrupt conventional membrane separation
	techniques by being selective based not only on size and charge, but on chemical
	functionality as well.
	\item Their pores are lined with LLC monomer functional groups which can potentially
	be designed to interact with solutes in a chemically-specific manner
  \end{itemize}

  There are limits to what we can learn from experiment about LLC membrane design.
  \begin{itemize}
    \item Experimental observables like permeability and selectivity allow us to speculate
    about the molecular origins of separation processes.
    %MRS2: not quite sure what you mean below?
    \item This drives an empirical design approach which can potentially neglect key
    interactions which influence selectivity.
    \item LLC membranes have been shown to exhibit selectivities which cannot be fully 
    explained by relatively simple macroscopic models. 
  \end{itemize}
  
  Molecular Dynamics (MD) simulations can give us mechanistic insights with atomistic 
  resolution so that we can intelligently design new membranes for solute-specific 
  separations.
  \begin{itemize}
    \item In our previous work, we built a detailed atomistic model which we used
    to understand the nanoscopic structure of an LLC Membrane.~\cite{coscia_understanding_2019}
    \item We also used the model in order to gain a qualitative understanding of 
    trapping mechanisms which lead to subdiffusive transport behavior.~\cite{coscia_chemically_2019}
  \end{itemize}

  Unfortunately, the timescales that we can simulate with MD are insufficient to be
  able to make well-converged predictions of macroscopic transport properties 
  traditionally used to characterize membranes in the lab.
  \begin{itemize}
    %MRS2: would be good to get this next point into the first line of a paragraph, since it's quite important. 
    \item However, if we use descriptive stochastic models that can capture solute
    dynamics, then we could project long timescale behavior in addition to gaining
    a deeper understanding of solute behavior on short timescales.
  \end{itemize}
  
  In our previous work, we designed two different approaches which used
  solute time series in order to parameterize stochastic models that could be used
  to project transport on much longer timescales.
  \begin{itemize}
  	\item In our first approach we modeled solute trajectories as subordinated fractional
  	Brownian and L\'evy motion, called the anomalous diffusion (AD) model. 
  	\item We generated solute trajectories by generating a series
  	of anti-correlated hops separated by random periods of entrapment drawn from a 
  	power law distribution.
  	\item Our second approach treated solute motion as a Markov state model with
  	state-dependent dynamics, called the Markov state-dependent dynamical model (MSDDM).
  	\item We parameterized the state transition probabilities between
  	each of eight discrete states as well as the solute dynamics within each of these
  	states. We generated stochastic trajectory realizations by drawing a state
  	sequence based on the transition probability matrix and incorporating the state dynamics
  	while solutes were trapped in each state.
%  	\item Both models had moderate success reproducing the mean squared displacements (MSDs)
%  	exhibited by solutes in our MD simulations.
  \end{itemize}
  
  Although both models had reasonable success at predicting solute mean squared 
  displacements (MSDs) on MD simulation timescales, they had shortcomings.
  \begin{itemize}
  	\item The MSDDM failed to reproduce the hopping and trapping behavior that
  	characterizes solute center-of-mass trajectories in our MD simulations.
  	\item The AD model did not suffer this qualitative shortcoming, but the 
  	persistent curvature of the predicted MSD curves suggested that the model
  	might underestimate MSDs on long timescales. 
  	\item The formulation of both models required careful examination and
  	characterization of the interactions and dynamics exhibit by MD trajectories 
  	which required considerable human effort.
  \end{itemize}
  
  % BJC: Not sure whether to call it the infinite hidden markov model or hierarchical dirichlet process hidden markov model
  % The former is definitely simpler.
  % MRS: True.  I would probably see which is used more in the literature. 
  % BJC1: going with IHMM. HDPHMM seems to mostly be used by Emily Fox
  In this work, we apply the infinite hidden Markov Model (IHMM), a modeling
  approach that is agnostic to the source of time series data, in order to 
  automatically detect and infer the parameters of an unknown number of latent
  autoregressive (AR) modes present in solute center-of-mass time series.
  \begin{itemize}
  	\item In addition to AR parameters for each state, the IHMM estimates the
  	state transition probability matrix.
  	\item The model helps simultaneously uncover underlying transport mechanisms which
  	give rise to dynamical behavior and project that behavior on longer timescales so
  	that we can estimate macroscopic transport observables.
  \end{itemize}
  
  We use the parameters of the states identified by the IHMM in order to infer 
  dominant solute-membrane interactions and transport mechanisms.
  \begin{itemize}
    \item We compare the inferred mechanisms to those which we manually
    identified in our previous work. 
    %MRS2: don't put the conclusions in the introduction.  Instead, state clearly the questions you will ask and the hypotheses you will test.
   \item Some kind of conclusion here. Did we find more or less states. Any new states/ subdivisions of states?
  \end{itemize}
  
  We can also use the IHMM to generate stochastic trajectory realizations that share
  the same dynamical characteristics as solute trajectories observed in our MD 
  simulations. 
  \begin{itemize}
    \item The trajectories are qualitatively similar, showing expected hopping and trapping
    behavior.
    \item They are quantitatively similar in that they reproduce the MSDs measured in MD.
  \end{itemize}

  Finally, we use the stochastic trajectory realizations in order to compute the 
  macroscopic flux of each solute and selectivity of the LLC membranes studied towards
  each solute.
  \begin{itemize}
    %MRS2: shouldn't reach conclusions in intro, just set up questions to be answered (conclusions SHOULD be in abstract).
    \item We relate these macroscopic properties to our nanoscopic model by simulating 
    mean first passage time (MFPT).
    \item Some kind of conclusion. This membrane is selective towards solutes with this
    functionality. 
    \item Does the conclusion agree with our previous work? Any length dependence? (I think not)
  \end{itemize}
    
  \section{Methods}
    
  We ran all MD simulations and energy minimizations using GROMACS 2018. We
  performed all post-simulation trajectory using python scripts which are available
  online at \\ \texttt{https://github.com/shirtsgroup/LLC\_Membranes}.

  \subsection{Molecular Dynamics Simulations}
  
  % BJC: Could I just reference previous work here? "MD simulations were run as 
  % described in our previous work..."

  We studied transport of solutes in the H\textsubscript{II} phase using an
  atomistic molecular model of four pores in a monoclinic unit cell with 
  10 \% water by weight. 
  \begin{itemize}
    \item Approximately one third of the water molecules occupy the tail region 
    with the rest near the pore center.
    \item We chose to study the 10 wt \% water system because solutes move 
    significantly faster than in the 5 wt \% system studied previously.
    \item Appropriate stochastic modeling requires that solutes sample the 
    accessible mechanisms with representative probability.%MRS: only possible on the ~1-10 \mu scale with relatively fast solutes.
  \end{itemize}
  
  We chose to study a subset of 4 of the fastest moving solutes from our previous
  work: methanol, acetic acid, urea and ethylene glycol.
  \begin{itemize} 
    \item In addition to exploring membrane structural space the most, these solutes
    have a relatively diverse set of chemical functionality.   
    \item For each solute we created a separate system and to each system we
    added 6 solutes per pore for a total of 24 solutes.
    \item This number of solutes per pore provides a balance of a low 
    degree of interaction between solutes and sufficient amount of data from
    which to generate statistics on the time scales which we simulate.
    \item Further details on the setup and equilibration of these systems can
    be found in our previous work.\cite{coscia_chemically_2019}
  \end{itemize}
  
  \noindent We extended the 1 $\mu$s simulations of our previous work to 5 $\mu$s in order
  to collect ample data.
  \begin{itemize}
    \item We simulated the system with a time step of 2 fs at a pressure of 1 bar
    and 300 K controlled by the Parinello-Rahman barostat and the v-rescale thermostat
    respectively.
    \item We recorded frames every 0.5 ns
  \end{itemize}

  \subsection{The Infinite State Hidden Markov Model}\label{method:IHMM}
  %BJC: There is a fair amount of detail here just so I could understand better by writing it out
  
  Hidden Markov models (HMMs) are a useful and widely used technique
  for modeling sequences of observations where the probability of the next observation
  in a sequence depends on a previous unobserved, latent or hidden, state.~\cite{beal_infinite_2002}
  %MRS2: it can depend on other things, but ALSO depends on the latent state.
  \begin{itemize}
    \item In the context of our simulations, the observations correspond to 
    the center of mass coordinates of the solutes versus time, and the states
    correspond to the dynamical behavior which give rise to those types
    of observations.
    \item Unfortunately, standard HMMs require the number of hidden states to be known
    a priori.
    \item One can partially overcome this by testing a range of numbers of 
    hidden states and determining which is the best representation of the
    data.
  \end{itemize}
  
  %BJC: I'm not sure how much of this I should include. This is pretty much 
  % a reproduction of Fox et al's explanation: https://ieeexplore.ieee.org/abstract/document/5563110
  The infinite-state HMM overcomes this drawback by placing a hierarchical
  Dirichlet process (HDP) prior on the transition probabilities.
  \begin{itemize}
    \item Using some base probability distribution, H, a Dirichlet process 
    (DP) generates distributions over a countably infinite number of 
    probability measures:
    \begin{equation}
      G_0 = \sum_{k=1}^{\infty} \beta_k \delta_{\theta_k} ~~ \theta_k \sim H, \beta \sim GEM(\gamma)
    \end{equation}
    where the $\theta_k$ are values drawn from the base distribution and the
    weights $\beta_k$ come from a stick-breaking process parameterized by the concentration 
    parameter $\gamma$ (equivalently referred to as GEM($\gamma$)). 
    \item The concentration parameter expresses one's confidence in H relative to the posterior 
    and is closely related to the number of data observations.
    \item Each row, $G_j$, of the transition matrix is produced by drawing from a DP specified 
    using the $\beta$ vector as a discrete base distribution and a separate concentration
    parameter, $\alpha$.
    \begin{equation}
      G_j = \sum_{k=1}^{\infty} \pi_{jk} \delta_{\theta_k} ~~ \pi_j \sim DP(\alpha, \beta)
    \end{equation}
    \item This hierarchical specification ensures that the transition probabilities in 
    each row share the same support points \{$\theta_1$, ..., $\theta_k$\}.
    \item Once the model has converged only a finite number of states will have significant
    sampling.
  \end{itemize}
  
  %BJC: I do need to include this
  \noindent We describe the dynamics of each state using a vector autoregressive (VAR) model. 
  \begin{itemize}
  	\item A VAR($r$) process is characterized by a vector of observations in a time series 
  	that are linearly dependent on $r$ previous values of the time series vector:
  	\begin{equation}
  	\mathbf{y}_t = \mathbf{c} + \sum_{i=1}^r A_i\mathbf{y}_{t-i} + \mathbf{e}_t~~~~\mathbf{e}_t \sim N(0, \Sigma)
  	\end{equation}
  	Previous observations are weighted by coefficient matrices $A_i$. The VAR($r$) 
  	process is further characterized by a shift in the mean of each dimension by the
  	vector $\mathbf{c}$ and a white noise term $\mathbf{e}_t$.~\cite{hamilton_time_1994}
        %MRS2: mean is zero in Z, is it necessarily in R (I realize we've talked about this before, but wanted to get the clarification in text)
  	\item We assumed multivariate Gaussian noise, with mean zero and covariance, $\Sigma$.
  	\item We limited our analysis to an autoregressive order of $r=1$. %BJC: might try higher orders, but probably not worth it
  	\item We used a conjugate matrix-normal inverse-Wishart prior on parameters
  	$A$ and $\Sigma$ and a conjugate Gaussian prior on $\mathbf{c}$ in order to analytically
  	draw from the posterior.~\cite{fox_nonparametric_2009}
  \end{itemize}   
  
  %Based on the algorithms developed by Fox et al.~\cite{fox_sticky_2007}, 
  Using the IHMM framework, we estimated the most likely number and sequence of hidden states
  while simultaneously estimating VAR(1) parameters for each state and the overall 
  state transition probability matrix, $T$.
  \begin{itemize}
    \item We created a python implementation of this process which we heavily adapted from
    the MATLAB code of Fox et al.~\cite{fox_sticky_2007} 
    \item Parameter estimation is iterative. Therefore, we looked for convergence 
    as shown in SI.
    \item We refer the interested reader to much more extensive descriptions of 
    this process and its implementation. 
    ~\cite{beal_infinite_2002,teh_hierarchical_2006,van_gael_beam_2008,fox_nonparametric_2009,fox_bayesian_2010}
  \end{itemize}
  
  We applied the IHMM algorithm to the radial and axial coordinates of each 
  solute center-of-mass trajectory.
  \begin{itemize}
    \item We measured the radial coordinate as the distance from the closest
    pore center.
    \item Because the pores are somewhat tortuous, we approximated the pore 
    center with a spline. See previous work.
    %MRS2: approximated the pore center line with a spline?
  \end{itemize}
  
  We clustered like parameter sets in order to reduce the state space to
  a more easily interpretable size.
  \begin{itemize}
    %MRS2: can you enforce that the mean in the z-component is zero?
    %MRS2: I'm trying to think what other constraints should be enforced. 
    %MRS2: the issue with state clustering is that when transitionin between two states that are clustered, 
    %MRS2: it destroys the correlation that would have been within if it was all the same statee.
    %MRS2: I'm not sure what the implications are. If the simulations are RUN with grouped states, 
    %MRS2: then the correlation can be reintroduced, so maybe it doesn't matter?
   \item Since the mean vector, $\mathbf{c}$ of each state can take on 
    continuous values, the IHMM algorithm tends to find a large number of states,
    with very few states revisited.
    % BJC1: This will need to be updated depending on how I cluster. Might need to describe
    % multiple clustering approaches.
    \item However, $\mathbf{A_i}$ and $\mathbf{\Sigma}$ tend to share similarities with
    independently identified states. Therefore, we clustered based on $\mathbf{A_i}$
    and $\mathbf{\Sigma}$, ignoring the mean.
    \item Since  we do not know the number of similar states beforehand, we used
    a non-parametric Bayesian Gaussian mixture model % cite scikit-learn
    \item We used the clustered parameters in order to relabel states in the estimated
    state sequences and to recalculate the state transition probability matrix.
  \end{itemize}
  
  We generated stochastic trajectory realizations by drawing state sequences 
  based on the rows of $T$.
  \begin{itemize}
    \item While in a given state, we simulated motion according to the VAR(1)
    parameterization of that state.
    \item We set the unconditional mean of each state based on the position 
    before the state transition occurred.
  \end{itemize} 
  
  \subsection{Estimating Flux and Selectivity}
  %BJC: can make this discussion somewhat short since it's already in previous paper. 
  % Should probably just hit all the main equations quickly.
  
  \noindent We calculate first passage times by propagating stochastic trajectories until they
  reach distance $L$. \\
  
  We determine the mean first passage time (MFPT) using the following equation:~\cite{cussler_diffusion:_2009}
  \begin{equation}
  P(t) = -\frac{1}{\sqrt{\pi}}e^{-(L - vt)^2 / (4Dt)}\bigg(-\frac{D(L - vt)}{4(Dt)^{3/2}} - \frac{v}{2\sqrt{Dt}}\bigg)
  \label{eqn:passage_times}
  \end{equation}
  
  \noindent Flux, $J$, is simply 1 / MFPT by the Hill relation.~\cite{hill_free_1989} \\
  
  In our previous work, we showed that, in the absence of convective solute flux, selectivity
  towards solute $i$ versus solute $j$ can be calculated by:  
  \begin{equation}
  S_{ij} = \frac{J_i / \Delta C_i}{J_j / \Delta C_j}
  \label{eqn:selectivity}
  \end{equation}
  where $\Delta C_j$ is the trans-membrane concentration difference.

  \section{Results and Discussion}
  %MRS2: how stable are the results to realizations of the system
  \subsection{Inferring Solute Transport Mechanisms}
  
  Clustering parameters sets results in X distinct dynamical modes.
  \begin{itemize}
    \item In the figure below, we show time series simulations that qualitatively
    illustrate the difference in dynamical behavior between modes.
  \end{itemize}
   
  %BJC1: a figure showing some representative fluctuations in each mode

  \noindent We can relate the identified states back to transport mechanisms.
  \begin{itemize}
  	\item More detailed discussion of identified states
  	\item How size of fluctuations, autoregressive parameters are influenced by trapping mechanisms
  	\item How do these states compare to those identified in our previous work?
  	\item Any new states?
  \end{itemize}
  
  \subsection{Reproducing MD Trajectories and MSDs with the IHMM}
  
  \noindent Trajectory realizations qualitatively match MD simulation trajectories.

  \begin{itemize}
    %MRS2: think about how to quantify the simiilarity of the hopping and trapping behavior.
    \item Look for hopping and trapping behavior
  \end{itemize}
  
  \noindent MSDs generated from stochastic trajectories match those from MD.
  \begin{itemize}
    %MRS2: again, think about the best ways to quantify similar between MD and stochastic trajectories.
  	\item Look at curvature and 1-$\sigma$ confidence intervals
  \end{itemize}
  
  \subsection{Estimating Solute Flux and Selectivity}  
  
  \noindent We can predict macroscopic flux and selectivity.
  \begin{itemize}
  	\item Flux as function of pore length
  	\item Selectivity as function of pore length (if flux scaling is length-dependent)
  \end{itemize}
 
  \section{Conclusion}
  
  \noindent We have shown that the IHMM can be used to parameterize solute time series
  with an unknown number of latent dynamical modes. \\
  
  \noindent We can use the IHMM to help identify mechanisms by relating the latent
  states to observed solute behavior. \\
  
  \noindent We can use the IHMM to predict macroscopic transport properties. \\
  
  %MRS2: below probably doesn't need to be stated?
  \noindent The IHMM is not limited to the H\textsubscript{II} phase.
  
  \section*{Supporting Information}

  Detailed explanations and expansions upon the results and procedures mentioned in
  the main text are described in the Supporting Information. This information is
  available free of charge via the Internet at http://pubs.acs.org.

  \section*{Acknowledgements}

  %MRS2: add GAANN and PRF here to cover bases.
  This work was supported in part by the ACS Petroleum Research Fund
  grant \#59814-ND7 and the Graduate Assistance in Areas of National Need (GAANN) 
  fellowship which is funded by the U.S. Department of Education. 
  Molecular simulations were performed using the Extreme Science and
  Engineering Discovery Environment (XSEDE), which is supported by National
  Science Foundation grant number ACI-1548562. Specifically, it used the Bridges
  system, which is supported by NSF award number ACI-1445606, at the Pittsburgh
  Supercomputing Center (PSC). This work also utilized the RMACC Summit supercomputer,
  which is supported by the National Science Foundation (awards ACI-1532235 and
  ACI-1532236), the University of Colorado Boulder, and Colorado State
  University. The Summit supercomputer is a joint effort of the University of
  Colorado Boulder and Colorado State University.

  \clearpage

  \bibliographystyle{ieeetr}
  \bibliography{hdphmm}

  %\newpage

  %\section*{TOC Graphic}

\end{document}
