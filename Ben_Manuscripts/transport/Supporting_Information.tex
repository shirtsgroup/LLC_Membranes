\documentclass{article}
\usepackage{graphicx}
\usepackage{wrapfig}
\usepackage{subcaption}
\usepackage[margin=1in]{geometry}
\usepackage{amsmath} % or simply amstext
\usepackage{siunitx}
\usepackage{booktabs}
\usepackage[export]{adjustbox}
\newcommand{\angstrom}{\textup{\AA}}
\newcommand{\colormap}{jet}  % colorbar to use
\usepackage{cleveref}
\usepackage{booktabs}
\usepackage{gensymb}
\usepackage{float}

\renewcommand{\thefigure}{S\arabic{figure}}
\renewcommand{\thesection}{S\arabic{section}}
\renewcommand{\thepage}{S\arabic{page}}
\renewcommand{\thetable}{S\arabic{table}}

\title{Supporting Information: The Transport Mechanisms of Polar Solutes in a Cross-linked H\textsubscript{II} Phase Lyotropic Liquid Crystal Membrane}
\author{Benjamin J. Coscia \and Michael R. Shirts} 

\begin{document}

  \maketitle
  \graphicspath{{./supporting_figures/}}
  \bibliographystyle{ieeetr}

  \section{Water content equilibration}\label{section:water_content_equil}

  We initially tried to equilibrate our system with water by allowing water
  molecules to naturally penetrate the membrane from a water bath separating
  periodic images of the system in the $z$-direction (See Figure~\ref{fig:gap}).
  We allowed a dry, previously equilibrated system to further equilibrate in
  coexistence with a 3 nm-thick (in the $z$-direction) layer of water. Water
  readily enters the tail region where the density of monomers is low. About 3
  times more water molecules occupy the tail region after 1000 ns of
  equilibration (See Figure~\ref{fig:equilibrated_water_penetration}). Although
  the water level in the pore appears to plateau in this system, it is clear that
  equilibration of this system is kinetically limited since water does not fill
  the pores uniformly. The density of water along the pore axis, averaged over
  the last 50 nanoseconds of simulation, is close to zero at the membrane center.
  Therefore, we required a different equilibration technique in order to overcome
  the kinetic limitation.

  \begin{figure}[!htb]
  \centering
  \begin{subfigure}{0.18\textwidth}
  \includegraphics[width=\linewidth]{gap.pdf}
  \caption{}\label{fig:gap}
  \end{subfigure}  
  \begin{subfigure}{0.37\textwidth}
% Generated with : solute_partitioning.py -t shifted.xtc -g berendsen.gro -buffer 2 -r SOL
% in directory : /home/bcoscia/Documents/Gromacs/equilibration/solvation_equilibration/NaGA3C11/gaps/pre_equilibrated
  \includegraphics[width=\linewidth]{equilibrated_water_penetration.pdf}
  \caption{}\label{fig:equilibrated_water_penetration}
  \end{subfigure}  
  \begin{subfigure}{0.37\textwidth}
% Generated with custom script density.py located in same directory as above.
  \includegraphics[width=\linewidth]{penetration_density.pdf}
  \caption{}\label{fig:penetration_density}
  \end{subfigure}  
  \caption{(a) Using an equilibrated dry configuration, we inserted a layer of water between
           periodic copies of the system in the $z$-direction. (b) Water slowly enters 
	   the membrane. Most water enters the tail region where the density of monomers is
	   lowest. Water entering the pore plateaus after 500 ns. (c) Although the water
	   content of the pore appears equilibrated in (b), the density of water throughout
	   the pores is not uniform, with almost no water close to the pore center.
  }\label{fig:gap_solvation}
  \end{figure}

  We equilibrated 4 systems where we initially placed water in the pores and in
  the tails along with a water reservoir in between periodic images, much like
  Figure~\ref{fig:gap}. We tested a diverse set of systems with varying total
  water contents and ratios of pore : tail water contents. The intial pore radius
  dictates the water content of the pore if one is to avoid vacuum gaps. We
  explored systems with initial pore radii of 5, 6, 7 and 8 \AA. Table \_\_ shows
  the water composition of the pores and the tails for each system studied. In
  systems started with more water in the tails, the pore water tends to increase
  over time, while that of the tails decreases or stays stable. Systems started
  with more water in the pores tend to plateau relatively quickly, with $\sim$
  one third of the water staying in the tails.   

  \begin{table}
  \centering
  \begin{tabular}{|c|c|c|}
  \hline
  Pore Radius & wt \% water tails & wt \% water pores \\
  \hline
  5           &        5.67       &     1.09          \\
  6           &        2.88       &     2.38          \\
  7           &        1.91       &     4.12          \\
  8           &        2.78       &     6.00          \\
  \hline
  \end{tabular}
  \caption{}\label{table:water_content}
  \end{table}

  \begin{figure}
  \centering
  \begin{subfigure}{0.45\textwidth}
% Generated by running: solute_partitioning.py -t stabilized.xtc -g PR.gro -buffer 3 --savename water_partition.pl
% In directory: /home/bcoscia/Documents/Gromacs/equilibration/solvation_equilibration/NaGA3C11/water_content_experiments/5/1.09_5.67
  \includegraphics[width=\linewidth]{r5_gap.pdf}
  \caption{r = 5\AA}\label{fig:r5_gap}
  \end{subfigure}
  \begin{subfigure}{0.45\textwidth}
% Generate by running: solute_partitioning.py -t stabilized.xtc -g PR.gro -buffer 2.5 -r SOL --savename water_partition.pl
% In directory: /home/bcoscia/Documents/Gromacs/equilibration/solvation_equilibration/NaGA3C11/water_content_experiments/6/2.38_2.88
  \includegraphics[width=\linewidth]{r6_gap.pdf}
  \caption{r = 6\AA}\label{fig:r6_gap}
  \end{subfigure}
  \begin{subfigure}{0.45\textwidth}
% Generate by running: solute_partitioning.py -t stabilized.xtc -g last.gro -buffer 3 -r SOL --savename water_partition.pl
% In directory: /home/bcoscia/Documents/Gromacs/equilibration/solvation_equilibration/NaGA3C11/water_content_experiments/6/4.12_1.91
  \includegraphics[width=\linewidth]{r7_gap.pdf}
  \caption{r = 7\AA}\label{fig:r7_gap}
  \end{subfigure}
  \begin{subfigure}{0.45\textwidth}
% Generate by running: solute_partitioning.py -t stabilized.xtc -g last.gro -buffer 3 -r SOL --savename water_partition.pl
% In directory: /home/bcoscia/Documents/Gromacs/equilibration/solvation_equilibration/NaGA3C11/water_content_experiments/8/6.00_2.78
  \includegraphics[width=\linewidth]{r8_gap.pdf}
  \caption{r = 8\AA}\label{fig:r8_gap}
  \end{subfigure}
  \caption{}\label{fig:gap_prefilled_equil}
  \end{figure}

  Since the equilibrium water content is unclear based on the previous
  simulations, we elected to choose and study systems with two different water
  contents. We removed the water reservoir and allowed the pore and tail water
  contents to equilibrate with 5 and 10 wt \% water total. We placed one third of
  the total water needed in the tails. We considered the water content
  equilibrated once the water contents plateaued. The 10 wt \% water system
  equilibrated within the first 100 ns of simulation, while the 5 wt\% system did
  not plateau until $\sim$ 600 ns (See Figure~\ref{fig:solvation_equilibration}).
  The pores contain 72 \% and 69 \% of the total water in the 5 and 10 wt \%
  systems respectively.

  \begin{figure}
  \centering
  \begin{subfigure}{0.45\textwidth}
% Generated with solute_partitioning.py -t full_shifted.xtc -g berendsen.gro -r SOL -pr 1.6 --savename water_partitioning.pl
% in directory /home/bcoscia/Documents/Gromacs/equilibration/solvation_equilibration/NaGA3C11/offset/5wt
  \includegraphics[width=\textwidth]{5wt_offset_equil.pdf}
  \caption{}\label{fig:5wt_offset_equil}
  \end{subfigure}
  \begin{subfigure}{0.45\textwidth}
  \includegraphics[width=\textwidth]{10wt_offset_equil.pdf}
  \caption{}\label{fig:10wt_offset_equil}
% Generated with solute_partitioning.py -t shifted.xtc -g last.gro -r SOL -pr 1.48 --savename water_partitioning.pl
% in directory /home/bcoscia/Documents/Gromacs/equilibration/solvation_equilibration/NaGA3C11/offset/10wt
  \end{subfigure}
  \caption{We created solvated systems with one third of the total water
	  initially placed in the tail region. (a) With 5 wt \% total water, the water
	  content equilibrates after 600 ns, with $\sim$ 72 \% of the total water in the
	  pores. (b) With 10 wt \% total water, the water content equilibrates after 100
	  ns, with $\sim$ 69 \% the total water in the
	  pores.}\label{fig:solvation_equilibration}
  \end{figure}

  We cross-linked the equilibrated solvated systems, then allowed them to 
  equilibrate further for 100 ns. 

  \begin{figure}
  \centering
  \begin{subfigure}{0.45\textwidth}
  % To be made (10 wt % system in place for now)
  \includegraphics[width=\textwidth]{10wt_offset_xlinked_equil.pdf}
  \caption{}\label{fig:5wt_offset_xlinked_equil}
  \end{subfigure}
  \begin{subfigure}{0.45\textwidth}
  \includegraphics[width=\textwidth]{10wt_offset_xlinked_equil.pdf}
  \caption{}\label{fig:10wt_offset_xlinked_equil}
% Generated with solute_partitioning.py -t xlinked_equil.trr -g xlinked_equil.gro -r SOL -pr 1.48 --savename water_partitioning.pl
% In directory /home/bcoscia/Documents/Gromacs/equilibration/xlink_equilibration/NaGA3C11/63_percent
  \end{subfigure}
  \caption{}\label{fig:solvation_equilibration}
  \end{figure}

  \section{Choosing a transport model}\label{section:transport_model_selection}

  We used the toolbox created by Meroz and Sokolov in order to justify our
  choice of transport model.\cite{meroz_toolbox_2015} The solutes in our systems
  exhibit anomalous transport properties characteristic of a Continuous Time
  Random Walk (CTRW). 

  \subsection*{Mean Squared Displacement}

  The general form of a mean squared displacement (MSD) curve is:
  \begin{equation}
	\langle x^2(t) \rangle \sim t ^ \alpha
	\label{eqn:msd}
  \end{equation}
  For brownian motion, $\alpha = 1$ and the MSD is linear. When $\alpha \neq
  1$, the particle of interest exhibits anomalous diffusion. Values of $\alpha$
  greater than 1 give rise to superdiffusion, while values of $\alpha$ less than
  1 give rise to subdiffusion.

  We can calculate the ensemble-averaged MSD curve by averaging the MSDs of
  each particle trajectory, where each MSD is calculated using:
  \begin{equation}
	\delta^2(t) = \| \mathbf{r}(t) - \mathbf{r}(0) \|^2
	\label{eqn:ensemble_msd}
  \end{equation}
  where $\|\cdot\|$ represents the Euclidean norm. 

  The mean squared displacement of solutes in our model is a non-linear
  function of time, with $\alpha < 1$ which is indicative of anomalous
  subdiffusion. Figure \ref{fig:msd_power_law}a plots the ensemble-averaged MSD
  curve for 24 ethanol molecules diffusing in a 10 wt\% water H\textsubscript{II}
  LLC membrane system. We fit a power law of the form $Ae^{\alpha}$ to the MSD
  curve. We performed 2000 bootstrap trials by randomly sampling 24 MSD curves
  with replacement from the 24 total ethanol MSD curves. The bootstrapped average
  value of $\alpha$ is 0.75 for this system. 
 
  \begin{figure}[!htb]
  \centering
% Generated with : msd.py -t PR_nojump.xtc -g PR.gro -r ETH -ensemble -power_law -a z -nboot 2000
% in directory: /home/bcoscia/Documents/Gromacs/Transport/NaGA3C11/ETH/10wt
  \includegraphics[width=0.8\linewidth]{msd_power_law.pdf}
  \caption{(a) We fit a curve with the form of Equation~\ref{eqn:msd} to the
	  ensemble-averaged MSD curve. (b) The average value of $\alpha$, obtained using
	  fits to MSDs calculated from bootstrapped ensembles, is less than 1 suggesting
	  that ethanol molecules in our model exhibit subdiffusive
	  behavior.}\label{fig:msd_power_law}
  \end{figure}

  \subsection*{Ergodicity}

  The ergodicity of a system can help us narrow down the possible anomalous
  diffusion mechanisms. In an ergodic system, the time-averaged behavior of an
  observable should yield the same result as the ensemble average of the same
  observable. Examples of anomalous diffusion processes that are ergodic include
  random walks on fractals (RWF) and fractional brownian motion (FBM).
  Non-ergodic systems generally give rise to CTRWs with the possibility of
  combination with a RWF and/or FBM.\cite{meroz_toolbox_2015} 

  We tested the ergodicity of our system by comparing the ensemble-averaged
  and time-averaged MSD curves. We calculated the MSD of each ethanol trajectory
  using Equation~\ref{eqn:ensemble_msd} and a time-averaged algorithm: 
  \begin{equation}
	\delta^2(t) = \dfrac{1}{N-t} \sum_{i=0}^{N-t-1} \| \mathbf{r}(i + t) - \mathbf{r}(i) \|^2
  \end{equation}
  where N is the total number of simulation frames, and t represents the length
  of subinterval or number of frames per subinterval. We averaged the MSD curves
  from each trajectory in order to create final MSD plots.

  The ethanol molecules exhibit non-ergodic behavior because their
  time-averaged and ensemble-averaged MSDs do not agree with each other
  (Figure~\ref{fig:ethanol_msd_comparison}). We validated our analysis using a 1
  ns simulation of a box of tip3p water molecules. As expected, since the
  particles exhibit Brownian motion, the time-averaged and ensemble-averaged MSDs
  agree with each within error (Figure~\ref{fig:water_box_msd_comparison}).

  \begin{figure}[!htb]
  \centering
  \begin{subfigure}{0.45\textwidth}
% Generated with : msd.py -t PR_nojump.xtc -g PR.gro -r ETH -compare -nboot 2000 -a z
% in directory: /home/bcoscia/Documents/Gromacs/Transport/NaGA3C11/ETH/10wt
  \includegraphics[width=\textwidth]{ethanol_msd_comparison.pdf}
  \caption{}\label{fig:ethanol_msd_comparison}
  \end{subfigure} 
  \begin{subfigure}{0.45\textwidth}
% Generated with msd.py -t traj_nojump.xtc -g npt.gro -r SOL -compare --fracshow 0.4 -nboot 2000 -a z
% in directory: /home/bcoscia/Documents/Gromacs/Transport/Solvent/solvent_boxes/pure_water
  \includegraphics[width=\textwidth]{water_box_msd_comparison.pdf}
  \caption{}\label{fig:water_box_msd_comparison}
  \end{subfigure} 
  \caption{(a) The time-averaged and the ensemble-averaged MSDs for ethanol in
	  an H\textsubscript{II} nanopore are not in agreement, implying non-ergodicity.
	  (b) A box of tip3p water molecules is expected to be ergodic and it is shown to
	  be true here because both MSDs are in agreement. }\label{fig:msd_comparison}
  \end{figure}

  \subsection*{Autocorrelation of steps}

% From Sokolov paper: "Assigning different waiting times τ i to each step, and assuming that the
% steps are uncorrelated as in a regular RW, gives rise to the CTRW model" -- I
% might need to check autocorrelation of steps lengths when a hop occurs rather
% than every time step. Not sure if there is enough data for that, but could check 
% a particularly hoppy trajectory after longer simulation.


  Based on the previous two sections, our model can likey be studied as a CTRW. 
  However, it is still possible that our CTRW model might also be convoluted with
  an FBM or a RWF process. In a pure CTRW, the steps are uncorrelated. 
  Both FBM and RWF exhibit anti-correlated steps. 

  The steps in our system are not correlated. We showed this by calculating the
  autocorrelation function (ACF) of the step lengths in the $z$-direction. The
  ACF of a representative trajectory is shown in Figure~\ref{fig:eth_autocorrelation}.  
  
  \begin{figure}[!htb]
  \centering
% Generated with brownian_test.py -t PR_nojump.xtc -g PR.gro -r ETH (and appropriate uncommenting -- need to rework that script)
% in directory: /home/bcoscia/Documents/Gromacs/Transport/NaGA3C11/ETH/10wt
  \includegraphics[width=0.8\textwidth]{eth_autocorrelation.pdf}
  \caption{The autocorrelation function (right) of a representative ethanol
	   center of mass $z$-coordinate trajectory (left) almost immediately decays to zero,
	   indicating a complete loss of memory of it's previous position. Noise increases
	   at large time lags due to decreased sampling.}\label{fig:eth_autocorrelation}
  \end{figure}

  \clearpage
  \bibliography{transport}

\end{document}
