\documentclass[fontsize=11pt]{article}

\usepackage[margin=1in]{geometry}
\usepackage{graphicx}
\usepackage{wrapfig}
\usepackage{subcaption}
\usepackage{amsmath} 
\usepackage{siunitx}
\usepackage{booktabs}
\usepackage{gensymb}
\usepackage{hyperref}
\usepackage[export]{adjustbox}
\usepackage{anyfontsize}

\begin{document}

	\graphicspath{{./figures/}}

	%BJC: check address, phone number and fax (?) number
	\begin{figure}
	\centering
	\begin{minipage}{0.37\textwidth}
	\includegraphics[width=2.14in,left]{CUBoulder.pdf}
	\end{minipage}
	\begin{minipage}{0.62\textwidth}
	\scriptsize
	\noindent Department of Chemical and Biological Engineering \hfill t 303-735-7860~~~~~~~~~~~~~~~~~~ \\
	\noindent 596 UCB \hfill f 303-492-8425~~~~~~~~~~~~~~~~~~ \\
	\noindent Boulder, Colorado 80309 \hfill michael.shirts@colorado.edu \\
	\end{minipage}
	\end{figure}
	
	\noindent May 3\textsuperscript{rd}, 2019\\

	\noindent Dear Editors,\\
	
	\newcommand{\ManuscriptTitle}{Chemically Selective Transport in a Cross-linked 
	H\textsubscript{II} Phase Lyotropic Liquid Crystal Membrane}
	
	We are submitting ``\ManuscriptTitle'' by Benjamin J. Coscia and Michael R. Shirts, for
	consideration for publication as an Article in the Journal of Physical Chemistry
	B. We expect this work to be of significant interest to researchers who study 
	nanostructured polymer membranes for aqueous separations.
	
	In this article, we use molecular dynamics (MD) simulations in order to observe 
	transport of small solutes in an inverted hexagonal (H\textsubscript{II}) phase 
	self-assembled lytropic liquid crystal (LLC) polymer membrane. Recently, we 
	published our first article on this subject in which we built and characterized
	an atomistic molecular model of an LLC membrane that is maximally consistent with experimental
	observations. By observing transport of solutes within a similar model, we can 
	learn the optimal ways to tune the shape, size and chemical functionality
	of LC monomers in order to intelligently design the pore environment and facilitate
	solute-specific separations.
	
	In this work, we observe and quantify transport of water, sodium and 20 small
	polar solutes within the pores of our atomistic H\textsubscript{II} phase LLC membrane
	model. In general, all solutes exhibit subdiffusive transport behavior
	characterized by intermittent hops between long periods of entrapment. However, 
	due the pores' inhomogeneous architecture, we observed 3 different mechanisms
	of entrapment. First, solutes can diffuse out of the pores and become entangled 
	between tails. Second, many of the solutes are capable of donating hydrogen bonds
	to monomer head groups for extended periods of time. Finally, solutes can associate
	with monomer head group counter-ions within nanopores. Each solute is influenced 
	by each mechanism to varying degrees dependent on its size and chemical 
	functionality. We begin our discussion with a broad description of these mechanisms
	in the context of all solutes before restricting the discussion to more detailed 
	analyses of subsets of chemically-similar solutes. 
	
	%BJC: included phone numbers because it asked to do so in author's guide.
	\noindent Some suggestions for reviewers are:
	\begin{enumerate}

		\item Menachem Elimelech is a leader in the development of membrane technologies
		for	water desalination and water reuse (Yale University, 203-432-2789,
		\href{mailto:menachem.elimelech@yale.edu}{menachem.elimelech@yale.edu}).
	
		\item Francisco Hung has used molecular simulations to study nematic liquid crystals
		and various	nanostructured materials (Northeastern Univerity, 617-373-8619 	
		\href{mailto:f.hung@northeastern.edu}{f.hung@northeastern.edu})
		
		\item Eric Jankowski has experience modeling soft materials using molecular dynamics
		including work done simulating structure factors from simulations, much like
		the techniques we use in our work (Boise State University, 208-426-5681,
		\href{mailto:ericjankowski@boisestate.edu}{ericjankowski@boisestate.edu})
		
	\end{enumerate}
	
	\noindent Please send correspondence regarding this paper to Michael R. Shirts (contact
	details in letterhead).\\	
	
	\noindent Sincerely,
	
	\noindent Michael R. Shirts \\
	\noindent Benjamin J. Coscia \\
	
\end{document}

% LocalWords:  BJC
